\documentclass{article}
\usepackage[utf8]{inputenc}
\usepackage{mathrsfs}
\usepackage[spanish,es-lcroman]{babel}
\usepackage{amsthm}
\usepackage{amssymb}
\usepackage{enumitem}
\usepackage{graphicx}
\usepackage{caption}
\usepackage{float}
\usepackage{amsmath,stackengine,scalerel,mathtools}
\usepackage{xparse, tikz-cd, pgfplots}
\usepackage{xstring}
\usepackage{mathrsfs}
\usepackage{comment}
\usepackage{faktor}


\def\subnormeq{\mathrel{\scalerel*{\trianglelefteq}{A}}}
\newcommand{\Z}{\mathbb{Z}}
\newcommand{\La}{\mathscr{L}}
\newcommand{\crdnlty}[1]{
	\left|#1\right|
}
\newcommand{\lrprth}[1]{
	\left(#1\right)
}
\newcommand{\lrbrack}[1]{
	\left\{#1\right\}
}
\newcommand{\lrsqp}[1]{
	\left[#1\right]
}
\newcommand{\descset}[3]{
	\left\{#1\in#2\ \vline\ #3\right\}
}
\newcommand{\descapp}[6]{
	#1: #2 &\rightarrow #3\\
	#4 &\mapsto #5#6 
}
\newcommand{\arbtfam}[3]{
	{\left\{{#1}_{#2}\right\}}_{#2\in #3}
}
\newcommand{\arbtfmnsub}[3]{
	{\left\{{#1}\right\}}_{#2\in #3}
}
\newcommand{\fntfmnsub}[3]{
	{\left\{{#1}\right\}}_{#2=1}^{#3}
}
\newcommand{\fntfam}[3]{
	{\left\{{#1}_{#2}\right\}}_{#2=1}^{#3}
}
\newcommand{\fntfamsup}[4]{
	\lrbrack{{#1}^{#2}}_{#3=1}^{#4}
}
\newcommand{\arbtuple}[3]{
	{\left({#1}_{#2}\right)}_{#2\in #3}
}
\newcommand{\fntuple}[3]{
	{\left({#1}_{#2}\right)}_{#2=1}^{#3}
}
\newcommand{\gengroup}[1]{
	\left< #1\right>
}
\newcommand{\stblzer}[2]{
	St_{#1}\lrprth{#2}
}
\newcommand{\cmmttr}[1]{
	\left[#1,#1\right]
}
\newcommand{\grpindx}[2]{
	\left[#1:#2\right]
}
\newcommand{\syl}[2]{
	Syl_{#1}\lrprth{#2}
}
\newcommand{\grtcd}[2]{
	mcd\lrprth{#1,#2}
}
\newcommand{\lsttcm}[2]{
	mcm\lrprth{#1,#2}
}
\newcommand{\amntpSyl}[2]{
	\mu_{#1}\lrprth{#2}
}
\newcommand{\gen}[1]{
	gen\lrprth{#1}
}
\newcommand{\ringcenter}[1]{
	C\lrprth{#1}
}
\newcommand{\zend}[2]{
	End_{\mathbb{Z}}^{#2}\lrprth{#1}
}
\newcommand{\genmod}[2]{
	\left< #1\right>_{#2}
}
\newcommand{\genlin}[1]{
	\mathscr{L}\lrprth{#1}
}
\newcommand{\opst}[1]{
	{#1}^{op}
}
\newcommand{\ringmod}[3]{
	\if#3l
	{}_{#1}#2
	\else
	\if#3r
	#2_{#1}
	\fi
	\fi
}
\newcommand{\ringbimod}[4]{
	\if#4l
	{}_{#1-#2}#3
	\else
	\if#4r
	#3_{#1-#2}
	\else 
	\ifstrequal{#4}{lr}{
		{}_{#1}#3_{#2}
	}
	\fi
	\fi
}
\newcommand{\ringmodhom}[3]{
	Hom_{#1}\lrprth{#2,#3}
}
\newcommand{\catarrow}[4]{
	\if #4e
			#1:#2\twoheadrightarrow #3
	\else \if #4m
			#1:#2\hookrightarrow #3
	\else 	#1:#2\to #3
	\fi
	\fi
}
\newcommand{\nattrans}[4]{
	Nat_{\lrsqp{#1,#2}}\lrprth{#3,#4}
}
\ExplSyntaxOn

\NewDocumentCommand{\functor}{O{}m}
{
	\group_begin:
	\keys_set:nn {nicolas/functor}{#2}
	\nicolas_functor:n {#1}
	\group_end:
}

\keys_define:nn {nicolas/functor}
{
	name     .tl_set:N = \l_nicolas_functor_name_tl,
	dom   .tl_set:N = \l_nicolas_functor_dom_tl,
	codom .tl_set:N = \l_nicolas_functor_codom_tl,
	arrow      .tl_set:N = \l_nicolas_functor_arrow_tl,
	source   .tl_set:N = \l_nicolas_functor_source_tl,
	target   .tl_set:N = \l_nicolas_functor_target_tl,
	Farrow      .tl_set:N = \l_nicolas_functor_Farrow_tl,
	Fsource   .tl_set:N = \l_nicolas_functor_Fsource_tl,
	Ftarget   .tl_set:N = \l_nicolas_functor_Ftarget_tl,	
	delimiter .tl_set:N= \_nicolas_functor_delimiter_tl,	
}

\dim_new:N \g_nicolas_functor_space_dim

\cs_new:Nn \nicolas_functor:n
{
	\begin{tikzcd}[ampersand~replacement=\&,#1]
		\dim_gset:Nn \g_nicolas_functor_space_dim {\pgfmatrixrowsep}		
		\l_nicolas_functor_dom_tl
		\arrow[r,"\l_nicolas_functor_name_tl"] \&
		\l_nicolas_functor_codom_tl
		\\[\dim_eval:n {1ex-\g_nicolas_functor_space_dim}]
		\l_nicolas_functor_source_tl
		\xrightarrow{\l_nicolas_functor_arrow_tl}
		\l_nicolas_functor_target_tl
		\arrow[r,mapsto] \&
		\l_nicolas_functor_Fsource_tl
		\xrightarrow{\l_nicolas_functor_Farrow_tl}
		\l_nicolas_functor_Ftarget_tl
		\_nicolas_functor_delimiter_tl
	\end{tikzcd}
}

\ExplSyntaxOff


\newcommand{\limseq}[3]{
	\if#3u
	\lim\limits_{#2\to\infty}#1	
		\else
			\if#3s
			#1\to
			\else
				\if#3w
				#1\rightharpoonup
				\else
					\if#3e
					#1\overset{*}{\to}
					\fi
				\fi
			\fi
	\fi
}

\newcommand{\norm}[1]{
	\crdnlty{\crdnlty{#1}}
}

\newcommand{\inter}[1]{
	int\lrprth{#1}
}
\newcommand{\cerrad}[1]{
	cl\lrprth{#1}
}

\newcommand{\restrict}[2]{
	\left.#1\right|_{#2}
}

\newcommand{\realprj}[1]{
	\mathbb{R}P^{#1}
}

\newcommand{\fungroup}[1]{
	\pi_{1}\lrprth{#1}	
}

\ExplSyntaxOn

\NewDocumentCommand{\shortseq}{O{}m}
{
	\group_begin:
	\keys_set:nn {nicolas/shortseq}{#2}
	\nicolas_shortseq:n {#1}
	\group_end:
}

\keys_define:nn {nicolas/shortseq}
{
	A     .tl_set:N = \l_nicolas_shortseq_A_tl,
	B   .tl_set:N = \l_nicolas_shortseq_B_tl,
	C .tl_set:N = \l_nicolas_shortseq_C_tl,
	f      .tl_set:N = \l_nicolas_shortseq_f_tl,
	g   .tl_set:N = \l_nicolas_shortseq_g_tl,	
	lcr   .tl_set:N = \l_nicolas_shortseq_lcr_tl,	
	
	A		.initial:n =A,
	B		.initial:n =B,
	C		.initial:n =C,
	f    .initial:n =,
	g   	.initial:n=,
	lcr   	.initial:n=lr,
	
}

\cs_new:Nn \nicolas_shortseq:n
{
	\begin{tikzcd}[ampersand~replacement=\&,#1]
		\IfSubStr{\l_nicolas_shortseq_lcr_tl}{l}{0 \arrow{r} \&}{}
		\l_nicolas_shortseq_A_tl
		\arrow{r}{\l_nicolas_shortseq_f_tl} \&
		\l_nicolas_shortseq_B_tl
		\arrow[r, "\l_nicolas_shortseq_g_tl"] \&
		\l_nicolas_shortseq_C_tl
		\IfSubStr{\l_nicolas_shortseq_lcr_tl}{r}{ \arrow{r} \& 0}{}
	\end{tikzcd}
}

\ExplSyntaxOff

\newcommand{\pushpull}{texto}
\newcommand{\testfull}{texto}
\newcommand{\testdiag}{texto}
\newcommand{\testcodiag}{texto}
\ExplSyntaxOn

\NewDocumentCommand{\commutativesquare}{O{}m}
{
	\group_begin:
	\keys_set:nn {nicolas/commutativesquare}{#2}
	\nicolas_commutativesquare:n {#1}
	\group_end:
}

\keys_define:nn {nicolas/commutativesquare}
{	
	A     .tl_set:N = \l_nicolas_commutativesquare_A_tl,
	B   .tl_set:N = \l_nicolas_commutativesquare_B_tl,
	C .tl_set:N = \l_nicolas_commutativesquare_C_tl,
	D .tl_set:N = \l_nicolas_commutativesquare_D_tl,
	P .tl_set:N = \l_nicolas_commutativesquare_P_tl,
	f      .tl_set:N = \l_nicolas_commutativesquare_f_tl,
	g   .tl_set:N = \l_nicolas_commutativesquare_g_tl,
	h   .tl_set:N = \l_nicolas_commutativesquare_h_tl,
	k .tl_set:N = \l_nicolas_commutativesquare_k_tl,
	l .tl_set:N = \l_nicolas_commutativesquare_l_tl,
	m .tl_set:N = \l_nicolas_commutativesquare_m_tl,
	n .tl_set:N = \l_nicolas_commutativesquare_n_tl,
	pp .tl_set:N = \l_nicolas_commutativesquare_pp_tl,
	up .tl_set:N = \l_nicolas_commutativesquare_up_tl,
	diag .tl_set:N = \l_nicolas_commutativesquare_diag_tl,	
	codiag .tl_set:N = \l_nicolas_commutativesquare_codiag_tl,		
	diaga .tl_set:N = \l_nicolas_commutativesquare_diaga_tl,	
	codiaga .tl_set:N = \l_nicolas_commutativesquare_codiaga_tl,		
	
	A		.initial:n =A,
	B		.initial:n =B,
	C		.initial:n =C,
	D		.initial:n =D,
	P		.initial:n =P,
	f    .initial:n =,
	g    .initial:n =,
	h    .initial:n =,
	k    .initial:n =,
	l    .initial:n =,
	m    .initial:n =,
	n	 .initial:n =,
	pp .initial:n=h,	
	up .initial:n=f,
	diag .initial:n=f,
	codiag .initial:n=f,
	diaga .initial:n=,
	codiaga .initial:n=,
}

\cs_new:Nn \nicolas_commutativesquare:n
{
	\renewcommand{\pushpull}{\l_nicolas_commutativesquare_pp_tl}
	\renewcommand{\testfull}{\l_nicolas_commutativesquare_up_tl}
	\renewcommand{\testdiag}{\l_nicolas_commutativesquare_diag_tl}
	\renewcommand{\testcodiag}{\l_nicolas_commutativesquare_codiag_tl}
	\begin{tikzcd}[ampersand~replacement=\&,#1]
		\if \pushpull h
			\if \testfull t
				\l_nicolas_commutativesquare_P_tl
				\arrow[bend~left]{drr}{\l_nicolas_commutativesquare_l_tl
				}
				\arrow[bend~right,swap]{ddr}{\l_nicolas_commutativesquare_m_tl}\arrow{dr}{\l_nicolas_commutativesquare_n_tl}\& \& \\
			\fi
			\if \testfull t
			\&
			\fi\l_nicolas_commutativesquare_A_tl
			\if \testdiag t
				\arrow{dr}[near~end]{\l_nicolas_commutativesquare_diaga_tl}
			\fi
			\arrow{r}{\l_nicolas_commutativesquare_f_tl} \arrow{d}[swap]{\l_nicolas_commutativesquare_g_tl} \&
			\l_nicolas_commutativesquare_B_tl
			 \arrow{d}{\l_nicolas_commutativesquare_h_tl}
			 \if \testcodiag t
			 \arrow{dl}[near~end,swap]{\l_nicolas_commutativesquare_codiaga_tl}
			 \fi
			 \\
			\if \testfull t
			\&
			\fi\l_nicolas_commutativesquare_C_tl
			\arrow{r}[swap]{\l_nicolas_commutativesquare_k_tl} \& \l_nicolas_commutativesquare_D_tl
		\else \if \pushpull l
			\if \testfull t
			\l_nicolas_commutativesquare_P_tl
			\& \& \\
			\& \arrow{ul}[swap]{\l_nicolas_commutativesquare_n_tl}
			\fi
			\l_nicolas_commutativesquare_A_tl
			  \&
			\l_nicolas_commutativesquare_B_tl
			\if \testfull t
			\arrow[bend~right,swap]{ull}{\l_nicolas_commutativesquare_l_tl
			}
			\fi
			 \arrow{l}[swap]{\l_nicolas_commutativesquare_f_tl}
			 \\
			\if \testfull t
			\& \arrow[bend~left]{uul}{\l_nicolas_commutativesquare_m_tl}
			\fi
			\l_nicolas_commutativesquare_C_tl\arrow{u}{\l_nicolas_commutativesquare_g_tl}
			\if \testcodiag t
			\arrow{ur}[near~start]{\l_nicolas_commutativesquare_codiaga_tl}
			\fi
			  \& \l_nicolas_commutativesquare_D_tl\arrow{l}{\l_nicolas_commutativesquare_k_tl}
			  \if \testdiag t
			  \arrow{ul}[near~start,swap,crossing~over]{\l_nicolas_commutativesquare_diaga_tl}
			  \fi
			\arrow{u}[swap]{\l_nicolas_commutativesquare_h_tl}
			\fi
		\fi
		
	\end{tikzcd}
}

\ExplSyntaxOff


\newcommand{\redhomlgy}[2]{
	\tilde{H}_{#1}\lrprth{#2}
}
\newcommand{\copyandpaste}{t}
\newcommand{\moncategory}[2]{Mon_{#1}\lrprth{-,#2}}
\newcommand{\epicategory}[2]{Mon_{#1}\lrprth{#2,-}}
\theoremstyle{definition}
\newtheorem{define}{Definición}
\newtheorem{lem}{Lema}
\newtheorem*{lemsn}{Lema}
\newtheorem{teor}{Teorema}
\newtheorem*{teosn}{Teorema}
\newtheorem{prop}{Proposición}
\newtheorem*{propsn}{Proposición}
\newtheorem{coro}{Corolario}
\newtheorem*{obs}{Observación}

\title{Ejercicios 43-53}
\author{Luis Gerardo Arruti Sebastian\\ Sergio Rosado Zúñiga}
\date{}
\begin{document}
	\maketitle
	\begin{enumerate}[label=\textbf{Ej \arabic*.}]
		\setcounter{enumi}{42}
		%43
		\renewcommand{\copyandpaste}{_{i=1}^{n}}
		\item Sea $\lrbrack{\catarrow{\pi_i}{A}{A_i}{}}\copyandpaste$ una familia de morfismos en una categoría semiaditiva $\mathscr{C}$. Las siguientes condiciones son equivalentes
		\begin{enumerate}[label=\textit{\alph*)}]
			\item $A$ y $\lrbrack{\catarrow{\pi_i}{A}{A_i}{}}\copyandpaste$ son un producto para $\lrbrack{A_i}\copyandpaste$ en $\mathscr{C}$;
			\item $\exists\ \lrbrack{\catarrow{\mu_i}{A_i}{A}{}}\copyandpaste$ en $\mathscr{C}$ tal que $\sum\limits_{i=1}^{n}\mu_i\pi_i=1_A$ y, $\forall\ i,j\in\lrsqp{1,n}$, $\pi_i\mu_j=\delta_{i,j}^A$.
		\end{enumerate}
		\begin{proof}
			Se tiene el siguiente resultado 
			\begin{propsn}[1.9.2]
				Sea $\lrbrack{\catarrow{\mu_i}{A_i}{A}{}}\copyandpaste$ una familia de morfismos en una categoría semiaditiva $\mathscr{C}$. Las siguientes condiciones son equivalentes
				\begin{enumerate}[label=\textit{\alph*)}]
					\item $A$ y $\lrbrack{\catarrow{\mu_i}{A_i}{A}{}}\copyandpaste$ son un coproducto para $\lrbrack{A_i}\copyandpaste$ en $\mathscr{C}$;
					\item $\exists\ \lrbrack{\catarrow{\pi_i}{A}{A_i}{}}\copyandpaste$ en $\mathscr{C}$ tal que $\sum\limits_{i=1}^{n}\mu_i\pi_i=1_A$ y, $\forall\ i,j\in\lrsqp{1,n}$, $\pi_i\mu_j=\delta_{i,j}^A$.
				\end{enumerate}
			\end{propsn}
		Así, considerando que 
		\begin{align*}
			\opst{\lrprth{\delta_{i,j}^A}}&=\left\{
			\begin{tabular}[c]{cc}
				$\opst{0}$,&$i\neq j$ \\ $\opst{1_{A_i}}$,& $i=j$
			\end{tabular}
			\right. \\
			&=\left\{
			\begin{tabular}[c]{cc}
				${0}$,&$i\neq j$ \\ ${1_{A_i}}$,& $i=j$
			\end{tabular}
			\right.\\
			&=\delta_{i,j}^A
		\end{align*} y pasando a la categoría opuesta, se tiene:
			\begin{propsn}[1.9.2\textsuperscript{op}]
				Sea $\lrbrack{\catarrow{\opst{\mu}_i}{A_i}{A}{}}\copyandpaste$ una familia de morfismos en una categoría semiaditiva $\opst{\mathscr{C}}$. Las siguientes condiciones son equivalentes
				\begin{enumerate}[label=\textit{\alph*)}]
					\item $A$ y $\lrbrack{\catarrow{\opst{\mu}_i}{A_i}{A}{}}\copyandpaste$ son un coproducto para $\lrbrack{A_i}\copyandpaste$ en $\opst{\mathscr{C}}$;
					\item $\exists\ \lrbrack{\catarrow{\opst{\pi}_i}{A}{A_i}{}}\copyandpaste$ en $\opst{\mathscr{C}}$ tal que $\sum\limits_{i=1}^{n}\opst{\mu}_i\opst{\pi}_i=1_A$ y, $\forall\ i,j\in\lrsqp{1,n}$, $\opst{\pi}_i\opst{\mu}_j=\delta_{i,j}^A$.
				\end{enumerate}
			\end{propsn}
		Lo cual, sabiendo que la noción dual de coproducto es producto, nos da el siguiente resultado dual 
		\begin{propsn}[1.9.2\textsuperscript{*}]
			Sea $\lrbrack{\catarrow{{\mu}_i}{A}{A_i}{}}\copyandpaste$ una familia de morfismos en una categoría semiaditiva $\opst{\mathscr{C}}$. Las siguientes condiciones son equivalentes
			\begin{enumerate}[label=\textit{\alph*)}]
				\item $A$ y $\lrbrack{\catarrow{{\mu}_i}{A}{A_i}{}}\copyandpaste$ son un producto para $\lrbrack{A_i}\copyandpaste$ en ${\mathscr{C}}$;
				\item $\exists\ \lrbrack{\catarrow{{\pi}_i}{A_i}{A}{}}\copyandpaste$ en ${\mathscr{C}}$ tal que $\sum\limits_{i=1}^{n}{\pi}_i\mu_i=1_A$ y, $\forall\ i,j\in\lrsqp{1,n}$, ${\mu}_j{\pi}_i={\delta}_{i,j}^A$.
			\end{enumerate}
		\end{propsn}
		Podemos reescribir la proposición anterior intercambiando $\mu$ por $\pi$  y viceversa, con lo cual por el principio de dualidad se tiene lo deseado.\\
		\end{proof}
		%44
		\item Sean $\mathscr{C}$ una categoría semiaditiva, $A=\coprod\limits\copyandpaste A_i$ y $B=\coprod\limits_{i=1}^{n} B_i$ en $\mathscr{C}$. Si la aplicación $+$ está dada por
		\begin{align*}
			\descapp{+}{Mat_{m\times n}\lrprth{A,B}\times Mat_{m\times n}\lrprth{A,B}}{Mat_{m\times n\lrprth{A,B}}}{\lrprth{\alpha,\beta}}{\gamma}{,}\\
			\lrsqp{\gamma}_{i,j}&:=\lrsqp{\alpha}_{i,j}+\lrsqp{\beta}_{i,j}, && \forall\ i,j\in\lrsqp{1,n}
		\end{align*}
		con $+$ al lado derecho de la igualdad anterior siendo la operación suma en $\ringmodhom{\mathscr{C}}{A_j}{B_i}$, entonces $\lrprth{Mat_{m\times n}\lrprth{A,B},+}$ es un monoide abeliano.
		\begin{proof}
			Sean $\alpha,\beta,\gamma\in\catmatrix{m}{n}{A}{B}$. Dado que $\mathscr{C}$ es semiaditiva se tiene que $\forall\ \lrprth{r,t}\in\lrsqp{1,m}\times\lrsqp{1,n}$ $\ringmodhom{\mathscr{C}}{A_t}{B_r}$ tiene estructura de monoide abeliano, en patícular su operación es asociativa. Así
			\begin{align*}
				\lrsqp{\lrprth{\alpha+\beta}+\gamma}_{r,t}&=\lrsqp{\alpha+\beta}_{r,t}+\lrsqp{\gamma}_{r,t}\\
				&=\lrprth{\lrsqp{\alpha}_{r,t}+\lrsqp{\beta}_{r,t}}+\lrsqp{\gamma}_{r,t} \\
				&=\lrsqp{\alpha}_{r,t}+\lrprth{\lrsqp{\beta}_{r,t}+\lrsqp{\gamma}_{r,t}}\\
				&=\lrsqp{\alpha+\lrprth{\beta+\gamma}}_{r,t}; &&\forall \lrprth{r,t}\in\lrsqp{1,m}\times\lrsqp{1,n}\\
				\implies +\text{ en }&\catmatrix{m}{n}{A}{B}\text{ es asociativa.}
			\end{align*}
		En forma análoga a lo anterior, empleando ahora que la operación en cada $\ringmodhom{\mathscr{C}}{A_t}{B_r}$ es conmutativa, se verifica que $+$ en $\catmatrix{m}{n}{A}{B}$ también lo es y que, si $\lrprth{r,t}\in\lrsqp{1,m}\times\lrsqp{1,n}$ $e_{A_t,B_r}$ es el neutro en $\ringmodhom{\mathscr{C}}{A_t}{B_r}$ y $E$ la matriz en $\catmatrix{m}{n}{A}{B}$ dada por $\lrsqp{E}_{r,t}=e_{A_t,B_r}$, entonces $E$ es el neutro de $+$ en $\catmatrix{m}{n}{A}{B}$.\\
		\end{proof}
		%45
		\item Sean $\{R_i\}_{i=1}^n$ una familia de anillos (asociativos con $1$). Considere el conjunto $R:=\displaystyle\bigtimes_{i=1}^nR_i$, con las
		operaciones suma y producto dadas  coordenada a coordenada, i.e., para $\displaystyle x=(x_i)^n_{i=1}$ y $\displaystyle y=(y_i)_{i=1}^n$ en $R$,
		definimos $\displaystyle x+y:=(x_i+y_i)_{i=1}^n$ y $\displaystyle xy=(x_iy_i)_{i=1}^n$.\\
		
		Pruebe que 
		\begin{enumerate}
			\item[a)] Con las operaciones anteriores $R$ es un anillo con $1_R=(1_{R_i})_{i=1}^n.$
			\item[b)] Para cada $j\in [1,n]$, la $j$-ésima pritección $proy_j:R\to R_j,$\\ $x=(x_i)_{i=1}^n\mapsto x_j$, es un morfismo en $Rings$ y 
			suryectivo en $Sets$.
			\item[c)] $R$ y $\{proy_j:R\to R_i\}_{i=1}^n$ son un producto en $Rings$ para $\{R_i\}_{i=1}^n$.
		\end{enumerate}
		\begin{proof}
			\boxed{a)} Sean $x,y\in R$, con $x=(x_i)_{i=1}^n,\,\,y=(y_i)_{i=1}^n,$\\ 
			$z=(z_i)_{i=1}^n\,\,$y \,\,$0_R=(0_i)_{i=1}^n$ donde $0_i$ es el neutro aditivo de $R_i$ para cada $i\in I=\{1,2,\ldots,n\}$.\\
			
			Grupo con respecto a la suma
			\begin{itemize}
				\item[i)] Como $x_i,y_i\in R_i\,\,\,\forall i\in I$ entonces $x_i+y_i\in R_i\,\,\forall i\in I$, así \\$x+y:=(x_i+y_i)_{i=1}^n\in R$. Mas aún, 
				$(x_i+y_i)_{i=1}^n=(y_i+x_i)_{i=1}^n$, por lo que $x+y=y+x$.
				\item[ii)] Como $R$ es anillo para toda $i\in I$, entonces 
				\[(x+y)+z=[(x_i+y_i)+z_i]_{i=1}^n=[x_i+(y_i+z_i)]_{i=1}^n=x+(y+z).\]
				\item[iii)] $0_R+x=(0_i+x_i)_{i=1}^n=(x_i)_{i=1}^n=x$.
				\item[iv)] Definimos para cada $x\in R$\quad $-x:=(-x_i)_{i=1}^n$, entonces \\$x+(-x)=(x_i+(-x_i))_{i=1}^n=(0_i)_{i=1}^n=0_R.$
			\end{itemize}
			Por lo tanto $R$ es un grupo abeliano con la suma.\\
			
			Monoide con respecto a la multiplicación:\\
			
			Como $R_i$ es un anillo para cada $i\in I$ se tiene que
			\begin{itemize}
				\item[i)] $xy=(x_iy_i)_{i=1}^n\in R$ pues $x_iy_i\in R_i\quad \forall i\in I$.
				\item[ii)] $(xy)z=[(x_iy_i)z_i]_{i=1}^n=[x_i(y_iz_i)]_{i=1}^n=x(yz)$.
				\item[iii)] $x1_R=(x_i1_{R_i})_{i=1}^n=(x_i)_{i=1}^n=x=(x_i)_{i=1}^n=(1_{R_i}x_i)_{i=1}^n=1_Rx$.
				\item[iv)] $(x+y)z=[(x_i+y_i)z_i]^n_{i=1}=[x_iz_i+y_iz_i]^n_{i=1}=xz+yz.$
			\end{itemize}
			Por lo tanto $R$ es un anillo con $1_R=(1_{R_i})_{i=1}^n$.\\
			
			\boxed{b)} Sea $j\in \{1,2\ldots,n\}$, tomamos $proy_j:R\to R_j$ tal que \\
			$z=(z_i)_{i=1}^n\mapsto z_j$\quad y sean $x,y$ descritos como en el inciso a).\\
			
			Entonces $proy_j(x+y)=proy_j[(x_i+y_i)_{i=1}^n] =x_j+y_j=proy_j(x)+proy_j(y)$ y 
			$proy_j(xy)=proy_j[(x_iy_i)_{i=1}^n] =x_jy_j=proy_j(x)proy_j(y)$.\\
			
			Además, si $a\in R_j$ para alguna $j\in \{1,2,\ldots,n\}$, 
			se tiene el elemento $\hat{a}\in R$ tal que $\hat{a}=(a_i)_{i=1}^n$ donde $a_i=0\quad \forall i\neq j$ \, y \, $a_j=a$.
			Así $proy_j(\hat{a})=a$ y en consecuencia $proy_j$ es un morfismo de anillos suprayectivo.\\
			
			\boxed{c)} Sea $P$ un anillo y $\{p_i:P\to R_i\}_{i=1}^n$ una familia de morfismos de anillos. Sea $\varphi:P\to R$ tal que $\varphi(x)=x_p$ donde 
			$x$ es el elemento de $R$ tal que $x_p=(x_i)^n_{i=1}$ con $x_i=p_i(x)$ para cada $i\in \{1,2,\ldots,n\}$. Veamos que es morfismo de anillos.\\
			
			Como $p_i$ es morfismo de anillos y $p_i(x)\in R_i\quad \forall i\in \{1,2,\ldots,n\}$ y para cada $x\in P$, 
			entonces si $(a_i)_{i=1}^n=\varphi(x)$ se tiene que 
			$a_i=p_i(x)$ para cada $i\in \{1,2,\ldots,n\}$ por lo tanto $\varphi$ está bien definida.\\
			
			Sean $x,y\in P$, entonces 
			\begin{gather*}
				\varphi(x+y)=(x+y)_p=(p_i(x+y))_{i=1}^n=(p_i(x)+p_i(y))_{i=1}^n\\
				=(p_i(x))_{i=1}^n+(p_i(y))_{i=1}^n=x_p+y_p=\varphi(x)+\varphi(y)
			\end{gather*}
			y
			\begin{gather*}
				\varphi(xy)=(xy)_p=(p_i(xy))_{i=1}^n=(p_i(x)p_i(y))_{i=1}^n\\
				=(p_i(x))_{i=1}^n(p_i(y))_{i=1}^n=x_py_p=\varphi(x)\varphi(y).
			\end{gather*}
			
			Por lo tanto $\varphi$ es un morfismo de anillos.
			Notemos que, para toda $x\in P$, $proy_j\circ \varphi(x)=proy_j(x_p)=p_j(x)$ para cada $j\in \{1,2,\ldots,n\}$, por lo tanto $p_j=proy_j\circ \varphi$.\\
			
			Por último si existiera $\eta:P\to R$ tal que $p_j=proy_j\eta$ para cada \\$j\in \{1,2,\ldots,n\}$, entonces para cada $x\in P$ se tiene que 
			$\eta(x)\in R$, es decir, $\eta(x)=(x_i)_{i=1}^n$ con $x_i\in R_i\quad \forall i\in \{1,2,\ldots,n\}$.
			Ahora, como\\ $proy_j\eta(x)=p_j(x)$, entonces $x_j=p_j(x)\quad \forall j\in \{1,2,\ldots,n\}$, es decir, $\eta(x)=(p_i(x))_{i=1}^n=x_p=\varphi(x).$\\
			
			Por lo tanto $\varphi$ es único y así $R$ y $\{proy_j:R\to R_i\}_{i=1}^n$ son un producto en $Rings$ para $\{R_i\}_{i=1}^n$.
			
		\end{proof}
		%46
		\item Para una categoría $\mathscr{C}$, pruebe que las siguientes condiciones son equivalentes
		
		\begin{itemize}
			\item[a)] $\mathscr{C}$ tiene objeto cero y biproductos $A\coprod A$ en  $\mathscr{C}$, $\forall A\in \mathscr{C}$.
			\item[b)] $\mathscr{C}^{op}$ tiene objeto cero y biproductos $A\coprod A$ en  $\mathscr{C}^{op}$, $\forall A\in \mathscr{C}$.
		\end{itemize}
		
		\begin{proof}
			Notemos que $\mathscr{C}$ tiene objeto cero si y sólo si $\mathscr{C}^{op}$ tiene objeto cero, pues si $\mathscr{C}$ tiene objeto cero $0$, entonces\\
			$|\operatorname{Hom}_{\mathscr{C}}(X,0)|=1=|\operatorname{Hom}_{\mathscr{C}}(0,X)|,\quad \forall X\in \mathscr{C}$, pero esto pasa si \\
			y sólo si $|\operatorname{Hom}_{\mathscr{C}}(0,X)|=1=|\operatorname{Hom}_{\mathscr{C}}(X,0)|,\quad \forall X\in \mathscr{C}^{op}$.\\
			
			\boxed{a)\Rightarrow b)} Como $\mathscr{C}$ y $\mathscr{C}^{op}$ tienen objeto cero, si $A\coprod A$ es biproducto en $\mathscr{C}$, entonces 
			existe $A\prod A$ en $\mathscr{C}$ y $\delta:A\coprod A\longrightarrow A\prod A$ un isomorfismo. Sean $\{\mu_1,\mu,2:A\longrightarrow A\coprod A\}$
			\quad y \quad $\{\pi_1,\pi_2:A\prod A \longrightarrow A\}$ los morfismos canonicos del coproducto y producto respectivamente, entonces por el 
			ejercicio 40 $A\coprod A$ con $\{\mu_1^{op},\mu_2^{op}:A\coprod A\longrightarrow A\}$ y $A\prod A$ con 
			$\{\pi_1^{op},\pi_2^{op}:A\longrightarrow A\prod A \}$ son un producto y un coproducto en $\mathscr{C}^{op}$ respectivamente tales que 
			$\delta^{op}:A\prod A\longrightarrow A\coprod A$ es un isomorfismo al ser $\delta$ iso. Por lo tanto $A\coprod A$ es un biproducto en 
			$\mathscr{C}^{op}$.\\
			
			\boxed{b)\Rightarrow a)} Es análogo a lo anterior pues $(\mathscr{C}^{op})^{op}=\mathscr{C}$.
			
		\end{proof}
		%47
		\item Sean $\mathscr{C}$ una categoría preaditiva, $A\in\mathscr{C}$ y $\theta\in\cend{\mathscr{C}}{A}$. Si $\theta$ es idempotente, entonces $1_A-\theta$ también lo es.
		\begin{proof}
			Se tiene que $\theta^2=\theta$ y, como $\mathscr{C}$ es preaditiva, la composición de morfismos  en $\mathscr{C}$ es bilineal con respecto a $+$ en $\cend{\mathscr{C}}{A}$. Así
			\begin{align*}
				\lrprth{1_A-\theta}^2&=\lrprth{1_A-\theta}\lrprth{1_A-\theta}=1_A^2-1_A\theta-\theta 1_A+\theta^2\\
				&=1_A-\theta-\theta+\theta\\
				&=1_A-\theta.
			\end{align*}
		\end{proof}
		%48
		\item Sea $\mathscr{C}$ una categoría, entonces:
		\begin{enumerate}[label=\textit{\alph*)}]
			\item $\mathscr{C}$ es abeliana si y sólo si $\opst{\mathscr{C}}$ es abeliana.
			\item Supongamos que $\mathscr{C}$ es abeliana y sean $\catarrow{\alpha}{A}{B}{}, \catarrow{\beta}{C}{B}{}$ en $\mathscr{C}$. Si $\alpha$, o $\beta$, es epi, entonces $\lambda$ el morfismo asociado a la matriz $\lrprth{\alpha\ \beta}$ es epi.
		\end{enumerate}
		\begin{proof}
			\boxed{a)} Se tiene del Teorema 1.10.1 d) que  una categoría es abeliana si y sólo si  satisface las siguientes dos condiciones
			\begin{enumerate}[label=\textit{C\arabic*)}]
				\item $\mathscr{C}$ es normal y conormal,
				\item $\mathscr{C}$ tiene pull-backs y push-outs.
			\end{enumerate}
			Dado que $C1)$ y $C2)$ son condiciones autoduales, pues (normal)\textsuperscript{*}=conormal y (pull-back)\textsuperscript{*}=push-out, entonces una categoría $\mathscr{C}$ las satisface si y sólo si $\opst{\mathscr{C}}$ las satisface.\\
			
			\boxed{b)} Sean $\lrbrack{\pi_1,\pi_2}$ las proyecciones naturales y $\lrbrack{\mu_1,\mu_2}$ las inclusiones naturales del biproducto $A\coprod C$. Entonces $\lambda=\alpha\pi_1+\beta\pi_2\in\ringmodhom{\mathscr{C}}{A\coprod C}{B}$. Supongamos que $f,g\in\ringmodhom{\mathscr{C}}{B}{D}$ son tales que $f\lambda=g\lambda$. Así
			\begin{align*}
				f\lrprth{\alpha\pi_1+\beta\pi_2}&=g\lrprth{\alpha\pi_1+\beta\pi_2},\\
				\implies f\lrprth{\alpha\pi_1+\beta\pi_2}\mu_1&=f\lrprth{\alpha\pi_1+\beta\pi_2}\mu_1\\
				\implies f\lrprth{\alpha\lrprth{\pi_1\mu_1}+\beta\lrprth{\pi_2\mu_1}}&=g\lrprth{\alpha\lrprth{\pi_1\mu_1}+\beta\lrprth{\pi_2\mu_1}}\\
				\implies f\lrprth{\alpha\lrprth{1_A}+\beta 0}&=g\lrprth{\alpha 1_A+\beta 0}\\
				\implies f\alpha=g\alpha.
			\end{align*}
			De lo anterior se sigue que $f=g$ si $\alpha$ es epi. La prueba es análoga, componiendo por $\mu_2$ a derecha, si suponemos que $\beta$ es epi.\\
		\end{proof}
		%49
		\item Pruebe que, para un anillo $R$
		\begin{itemize}
			\item[a)] $Mod(R)$ es abeliana.
			\item[b)] $mod(R)$ es abeliana si $R$ es un anillo artiniano izquierdo, donde $mod(R)$ es la subcategoría de $Mod(R)$, cuyos objetos son los $R$-módulos
			finitamente generados.
		\end{itemize}
		\begin{proof}
			\boxed{a)} Por los ejercicios 29 y 30, $Mod(R)$ tiene kerneles y cokerneles, por el ejercicio 32 es normal y conormal y por el ejercicio 41 
			tiene productos y coproductos (en particular tiene productos y coproductos finitos) entonces por el teorema 1.10.1 c) se tiene que $Mod(R)$ es abeliana.	
		\end{proof}		
		%50
		\item Sean $\mathscr{A}$ y $\mathscr{B}$ categorías aditivas y $F:\mathscr{A}\to \mathscr{B}$ un funtor de cualquier varianza. Pruebe que los 
		siguientes son equivalentes:
		\begin{itemize}
			\item[a)] $F$ es aditivo.
			\item[b)] $F_{op}:=F\circ D_{\mathscr{A}^{op}}: \mathscr{A}^{op}\longrightarrow \mathscr{B}$ es aditivo.
			\item[c)] $F^{op}:=D_{\mathscr{B}}\circ F: \mathscr{A}\longrightarrow \mathscr{B}^{op}$ es aditivo.
			\item[d)] $F_{op}^{op}:=D_{\mathscr{B}}\circ F \circ D_{\mathscr{A}^{op}}: \mathscr{A}^{op}\longrightarrow  \mathscr{B}^{op}$ es aditivo.
		\end{itemize}
		(Se cambió ligeramente el enunciado para fines practicos de la demostración.)
		
		\begin{proof} Recordemos que $\mathscr{A}^{op}$ y $\mathscr{B}^{op}$ son categorías abelianas por 1.9.15, también que
			$D_{\mathscr{A}^{op}}:\mathscr{A}^{op}\longrightarrow \mathscr{A}$ es un funtor contravariante tal que 
			$(\begin{tikzcd} A\arrow{r}{f}&B\end{tikzcd})\mapsto(\begin{tikzcd} B\arrow{r}{f^{op}}&A\end{tikzcd})$ y, como $\operatorname{Hom}_\mathscr{A}
			(X,Y)$ \\es un grupo abeliano, $\displaystyle\operatorname{Hom}_{\mathscr{A}^{op}}(Y,X)$ es también un grupo abeliano.\\
			
			Se probará el caso en que $F$ es covariante
			
			\boxed{a)\Rightarrow b)} Supongamos $F$ es aditivo. Entonces \\$F:\operatorname{Hom}_\mathscr{A}(X,Y)\longrightarrow
			\operatorname{Hom}_\mathscr{B}(F(X),F(Y))$ es un morfismo en $Ab$ para toda $X,Y\in \mathscr{A}$. Así para cualesquiera
			$f^{op}:\operatorname{Hom}_{\mathscr{A}^{op}}(B,A)$\quad y\\
			$g^{op}:\operatorname{Hom}_{\mathscr{A}^{op}}(C,B)$, entonces
			\begin{gather*}
				F_{op}(f^{op}\circ g^{op})=F\circ D_{\mathscr{A}^{op}}(f^{op}\circ g^{op})=F\circ D_{\mathscr{A}^{op}}((g\circ f)^{op})=F(g\circ f)\\
				=F(g)\circ F(f)=F\circ D_{\mathscr{A}^{op}}(g^{op})\circ F\circ D_{\mathscr{A}^{op}}(f^{op})=F_{op}(g^{op})\circ F_{op}(f^{op}).
			\end{gather*}
			Es decir,  $F_{op}$ es un funtor aditivo (pues es contravariante).
			
			\boxed{b)\Rightarrow c)} Supongamos $F_{op}$ es aditivo.Entonces \\$F_{op}:\operatorname{Hom}_{\mathscr{A}^{op}}(Y,X)\longrightarrow
			\operatorname{Hom}_\mathscr{B}(F(X),F(Y))$ es un morfismo en $Ab$ para toda $X,Y\in \mathscr{A^{op}}$. Así para cualesquiera
			$f:\operatorname{Hom}_{\mathscr{A}}(B,C)$\quad y\\
			$g:\operatorname{Hom}_{\mathscr{A}}(A,B)$, entonces
			\begin{gather*}
				F^{op}(f\circ g)=D_{\mathscr{B}}\circ F(f\circ g)=D_{\mathscr{B}}(F(f\circ g))=(F(f\circ g))^{op}=[F((g^{op}\circ f^{op})^{op})]^{op}\\
				=[F\circ D_{\mathscr{A}^{op}}(g^{op}\circ f^{op})]^{op}=[F\circ D_{\mathscr{A}^{op}}(f^{op})\circ F\circ D_{\mathscr{A}^{op}}(g^{op})]^{op}\\
				=[F(f)\circ F(g)]^{op}=(F(g))^{op}\circ (F(f))^{op}=D_{\mathscr{B}}\circ F(g)\circ D_{\mathscr{B}}\circ F(f)=F^{op}(g)\circ F^{op}(f).
			\end{gather*}
			Es decir,  $F^{op}$ es un funtor aditivo (pues es contravariante).
			
			\boxed{c)\Rightarrow d)} Supongamos $F^{op}$ es aditivo.Entonces \\$F^{op}:\operatorname{Hom}_{\mathscr{A}}(X,Y)\longrightarrow
			\operatorname{Hom}_{\mathscr{B}^{op}}(F(Y),F(X))$ es un morfismo en $Ab$ para toda $X,Y\in \mathscr{A}$. Así para cualesquiera
			$g:\operatorname{Hom}_{\mathscr{A}}(A,B)$\quad y\\
			$f:\operatorname{Hom}_{\mathscr{A}}(B,C)$, entonces
			
			\begin{gather*}
				F_{op}^{op}(f^{op}\circ g^{op})=D_{\mathscr{B}}\circ F \circ D_{\mathscr{A}^{op}}(f^{op}\circ g^{op})
				=D_{\mathscr{B}}\circ F \circ D_{\mathscr{A}^{op}}((g\circ f)^{op})\\
				=D_{\mathscr{B}}\circ F (g\circ f)=D_{\mathscr{B}}\circ F(f)\circ D_{\mathscr{B}}\circ F(g)\\
				=D_{\mathscr{B}}\circ F \circ D_{\mathscr{A}^{op}}(f^{op})\circ D_{\mathscr{B}}\circ F \circ D_{\mathscr{A}^{op}}(g^{op})\\
				=F_{op}^{op}(f^{op})\circ F_{op}^{op}(g^{op}).
			\end{gather*}
			Es decir, $F_{op}^{op}$ es un funtor aditivo covariante.
			
			\boxed{d)\Rightarrow a)} Supongamos $F_{op}^{op}$ es aditivo.Entonces \\$F_{op}^{op}:\operatorname{Hom}_{\mathscr{A}^{op}}(X,Y)\longrightarrow
			\operatorname{Hom}_{\mathscr{B}^{op}}(F(X),F(Y))$ es un morfismo en $Ab$ para toda $X,Y\in \mathscr{A}$. Así para cualesquiera
			$g:\operatorname{Hom}_{\mathscr{A}}(A,B)$\quad y\\
			$f:\operatorname{Hom}_{\mathscr{A}}(B,C)$, entonces
			
			\begin{gather*}
				F(f\circ g)=F\circ D_{\mathscr{A}^{op}}(g^{op}\circ f^{op})=[D_{\mathscr{B}}\circ F \circ D_{\mathscr{A}^{op}}(g^{op}\circ f^{op})]^{op}\\
				=[D_{\mathscr{B}}\circ F \circ D_{\mathscr{A}^{op}}(g^{op})\circ D_{\mathscr{B}}\circ F \circ D_{\mathscr{A}^{op}}(f^{op})]^{op}
				=[D_{\mathscr{B}}\circ F(g)\circ D_{\mathscr{B}}\circ F(f)]^{op}\\
				=(D_{\mathscr{B}}\circ F(f))^{op}\circ (D_{\mathscr{B}}\circ F(g))^{op}= F(f)\circ F(g).
			\end{gather*}
			Es decir, $F_{op}^{op}$ es un funtor aditivo covariante.\\
			
			El caso en que $F$ es contravariante es análogo a esta demostración.\\
		\end{proof}
		%51
		\item Sean $\mathscr{A}$ una categoría aditiva y $A\in\mathscr{A}$ tal que $1_A=0_{A,A}$. Entonces $A$ es un objeto cero en $\mathscr{A}$.
		\begin{proof}
			Como $\mathscr{A}$ es aditiva, en partícular es una $\mathbb{Z}$-categoría, con lo cual por la Observación $1.9.1(2)$ todo objeto inicial en $\mathscr{A}$ es un objeto cero en $\mathscr{A}$. Así pues basta con verificar que bajo estas condiciones $A$ es un objeto inicial en $\mathscr{A}$. \\
			Sea $X\in\mathscr{A}$ y $f\in\ringmodhom{\mathscr{A}}{A}{X}$. Como $\mathscr{A}$ tiene objeto cero, por ser aditiva, entonces existe un (único) morfismo cero $0_{A,X}\in\ringmodhom{\mathscr{A}}{A}{X}$. Además
			\begin{align*}
				f&=f 1_A=f 0_{A,A}= 0_{A,X},\\
				&\implies \ringmodhom{\mathscr{A}}{A}{X}=\lrbrack{0_{A,X}}.
			\end{align*}
		\end{proof}
		%52
		\item Sean $\mathscr{A}, \mathscr{B}$ categorías aditivas, $X=\coprod\limits\copyandpaste X_i$, $Y=\coprod\limits_{j=1}^{m}Y_i$ en $\mathscr{A}$, $F$ un funtor que preserva coproductos finitos, $\alpha\in\catmatrix{m}{n}{X}{Y}$ y $\overline{\alpha}$ el morfismo en $\ringmodhom{\mathscr{A}}{X}{Y}$ asociado a $\alpha$. Entonces la matriz asociada al morfismo $F\lrprth{\overline{\alpha}}$, $\varphi_{FY,FX}\lrprth{F\lrprth{\overline{\alpha}}}\in\catmatrix{m}{n}{FX}{FY}$, está dada por
		\begin{align*}
			\lrsqp{\varphi_{FY,FX}\lrprth{F\lrprth{\overline{\alpha}}}}_{i,j}&=F\lrprth{\lrsqp{\alpha}_{i,j}}, && \forall\ i,j
		\end{align*}
		\begin{proof}
			Dado que $\mathscr{A}$ es abeliana, $X$ y $Y$ son biproductos en $\mathscr{A}$, al igual que $FX$ y $FY$ lo son en $\mathscr{B}$ por ser esta última abeliana y ser $F$ un funtor que preserva coproductos finitos. Más aún, si $\fntfam{\mu_i^X}{i}{n}$, $\fntfam{\mu_i^Y}{i}{m}$, $\fntfam{\pi_i^X}{i}{n}$ y $\fntfam{\pi_i^Y}{i}{m}$ son respectivamente las inclusiones y proyecciones naturales de $X$ y $Y$, entonces $\fntfam{F\lrprth{\mu_i^X}}{i}{n}$, $\fntfam{F\lrprth{\mu_i^Y}}{i}{m}$, $\fntfam{F\lrprth{\pi_i^X}}{i}{n}$ y $\fntfam{F\lrprth{\pi_i^Y}}{i}{m}$ son respectivamente las inclusiones y proyecciones naturales de $FX$ y $FY$. Así
			\begin{align*}
				\lrsqp{\varphi_{FY,FX}\lrprth{F\lrprth{\overline{\alpha}}}}_{i,j}&=F\lrprth{\pi_i^Y}F\lrprth{\overline{\alpha}}F\lrprth{\mu_j^X}\\
				&=F\lrprth{\pi_i^Y\overline{\alpha}\mu_j^X}\\
				&=F\lrprth{\pi_i^Y\lrprth{\sum_{r,t}\mu_t^Y\lrsqp{\alpha}_{i,j}\pi_r^X}\mu_j^X}\\
				&=F\lrprth{\sum_{r,t}\lrprth{\lrprth{\pi_i^Y\mu_t^Y}\lrsqp{\alpha}_{i,j}\lrprth{\pi_r^X\mu_j^X}}}\\
				&=F\lrprth{\sum_{r,t}\delta_{i,t}^Y\lrsqp{\alpha}_{i,j}\delta_{r,j}^X}\\
				&=F\lrprth{1_{Y_i}\lrsqp{\alpha}_{i,j} 1_{X_j}}\\
				&=F\lrprth{\lrsqp{\alpha}_{i,j}}.
			\end{align*}
		\end{proof}
		%53
		\item Sea $G:\mathscr{A}\longrightarrow \mathscr{B}$ un funtor contravariante entre categorías aditivas. Pruebe que $G$ s aditivo si y sólo si
		manda productos finitos en $\mathscr{A}$ a coproductos finitos en $\mathscr{B}$.
		
		\begin{proof}
			Decimos que un funtor contravariante $G:\mathscr{A}\longrightarrow \mathscr{B}$ entre categorías aditivas manda productos finitos en $\mathscr{A}$ 
			a coproductos finitos en $\mathscr{B}$ si el funtor $G_{op}:=G\circ D_{\mathscr{A}^{op}}$ preserva coproductos finitos en $\mathscr{A}^{op}$.\\
			
			Supongamos $G$ es aditivo, entonces $G_{op}:=G\circ D_{\mathscr{A}^{op}}$ es aditivo por el ejercicio 50. Así, por 1.10.2 $G_{op}$ preserva
			coproductos finitos en $\mathscr{B}$.\\
			
			Ahora, si suponemos que $G$ manda productos finitos de $\mathscr{A}$ en coproductos finitos de $\mathscr{B}$ se tiene por definición que
			$G\circ D_{\mathscr{A}^{op}}$ preserva coproductos finitos en ${\mathscr{A}^{op}}$, entonces por 1.10.2 $G_{op}$ es aditivo, así
			$G$ es aditivo.\\
		\end{proof}
	\end{enumerate}		
\end{document}