\documentclass{article}
\usepackage[utf8]{inputenc}
\usepackage{mathrsfs}
\usepackage[spanish,es-lcroman]{babel}
\usepackage{amsthm}
\usepackage{amssymb}
\usepackage{enumitem}
\usepackage{graphicx}
\usepackage{caption}
\usepackage{float}
\usepackage{amsmath,stackengine,scalerel,mathtools}
\usepackage{xparse, tikz-cd, pgfplots}
\usepackage{xstring}
\usepackage{mathrsfs}
\usepackage{comment}
\usepackage{faktor}
\usepackage[all]{xy}
\input{C:/Users/Darkandlive/Desktop/Latex/Álgebra/Tercer semestre/Comandos Matemagicos}
\title{Ejercicios 32-42}
\author{Luis Gerardo Arruti Sebastian\\ Sergio Rosado Zúñiga}
\date{}
\begin{document}
	\maketitle
	\begin{enumerate}[label=\textbf{Ej \arabic*.}]
		\setcounter{enumi}{31}
		%32
		\item $Mod\lrprth{R}$ es normal y conormal.
		\begin{proof}
			Se tiene que, por el Ej. 28, $Mod\lrprth{R}$ tiene objeto cero y más aún, que $\forall\ M,N\in Mod\lrprth{R}$ el morfismo $0$ de $M$ en $N$ está dado por
			\begin{align*}
				\descapp{0_{M,N}}{M}{N}{m}{0_N}{,}
			\end{align*} con $0_N$ el neutro aditivo de $N$. En vista de lo anterior, en lo sucesivo prescindiremos de los subíndices en la notación de los morfismos cero.
			
			\boxed{\text{Normal}} Sean $\catarrow{\alpha}{M}{N}{}$ en $Mod\lrprth{R}$, $P:=\faktor{N}{Im\lrprth{\alpha}}$ y $\beta$ el epi canónico dado por
			\begin{align*}
				\descapp{\beta}{N}{P}{n}{n+In\lrprth{\alpha}}{.}
			\end{align*}
			Afirmamos que $\alpha$ es un kernel para $\beta$. En efecto:\\
			Dado que $Im\lrprth{\alpha}$ es el neutro aditivo de $P$ y $Im\lrprth{\alpha}=\lrbrack{\alpha\lrprth{m}\ \vline\ m\in M}$, se tiene que $\beta\alpha=0$.\\
			Supongamos ahora que $\catarrow{\alpha'}{M'}{N}{}$ en $Mod\lrprth{R}$ es tal que $\beta\alpha'=0$, así
			\begin{align*}
				\beta\lrprth{\alpha'\lrprth{a}}&=Im\lrprth{\alpha} && \forall\ a\in M'\\
				\implies Im\lrprth{\alpha'}&\subseteq Im\lrprth{\alpha}
			\end{align*}
			De lo cual se sigue que	$\forall\ a\in M' \exists\ b_a\in M$ tal que $\alpha\lrprth{b_a}=\alpha'\lrprth{a}$. Más aún, como $\alpha$ es un monomorfismo se tiene que tal $b_a$ es único, y por lo tanto la siguiente aplicación es una función bien definida
			\begin{align*}
				\descapp{\gamma}{M'}{M}{a}{b_a}{.}
			\end{align*}
			Sean $r\in R$, $a_1,a_2\in M'$. Así
			\begin{align*}
				\alpha\lrprth{b_{ra_1-a_2}}&=\alpha'\lrprth{ra_1-a_2}=r\alpha'\lrprth{a_1}-\alpha\lrprth{a_2}\\
				&=r\alpha\lrprth{b_{a_1}}-\alpha\lrprth{b_{a_2}}=\alpha\lrprth{rb_{a_1}-b_{a_2}},\\
				\implies b_{ra_1-a_2}&=rb_{a_1}-b_{a_2}, && \alpha\text{ es mono}\\
				\implies \gamma\lrprth{ra_1-a_2}&=r\gamma\lrprth{a_1}-\gamma\lrprth{a_2}.
			\end{align*}
			Con lo cual $\gamma$ es un morfismo de $R$-módulos que satisface que, si $a\in M'$, entonces
			\begin{align*}
				\alpha\gamma\lrprth{a}&=\alpha\lrprth{b_a}=\alpha'\lrprth{a},\\
				\therefore\ \alpha\gamma=\alpha'.
			\end{align*}
			Más aún, puesto que $\alpha$ es mono, $\gamma$ es el único morfismo de $R$-módulos de $M'$ en $M$ que satisface lo anterior, con lo cual se ha verificado la afirmación.\\
			
			\boxed{\text{Conormal}} Ahora supongamos que $\catarrow{\alpha}{M}{N}{}$ es epi en $Mod\lrprth{R}$ y denotemos por $\beta$ al morfismo inclusión de $Ker\lrprth{\alpha}$ en $M$. Afirmamos que $\alpha$ es un cokernel para $\beta$, en efecto:\\
			Como $Ker\lrprth{\alpha}=\lrbrack{m\in M\ \vline\ \alpha\lrprth{m}=0_N}$, entonces $\alpha\beta=0$. Sea $\catarrow{\alpha'}{M}{N'}{}$ en $Mod\lrprth{R}$ tal que $\alpha'\beta=0$, así 
			\begin{equation*}
				Ker\lrprth{\alpha'}\supseteq Im\lrprth{\beta}=Ker\lrprth{\alpha}.
			\end{equation*}
			Como $\alpha$ es epi se tiene que $N=Im\lrprth{\alpha}$. Así, consideremos la aplicación
			\begin{align*}
				\descapp{\gamma}{N}{N'}{\alpha\lrprth{m}}{\alpha'\lrprth{m}}{,}
			\end{align*}
			la cual es una función bien definida, puesto que si $m,o\in M$ son tales que $\alpha\lrprth{m}=\alpha\lrprth{o}$, entonces
			\begin{align*}
				m-o&\in Ker\lrprth{\alpha}\subseteq Ker\lrprth{\alpha'}\\
				\implies \alpha'\lrprth{m}&=\alpha'\lrprth{o}.
			\end{align*}
			Más aún, es un morfismo de $R$-módulos, pues $\alpha$ y $\alpha'$ lo son, que satisface que $\gamma\alpha=\alpha'$. Finalmente $\gamma$ es el único morfismo de $R$-módulos que satisface la igualdad anterior dado que $\alpha$ es epi.\\
		\end{proof}
		%33
\item Pruebe que $Mod(R)$ es colocalmente pequeña.
\begin{proof}
Este ejercicio es consecuencia de varios resultados pasados:\\
Dado un anillo $R$ se cumple que:
\begin{itemize}
\item $Mod(R)$ tiene kerneles y Cokerneles (ejercicios 29 y 30 respectivamente).
\item $Mod(R)$ es localmente pequeña (ejercicio 31).
\item $Mod(R)$ es normal y conormal (ejercicio 32).
\end{itemize}
Entonces por el teorema 1.6.3 $Mod(R)$ es colocalmente pequeña.
\end{proof}
             %34
\item Sean $\mathscr{C}$ una categoría exacta y \\

\centerline{
\xymatrix{0\ar[r] & A\ar[r]^\alpha & B\ar[r]^{\beta}\ar[d]^\gamma & C\ar[r] & 0\\
0\ar[r] & A'\ar[r]^{\alpha'} & B'\ar[r]^{\beta'} & C'\ar[r] & 0
}}

un diagrama en $\mathscr{C}$ con filas exactas. Pruebe que $\exists f:A\to A'$ tal que $\gamma\alpha=\alpha' f\quad \iff\quad \exists g:C\to C'$ tal que
$\beta'\gamma=g \beta$. Mas aún, dado uno de ellos ("$f$"\,o "$g$") el otro queda deteminado univocamente.
\begin{proof}
Supongamos tenemos el diagrama de las hipotesis sobre una categoría exacta $\mathscr{C}$. Como $\alpha'$ es mono y la sucesión es exacta, se tiene
que $\alpha'\simeq Im(\alpha')\simeq Ker(\beta').$\\

Si existe $g:C\to C'$ tal que $\beta'\gamma=g \beta$, entonces $\beta'(\gamma \alpha)=(\beta'\gamma)\alpha=g\beta\alpha\\=g0=0$, por la pripiedad 
universal del $Ker(\beta')$ existe una única \\$f:A\to A'$ tal que $\alpha' f=\gamma \alpha$.\\

Ahora, como $\mathscr{C}$ es exacta, por 1.7.3 se tiene el siguiente diagrama con renglones exactos en $\mathscr{C}^{op}$\\

\centerline{
\xymatrix{0\ar[r] & C'\ar[r]^ {(\beta')^{op}} & B'\ar[r]^{(\alpha')^{op}}\ar[d]^{\gamma^{op}} & A'\ar[r] & 0\\
0\ar[r] & C\ar[r]^{\beta^{op}} & B\ar[r]^{\alpha^{op}} & A\ar[r] & 0\,.
}}
Si existiera $f:A\to A'$ tal que $\alpha' f=\gamma\alpha$ entonces existe $f^{op}\in Mor(\mathscr{C}^{op})$ tal que $\alpha^{op}\gamma^{op}=
f^{op}(\alpha')^{op}$. Así, como $\mathscr{C}^{op}$ es exacta y como se tienen las hipótesis de la primera parte de la demostración, 
entonces existe una única $g^{op}:C'\to C$ tal que $\beta^{op}g^{op}=\gamma^{op}(\beta')^{op}=(\beta'\gamma)^{op}$. \\

Por lo tanto existe una única $g:C\to C'$ tal que $(\beta'\gamma)^{op}=\beta^{op}g^{op}=(g\beta)^{op}$ por lo tanto $\beta'\gamma=g\beta$.


\end{proof}



		%35
		\item Construiremos la noción dual a la intersección de una familia de subobjetos.\\
		\textbf{Intersección:} $\catarrow{\mu}{B}{A}{}$ es una intersección para $\lrbrack{\catarrow{\mu_i}{A_i}{A}{m}}$ en $\mathscr{C}$ si \begin{enumerate}[label=Int\Roman*)]
			\item $\forall\ i\in I$ $\exists\ \catarrow{\lambda_i}{B}{A_i}{}$ tal que $\mu=\mu_i\lambda_i$;
			\item si $\catarrow{\nu}{C}{A}{}$ satisface que $\forall\ i\in I$ $\exists\ \catarrow{\eta_i}{C}{A_i}{}$ tal que $\nu=\mu_i\eta_i$, entonces $\exists\ \catarrow{\eta}{C}{B}{}$ tal que $\nu=\mu\eta$.
		\end{enumerate}
		\textbf{Intersección\textsuperscript{op}:} $\catarrow{\opst{\mu}}{B}{A}{}$ es una intersección para $\lrbrack{\catarrow{\opst{\mu_i}}{A_i}{A}{m}}$ en $\mathscr{C}$ si\begin{enumerate}[label=Int\textsuperscript{op}\Roman*)]
			\item $\forall\ i\in I$ $\exists\ \catarrow{\opst{\lambda_i}}{B}{A_i}{}$ tal que $\opst{\mu}=\opst{\mu_i}\opst{\lambda_i}$;
			\item si $\catarrow{\opst{\nu}}{C}{A}{}$ satisface que $\forall\ i\in I$ $\exists\ \catarrow{\opst{\eta_i}}{C}{A_i}{}$ tal que $\opst{\nu}=\opst{\mu_i}\opst{\eta_i}$, entonces $\exists\ \catarrow{\opst{\eta}}{C}{B}{}$ tal que $\opst{\nu}=\opst{\mu}\opst{\eta}$.
		\end{enumerate}
		Así, aplicando el funtor $D_{\opst{\mathscr{C}}}$ a las flechas que aparecen en lo anterior, y sabiendo que el dual de mono es epi, se llega a la siguiente definición
		\textbf{Intersección\textsuperscript{*}:}
		\begin{definesn}
			$\catarrow{\beta}{A}{B}{}$ es una \textbf{cointersección} para $\lrbrack{\catarrow{\beta_i}{A}{A_i}{e}}$ en $\mathscr{C}$ si
		\begin{enumerate}[label=Coint\Roman*)]
			\item $\forall\ i\in I$ $\exists\ \catarrow{\delta_i}{A_i}{B}{}$ tal que $\beta=\delta_i\beta_i$;
			\item si $\catarrow{\omega}{A}{C}{}$ satisface que $\forall\ i\in I$ $\exists\ \catarrow{\gamma_i}{A_i}{C}{}$ tal que $\omega=\gamma_i\beta_i$, entonces $\exists\ \catarrow{\gamma}{B}{C}{}$ tal que $\omega=\gamma\beta$.
		\end{enumerate}
		\end{definesn}		
		%36
				\renewcommand{\copyandpaste}{i\in I}
		\item Sean $\mathscr{C}$ una categoría exacta y $\catarrow{\theta}{A}{A'}{e}$, $\lrbrack{\catarrow{\alpha_i}{A_i}{A}{m}}_{i\in I}$  y, $\forall\ i\in I$, $\beta_i:=coker\lrprth{\alpha_i}$, en $\mathscr{C}$. Si $\theta$ es una cointersección para $\lrbrack{\beta_i}_{i\in I}$, entonces $ker\lrprth{\theta}$ es una unión para $\lrbrack{\alpha_i}_{i\in I}$.
		\begin{proof}
			Denotemos por $k_{\theta}$ un kernel de $\theta$. Se tiene que $k_{\theta}$ es un subobjeto de $A$.\\
			\boxed{I=\varnothing} En este caso, por la vacuidad de $I$, basta con verificar que si $\catarrow{f}{A}{B}{}$ y $\catarrow{\mu}{B'}{B}{m}$, entonces $\theta$ es llevado a $\mu$ vía $f$. Notemos que por vacuidad $f$ satisface la condición CointI) para la familia $\lrbrack{\beta_i}_{\copyandpaste}$, y así por la propiedad universal de la cointersección, CointII), $\exists\ \catarrow{\gamma}{A'}{B}{}$ tal que $f=\gamma \theta$. Con lo cual $fk_{\theta}=f\gamma\lrprth{\theta k_{\theta}}=0$, y por tanto si denotamos por $\rho$ al morfismo $0$ de $A$ en $B'$ se tiene que
			\begin{equation*}
				fk_{\theta}=0=\rho\mu,
			\end{equation*}
			i.e. $\theta$ es llevado a $\mu$ vía $f$.\\
			
			\boxed{I\neq\varnothing} Dado que $\theta$ es una cointersección para $\lrbrack{\beta_i}_{\copyandpaste}$ se tiene en partícular que $\forall\ i\in I\ \exists$ $\catarrow{\eta_i}{\faktor{A}{A_i}}{A'}{}$ tal que $\theta=\eta_i\beta_i$, así
			\begin{align*}
				\theta\alpha_i&=\lrprth{\eta_i\beta_i}\alpha_i=\eta_i\lrprth{\beta_i\alpha_i}=0, && \beta_i=coker\lrprth{\alpha_i}
			\end{align*}
			Luego para cada $i\in I$, por la propiedad universal del kernel, se tiene que $\exists !\ \catarrow{\lambda_i}{A_i}{Ker\lrprth{\theta}}{}$ tal que $\alpha_i=k_{\theta}\lambda_i$. Por lo tanto $\forall\ \copyandpaste$ $\alpha_i\leq k_{\theta}$.\\
			
			Ahora, sean $\catarrow{f}{A}{B}{}$ y $\catarrow{\mu}{B'}{B}{m}$ en $\mathscr{C}$ tales que $\alpha_i$ es llevado a $\mu$ vía $f$, $\forall\ i\in I$, i.e., tales que $\forall\ \copyandpaste$ $\exists\ \catarrow{\rho_i}{A_i}{B'}{}$ de modo que el siguiente diagrama conmuta
			\begin{equation*}\tag{*}\label{pdc}
				\commutativesquare{A=A_i,B=B',C=A,D=B, f=\rho_i,g=\alpha_i,h=\mu,k=f,}
			\end{equation*}
			Si $c_{\mu}$ es un cokernel para $\mu$, entonces por lo anterior se tiene que
			\begin{align*}
				\lrprth{c_{\mu}f}\alpha_i&=\lrprth{c_{\mu}\mu}\rho_i=0, && \forall\ \copyandpaste
			\end{align*}
			Luego, aplicando para cada $i\in I$ la propiedad universal del cokernel, se tiene que $\forall\ \copyandpaste$ $\exists !$ $\catarrow{\chi_i}{\faktor{A}{A_i}}{\faktor{B}{B'}}{}$ tal que 
			\begin{align*}
				c_{\mu}f&=\chi_icoker\lrprth{\mu_i}=\chi_i\beta_i.
			\end{align*}
			Esto último, por la propiedad universal de la cointersección, garantiza que $\exists\ \catarrow{\chi}{A'}{\faktor{B}{B'}}{}$ tal que el siguiente diagrama conmuta
			\begin{equation*}\tag{**}\label{sdc}
				\commutativesquare{C=A',D=\faktor{B}{B'},f=f,g=\theta,h=c_{\mu},k=\chi,}
			\end{equation*}
			De (\ref{pdc}) y (\ref{sdc}) se sigue que
			\begin{align*}
				c_{\mu}\lrprth{fk_{\theta}}&=\chi\lrprth{\theta k_{\theta}}\\
				&=0,
			\end{align*}
			lo cual, en conjunto a que 
			\begin{align*}
				\mu\simeq Im\lrprth{\mu}\simeq Ker\lrprth{Coker\lrprth{\mu}}\simeq Ker\lrprth{c_{\mu}}, && \text{en }\moncategory{\mathscr{C}}{B}
			\end{align*}
			(pues  $\mathscr{C}$ es exacta y $\mu$ es mono) garantiza que por medio de la propiedad universal del kernel $\exists !$ $\catarrow{\rho}{Ker\lrprth{\theta}}{B'}{}$ tal que $fk_{\theta}=\rho\mu$, i.e. el siguiente diagrama conmuta
			\begin{center}
				\commutativesquare{A=Ker\lrprth{\theta},B=B',C=A,D=B,f=\rho,g=k_{\theta},h=\mu,k=f,}
			\end{center}
			y así se tiene lo deseado.\\
		\end{proof}
		%37
\item Sea $\mathscr{C}$ una categoría y $\{A_i\}_{i\in I}$ en $\mathscr{C}$. Pruebe que si $I=\emptyset$, el producto de esa familia (si es que existe) es
un objeto final en $\mathscr{C}$.\\

\textbf{Notación:} En una categoría $\mathscr{C}$ con objeto cero, para cada $\{A_i\}_{i\in I}$ en $\mathscr{C}$, se define la familia de morfismos
$\delta^A_i:=\{\delta^A_{ij}:A_i\to A_j\}_{(i,j)\in I^2}$ en $\mathscr{C}$\\

\centerline{
$\delta^A_{i,j}:=\left\{ 
\begin{array}{c} 0\quad \text{si}\quad i\neq j\\ 1_{A_j}\quad \text{si}\quad i=j\,.
\end{array}\right .$
}
\begin{proof}
Como $I=\emptyset$ por vacuidad se tiene que para todo $C\in \mathscr{C}$, se tiene una familia $\{\alpha_i:C\to A_i\}_{i\in I}$ en $\mathscr{C}$.
Entonces (puesto que el producto existe) existe una única $\alpha:C\to P$ tal que $\pi_i\alpha=\alpha_i\quad \forall i\in I$, donde $\pi_i$ 
son los morfismos que cumplen la propiedad universal del producto. Si existiera otro morfismo $\gamma:C\to P$ éste cumpliría por vacuidad que 
$\pi_i\gamma=\alpha_i\quad \forall i\in I$, y como existe un único morfismo con esta propiedad, tenemos que para cada objeto $C\in \mathscr{C}\quad 
\big|\operatorname{Hom}_{\mathscr{C}}(C,P)\big|=1$ por lo que $P$ es objeto final.
\end{proof}
	%38
\item Pruebe que, $\forall \{A_i\}_{i\in I}$ en Sets, el producto de conjuntos (cartesiano) es el categórico. 

\begin{proof}
Primero se mostrará el caso en que $I\neq \emptyset$ y \\$A_j\neq \emptyset\quad \forall j\in J$, definimos $P=\{f:I\to \displaystyle\bigcup_{i\in I}A_i\}$ y
$ \{\pi_i:P\to A_i\}_{i\in I}$ las funciones tales que para cada 
\[x:I\to \displaystyle\bigcup_{i\in I}A_i, \hspace{1cm} \pi_j(x)=x(j)\in A_j\quad \forall j\in I.
\]
Observemos primero que si $Q=\emptyset$ entonces $\alpha_i$ es la función vacia para cada $i\in I$, entonces tomando $\alpha:Q\to P$ como la función
vacia, se tiene la propiedad universal del producto en $P$. De la misma forma, si alguna $\alpha_j=\emptyset$ para alguna $i\in I$, 
entonces $Q=\emptyset$ y se repite el argumento anterior.\\

Supongamos entoces que existe $Q\in Sets$ con $Q\neq \emptyset$ tal que existe la familia $\{\alpha_i:Q\to A_i\}_{i\in I}$ en $Sets$, 
como $A_j\neq \emptyset\quad \forall j\in I$
entonces $\alpha_i\neq \emptyset\quad \forall i\in I$, además como $\pi_i$ es suprayectiva
para toda $i\in I$, se tiene que para cada $i\in I$ y $\forall q\in Q$, $\alpha_i(q)=\pi_i(r_q)$ para algún $r_q\in P$. 
Así definimos $\alpha:Q\to P$ como $\alpha(q)=r_q$.\\

Se afirma que $\alpha$ está bien definida.\\
En efecto, si $r_q$ y $s_q$ son elementos en $P$ tales que \\$\pi_i(r_q)=\pi_i(s_q)=q\quad \forall i\in I$, entonces $r_q(i)=s_q(i)\quad \forall i\in I$
pero \\$r_q,s_q:I\to \displaystyle\bigcup_{i\in I}A_i$, entonces $r_q=s_q$. Mas aún, $\pi_i\alpha(q)=\pi_i(r_q)=\alpha_i(q).$\\

Supongamos que existe $\beta:Q\to P$ tal que $\pi_i\beta=\alpha_i\quad \forall i\in I$. Por definición, para toda $q\in Q$, $\beta(q)\in P$, 
es decir, $\beta(q)$ es una función con dominio $I$ y contradominio $ \displaystyle\bigcup_{i\in I}A_i$. Así, para toda $i\in I$
\begin{align*}
(\alpha(q))(i)&=\pi_i(\alpha(q))\\
&=\alpha_i(q)\\
&=\pi_i(\beta(q))\\
&=\beta(q)(i).
\end{align*}
Por lo tanto se cumple la unicidad.\\

Supongamos ahora que $A_j=\emptyset$ para alguna $j\in I$. Se tiene entonces que el producto cartesiano de la familia es $\varnothing$, pues no existe
 una función $\catarrow{u}{I}{\bigcup\limits_{i\in I}A_i}{}$ tal que $u\lrprth{j}\in A_j$. Así para cada $i\in I$ denotemos por $\pi_i$ a la función vacía 
de $\varnothing$ en $A_i$. Sean $Q\in Sets$ y $\{\alpha_i:Q\to A_i\}_{i\in I}$ en $Sets$, en partícular se tendría que existe una función de $Q$ en
 $A_j=\varnothing$ y por tanto necesariamente $Q=\varnothing$. Así, $\forall\ i\in I$, $\alpha_i=\pi_i=\pi_i1_\varnothing $, y más aún la función 
identidad $\catarrow{1_\varnothing}{\varnothing}{\varnothing}{}$ es la única que satisface lo anterior puesto que es la única función en
 $\ringmodhom{Sets}{\varnothing}{\varnothing}$. Con lo cual se tiene que $\varnothing$ en conjunto a la familia $\lrbrack{\pi_i}_{i\in I}$ 
es un producto categórico para $\arbtfam{A}{i}{I}$.\\

Por último se mostrará el caso en que $I=\emptyset$.\\

Sea $\{A_i\}_{i\in I}$ una familia de conjuntos. Sea $P$ un conjunto con un único elemento $*$ y $\{\pi_i\,:\,P\to A_i\}_{i\in I}$ una familia vacia 
de funciones. Observamos que, para toda $A\in Sets$ se tiene que: Si $A=\emptyset$, existe $\varphi:A\to P$ la 
función vacia y esta es única.\\
Si $A\neq \emptyset$ existe $\varphi:A\to P$ la función constante, donde $\varphi(x)=*$, esta es única, pues si $f:A\to P$ es función, para toda $r\in A$
se tiene que  $f(a)\in P$, por lo que $f(a)=*$. Así $\varphi$ es única.\\

Ahora, sea $Q\in Sets$ y $\{\alpha_i:Q\to A_i\}_{i\in I}$ en $Sets$, por lo anterior existe una única $\varphi:Q\to P$ y es tal que $\forall i\in I$
$\pi_i\alpha=\alpha_i$ (Como $I$ es vacio esta propiedad se cumple por vacuidad). Así $P$ con la familia $\{\pi_i:P\to A_i\}_{i\in I}$ es un 
producto en $Sets$.\\

Ahora, el producto cartesiano de $\displaystyle\prod_{i\in I}X_i=\{g:I\to \bigcup_{i\in I} X_i\,|\,\forall i\in I\quad g(i)\in X_i\}$, si $I$ es vacio, la única
$g:I\to \bigcup_{i\in I} X_i$ es la función vacia $f_\emptyset$, por lo que $\displaystyle\prod_{i\in I}X_i=\{f_\emptyset:\emptyset\to \emptyset\}=
\{\emptyset\}$, el cual es un conjunto con un único elemento.

\end{proof}

		%39
		\item Sean $\mathscr{C}$ una categoría y $\lrbrack{A_i}_{i\in I}$ en $\mathscr{C}$. Si $I=\varnothing$ y dicha familia admite un coproducto, entonces este es un objeto inicial en $\mathscr{C}$.
		\begin{proof}
			Se tiene que, por definición, dada una familia de objetos $\arbtfam{A}{i}{I}$, un objeto $C$ en conjunto a una familia de morfismos $\lrbrack{\catarrow{\mu_i}{A_i}{C}{}}$ es un coproducto para $\arbtfam{A}{i}{I}$ si $\forall\ B\in\mathscr{text}$ y $\forall\ \lrbrack{\catarrow{\beta_i}{A_i}{B}{}}$ $\exists !$ $\catarrow{\alpha}{C}{B}{}$ tal que $\beta_i=\alpha\mu_i$. De modo que si $I=\varnothing$ lo anterior se reduce a que $\forall\ B\in\mathscr{C}$ existe un único morfismo $\alpha\in\ringmodhom{\mathscr{C}}{C}{B}$, i.e. $C$ es un objeto inicial en $\mathscr{C}$.\\
			Notemos que, más aún, si $C$ es un objeto inicial entonces $C$ en conjunto una familia vacía de morfismos es un coproducto para cualquier familia vacía de objetos en $\mathscr{C}$.\\
		\end{proof}
		%40
		\item  Sean $\mathscr{C}$ una categoría, $C\in\mathscr{C}$ y $\lrbrack{\catarrow{\mu_i}{A_i}{C}{}}_{\copyandpaste}$ en $\mathscr{C}$.  $C$ y $\lrbrack{\catarrow{\mu_i}{A_i}{C}{}}_{\copyandpaste}$ es un coproducto para $\lrbrack{A_i}_{\copyandpaste}$ en $\mathscr{C}$ si y sólo si $C$ y $\lrbrack{\catarrow{\opst{\mu_i}}{C}{A_i}{}}_{\copyandpaste}$ es un producto para $\lrbrack{A_i}_{\copyandpaste}$ en $\opst{\mathscr{C}}$.
		\begin{proof}
			Si $I=\varnothing$ la equivalencia se sigue de los ejercicios 37 y 39, y que $A\in\mathscr{C}$ es un objeto inicial si y sólo si $A\in\opst{\mathscr{C}}$ es un objeto final. En adelante supondremos que $I\neq\varnothing$.
						
			Para la necesidad comencemos notando que $C$ también es un objeto de $\opst{\mathscr{C}}$. Sean $A$ y $\lrbrack{\catarrow{\opst{\gamma}_i}{A}{A_i}{}}_{\copyandpaste}$ en $\opst{\mathscr{C}}$, luego $A$ es un objeto de $\mathscr{C}$ y $\lrbrack{\catarrow{{\gamma}_i}{A_i}{A}{}}$ es una familia de morfismos en $\mathscr{C}$, con lo cual por la propiedad universal del coproducto $\exists !$ $\catarrow{\alpha}{C}{A}{}$ tal que $\forall\ i\in I$ $\gamma_i=\alpha\mu_i$ en $\mathscr{C}$. De modo que $\opst{\alpha}$ satisface que $\opst{\alpha}\in\ringmodhom{\opst{\mathscr{C}}}{A}{C}$ y $\forall\ \copyandpaste$ $\opst{\gamma}_i=\opst{\mu_i}\opst{\alpha}$. Finalmente, si suponemos que $\catarrow{\opst{\beta}}{A}{C}{}$ satisface que $\forall\ \copyandpaste$ $\opst{\gamma}_i=\opst{\mu_i}\opst{\beta}$, entonces $\beta\in\ringmodhom{\mathscr{C}}{C}{A}$ y $\forall\ \copyandpaste$ $\gamma=\beta\mu_i$. De esto último y la unicidad de $\alpha$ se sigue que ${\beta}={\alpha}$ en $\mathscr{C}$, y así  $\opst{\beta}=\opst{\alpha}$ en $\opst{\mathscr{C}}$, con lo cual se tiene lo dseeado.
			
			La suficiencia se verifica en forma análoga, puesto que tomar una familia de morfismos en la categoría $\mathscr{C}$ induce una familia de morfismos en $\opst{\mathscr{C}}$, empleando ahora la propiedad universal del producto.\\
		\end{proof}
		%41
\item Pruebe que $Mod(R)$ tiene productos y coproductos.
\begin{proof}
Por los ejercicios 37 y 39, se tiene que si el producto y el coproducto existen, en el caso de familias no vacias, entonces estos deben ser un objeto inicial y 
un objeto final respectivamente, los cuales para $Mod(R)$ existen y son el objeto cero.\\

Afirmamos entonces que, si $I=\emptyset$, $CP=\{0_R\}$ junto a $\{\pi_i:CP\to A_i\}_{i\in I}$ es el producto de $\{A_i\}_{i\in I}$ y 
 junto a $\{\mu_i:A_i\to CP\}_{i\in I}$ es el coproducto de $\{A_i\}_{i\in I}$ en $Mod(R)$, donde $\pi_i$ y $\mu_i$ son morfismos cero.\\

Sea $Q\in Mod(R)$ entonces $\varphi:Q\to CP$ y $\psi:CP\to Q$ dadas por 
$\varphi(q)=0_R$ y $\psi(0_R)=0_Q$ son $R$-morfismos de módulos, mas aún, son únicos.\\

En particular, si $\{\alpha_i:Q\to A_i\}_{i\in I}$ y $\{\beta:A_i\to Q\}$ son familias de morfismos en $Mod(R)$, por vacuidad de $I$ se cumple que 
$\pi_i\varphi=\eta_i$ y $\psi\mu_i=\beta_i\quad \forall i\in I$. Por lo tanto $CP$ es producto y coproducto de $\{A_i\}$.\\

Consideremos $I\neq \emptyset$ un conjunto, y a $\{A_i\}_{i\in I}$ una familia de $R$-módulos. Veamos que existe el producto.\\
Sea $P$ el producto cartesiano de conjuntos, es decir, $P=\{f:I\to \displaystyle\bigcup_{i\in I}A_i\}$ y sean $\{\pi_i:P\to A_i\}_{i\in I}$ las funciones
tales que para cada\\ $x:I\to  \displaystyle\bigcup_{i\in I}A_i,\quad \pi_j(x)=x(j)\in A_j\quad \forall j\in I$.\\

Por el ejercicio 38 sabemos que $\pi_i$ está bien definida para cada $i\in I$, veamos que es morfismo. Sean $r\in R$ y $a,b\in P$ (se dará por hecho
que $P$ es un $R$-módulo), entonces 
\[\pi_i(ra+b)=(ra+b)(i)=ra(i)+b(i)=r\pi_i(a)+\pi_i(b).
\]
Por lo tanto para toda $i\in I$, $\pi_i\in Mor(Mod(R))$.\\

Ahora, puesto que todo morfismo de $R$- módulos es función, e $I\neq \emptyset$, por el ejercicio 38 si $Q\in Mod(R)$ es tal que existe una familia\\
$\{\alpha_i:Q\to A_i\}$ en $Mod(R)$, entonces existe una única función $\alpha:Q\to P$ tal que $\pi_i\alpha=\alpha_i\quad \forall i\in I$ definida como 
$\alpha(q)=r_q$ con $q\in Q$ y $r_q\in P$ tales que $\pi_i(r_q)=\alpha_i(q).$\\

Esta función está bien definida, solo es necesario probar que es morfismo de $R$-módulos. Sean $s\in R$, y $a,b\in Q$, entonces, como $\pi_i$ es morfismo
para cada $i\in I$, se tiene que si $(s r_a+r_b)\in P$ se cumple que
\[\pi_i(sr_a+r_b)=s\pi_i(r_a)+\pi_i(r_b)=s\alpha_i(a)+\alpha_i(b)=\alpha_i(sa+b).
\]
Por lo tanto $r_{sa+b}=sr_a+r_b$, y así $\alpha(sa+b)=s\alpha(a)+\alpha(b)$, es decir, $\alpha$ es morfismo y en consecuencia $P$ es el producto\\
 categórico. \\

Veamos que existe el coproducto para $I\neq \emptyset$. Se afirma que \\$\displaystyle\sum_{i\in I}A_i\in Mod(R)$ junto con la familia
$\{\mu_i:A_i\to \displaystyle\sum_{i\in I}A_i\}_{i\in I}$ donde $\mu_i(a)=a$, "la inclusión natural", son un coproducto de $\{A_i\}_{i\in I}$.\\

Sean $B$ y $\{\beta_i:A_i\to B\}_{i\in I}$ una familia de morfismos. Como $I\neq \emptyset$ y $A_i,B$ son módulos para toda $i\in I$, 
$\beta_i$ no puede ser la función vacia para ninguna $i\in I$. Así, podemos tomar $x\in \displaystyle\sum_{i\in I}A_i$, es decir, 
$x=x_{i_1}+x_{i_2}\ldots +x_{i_n}$ para alguna $n\in \N,\quad i_k\in I$ y $x_{i_k}\in A_{i_k}\quad \forall k\in\{1,2\ldots,n\}$.\\

Definimos $\beta:\displaystyle\sum_{i\in I}A_i\longrightarrow B$ como $\beta(x)=\displaystyle\sum_{k=1}^n\beta_{i_k}(x_{i_k})$.\\

Como $\beta_{i_k}$ es morfismo $\forall i_k\in I$ entonces $\beta$ es morfismo de $R$-módulos. Además, $\forall i\in I$, si $x\in A_i$, se tiene que
$\beta\mu_i(x)=\beta(x)=\beta_i(x)$, por lo que $\beta\mu_i=\beta_i\quad \forall i\in I$.\\

Mas aún, si $\gamma:\displaystyle\sum_{i\in I}A_i\longrightarrow B$ es un morfismo tal que $\gamma\mu_i=\beta_i\quad \forall i\in I$, 
entonces, si $x\in \displaystyle\sum_{i\in I}A_i$ (y usando la descripción de la "$x$" que usamos anteriormente), 

\begin{align*}
\gamma(x)&=\gamma\left(\displaystyle\sum_{k=1}^nx_{i_k}\right)=\displaystyle\sum_{k=1}^n\gamma(x_{i_k})\\
&=\displaystyle\sum_{k=1}^n\gamma\mu_{i_k}(x_{i_k})=\displaystyle\sum_{k=1}^n\beta_{i_k}(x_{i_k})\\
&=\displaystyle\sum_{k=1}^n\beta\mu_{i_k}(x_{i_k})=\displaystyle\sum_{k=1}^n\beta(x_{i_k})\\
&=\beta\displaystyle\sum_{k=1}^n(x_{i_k})=\beta(x).
\end{align*}
Por lo que $\beta$ es única, y así $\displaystyle\sum_{i\in I}A_i$ es un coproducto.
\end{proof}

		%42

\item Sean $\mathscr{C}$ una categoría, $\lrbrack{\catarrow{\mu_i}{A_i}{\coprod\limits_{i\in I}A_i}{}}$ un coproducto en $\mathscr{C}$,
$C\in\mathscr{C}$ y $\lrbrack{\catarrow{\nu_i}{A_i}{C}{C}}_{\copyandpaste}$ en $\mathscr{C}$. Pruebe que las siguientes condiciones 
son equivalentes:
\begin{enumerate}[label=\textit{\alph*)}]
	\item $C$ y $\lrbrack{\catarrow{\nu_i}{A_i}{C}{C}}_{\copyandpaste}$ son un coproducto de $\lrbrack{A_i}_{\copyandpaste}$;
	\item $\exists\ \catarrow{\varphi}{\coprod\limits_{i\in I}A_i}{C}{i}$ tal que  $\varphi\mu_i=\nu_i$\quad $\forall\ i\in I$.
\end{enumerate}
\end{enumerate}		
\begin{proof}
Supongamos $C$ y $\{\nu_i:A_i\to C\}_{i\in I}$ son un coproducto de $\{A_i\}_{i\in I}$ entonces, como $\displaystyle\coprod_{i\in I}A_i$ es un
coproducto, existe una única \\$\alpha:\displaystyle\coprod_{i\in I}A_i\longrightarrow C$ tal que $\alpha\mu_i=\nu_i\quad i\in I$.\\
De la misma forma, como $\{\nu_i:A_i\to C\}_{i\in I}$ es un coproducto para $\{A_i\}_{i\in I}$, existe un único $\beta:C\to \displaystyle\coprod_{i\in I}A_i$
tal que $\beta\nu_i=\mu_i\quad \forall i\in I$.\\

Notemos ahora que $\beta\alpha:\displaystyle\coprod_{i\in I}A_i\longrightarrow \displaystyle\coprod_{i\in I}A_i$  es tal que 
\[(\beta\alpha)\mu_i=\beta(\alpha\mu_i)=\beta\nu_i=\mu_i.
\]
Pero $\displaystyle\coprod_{i\in I}A_i$ es coproducto, entonces sólo existe un morfismo con dicha propiedad, el cual, en este caso, sería 
$1_{\coprod A_i}$. Por lo tanto $\beta\alpha=1_{\coprod A_i}$. Análogamente \\$\alpha\beta:C\to C$ es tal que 
\[(\alpha\beta)\nu_i=\alpha(\beta\nu_i)=\alpha\mu_i=\nu_i\]
y como $C$ es coproducto $\alpha\beta=1_C.$ Así $\alpha:\displaystyle\coprod_{i\in I}A_i\to
C$ es un isomorfismo tal que $\mu_i=\nu_i\quad \forall i\in I$.\\

Supongamos ahora que existe $\catarrow{\varphi}{\coprod\limits_{i\in I}A_i}{C}{i}$ tal que $\varphi\mu_i=\nu_i\quad \forall i\in I$.\\

Sea $M\in \mathscr{C}$ y $\{\eta_i:A_i\to M\}_{i\in I}$ una famiilia en $\mathscr{C}$. Como $\displaystyle\coprod_{i\in I}A_i$ es coproducto
existe una única $\alpha:\displaystyle\coprod_{i\in I}A_i\longrightarrow M$ tal que $\alpha\mu_i=\eta_i\quad \forall i\in I$.\\
Tomando $\beta:=\alpha\varphi^{-1}:C\to M$ se tiene que \[\beta\nu_i=\alpha(\varphi^{-1}\nu_i)=\alpha\mu_i=\eta_i\,\quad \forall i\in I,\] entonces 
$C$, $\{\nu_i:A_i\to C\}_{i\in I}$ son un coproducto de $\{A_i\}_{i\in I}.$
\end{proof}

\end{document}