\documentclass{article}
\usepackage[utf8]{inputenc}
\usepackage{mathrsfs}
\usepackage[spanish,es-lcroman]{babel}
\usepackage{amsthm}
\usepackage{amssymb}
\input{C:/Users/HP/Desktop/respaldo/Documents/Maestria/S2/Algebras de Artin/Comandos Matemagicos}
\begin{document}
	Por último supongamos que $A_j=\emptyset$ para alguna $j\in I$. Se tiene entonces que el producto cartesiano de la familia es $\varnothing$, pues no existe una función $\catarrow{u}{I}{\bigcup\limits_{i\in I}A_i}{}$ tal que $u\lrprth{j}\in A_j$. Así para cada $i\in I$ denotemos por $\pi_i$ a la función vacía de $\varnothing$ en $A_i$. Sean $Q\in Sets$ y $\{\alpha_i:Q\to A_i\}_{i\in I}$ en $Sets$, en partícular se tendría que existe una función de $Q$ en $A_j=\varnothing$ y por tanto necesariamente $Q=\varnothing$. Así, $\forall\ i\in I$, $\alpha_i=\pi_i=\pi_i1_\varnothing $, y más aún la función identidad $\catarrow{1_\varnothing}{\varnothing}{\varnothing}{}$ es la única que satisface lo anterior puesto que es la única función en $\ringmodhom{Sets}{\varnothing}{\varnothing}$. Con lo cual se tiene que $\varnothing$ en conjunto a la familia $\lrbrack{\pi_i}_{i\in I}$ es un producto categórico para $\arbtfam{A}{i}{I}$.	
\end{document}
