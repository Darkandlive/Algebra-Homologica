\documentclass{article}
\usepackage[utf8]{inputenc}
\usepackage{mathrsfs}
\usepackage[spanish,es-lcroman]{babel}
\usepackage{amsthm}
\usepackage{amssymb}
\usepackage{enumitem}
\usepackage{graphicx}
\usepackage{caption}
\usepackage{float}
\usepackage{amsmath,stackengine,scalerel,mathtools}
\usepackage{xparse, tikz-cd, pgfplots,tikz}
\usepackage{mathrsfs}
\usepackage{comment}
\usepackage{faktor}
\usepackage[all]{xy}

\input{C:/Users/Darkandlive/Desktop/Latex/Álgebra/Tercer semestre/Comandos Matemagicos}

\title{Ejercicios 1-15}
\author{Arruti Sebastián Luis Gerardo}
\begin{document}
	\begin{enumerate}[label=\textbf{Ej \arabic*.}]
		\item Pruebe que en una categoría $\mathscr{C}$, vale lo siguiente
        \begin{itemize}
	\item[a)] La composición de monomorfismos (epimorfismos) es un monomorfismo (epimorfismo).
	\item[b)] Todo split-epi (split-mono) es un epimorfismo (monomorfismo).
	\item[c)] $fg$ mono $\Rightarrow$ $g$ es mono.
	\item[d)] $fg$ epi $\Rightarrow$ $f$ es epi.
	\item[e)] $f$ es iso $\Rightarrow$ $f$ es epi y mono.
	\item[f)]  $f$ es iso $\iff$ $f$ es split-mono y split-epi.
\end{itemize}
\begin{proof}
Sea $\mathscr{C}$ una categoría.\\
\boxed{a)} Supongamos que $f\colon A\longrightarrow B$ y $g\colon B\longrightarrow C$ son monomorfismos y que para todo $\alpha,\beta\in
Mor(\mathscr{C})$ se tiene que que $gf\alpha =gf\beta$, entonces considerando a los morfismos $f\alpha$ y $f\beta$ tenemos que, como $g$ es mono, 
$f\alpha=f\beta$, análogamente, como $f$ es mono, entonces $\alpha=\beta$ por lo que $gf$ es mono.\\

Ahora supongamos que $f\colon A\longrightarrow B$ y $g\colon B\longrightarrow C$ son epimorfismos y que para toda $\alpha,\beta\in
Mor(\mathscr{C})$ se tiene que que $\alpha gf =\beta gf$. Como $f$ es epi, entonces $\alpha g=\beta g$ y por ser $g$ epi $\alpha=\beta$ por lo tanto 
$gf$ es epi.\\

\boxed{b)} Supongamos $f\colon A\longrightarrow B$ es split-epi, entonces existe $f'\colon B\longrightarrow A$ tal que $ff'=1_B$ así, si 
$h,g\in Mor(\mathscr{C})$ son tales que $hf=gf$ entonces $hff'=gff'$, es decir, $h1_B=g1_b$ y por lo tanto $h=g$. Entonces $f$ es epi.\\

 Supongamos ahora que $f\colon A\longrightarrow B$ es split-mono, entonces existe \\ $f'\colon B\longrightarrow A$ tal que $f'f=1_A$ así, si 
$h,g\in Mor(\mathscr{C})$ son tales que $fh=fg$ entonces $f'fh=f'fg$, es decir, $1_A h=1_A g$ y por lo tanto $h=g$. Entonces $f$ es mono.\\

\boxed{c)}
Supongamos $fg$ es mono con $f,g\in Mor(\mathscr{C})$, entonces para todo $k,h\in Mor(\mathscr{C})$, si $gk=gh$ se tiene que $fgk=fgh$ y como
$fg$ es mono entonces $k=h$ por lo tanto $g$ es mono.\\

\boxed{d)}
Supongamos $fg$ es epi con $f,g\in Mor(\mathscr{C})$, entonces para todo \\ $k,h\in Mor(\mathscr{C})$, si $kf=hf$ se tiene que $kfg=hfg$ y como
$fg$ es epi entonces $k=h$ por lo tanto $f$ es epi.\\

\boxed{f)}
Supongamos que $f\colon A\longrightarrow B$ es iso, entonces 
\begin{gather*}
\exists g\colon B\longrightarrow A\quad \text{tal que}\quad fg=1_B,\,\, gf=1_A,\\
\text{entonces}\quad
\exists g\colon B\longrightarrow A\quad \text{tal que}\quad fg=1_B\\
\quad\text{y}\quad\quad \exists g\colon B\longrightarrow A\quad \text{tal que}\quad gf=1_A
\end{gather*}
por lo que  $f$ es split-epi y $f$ es split-mono.\\

Ahora supongamos que $f\colon A\longrightarrow B$  es split-mono y split-epi. Entonces existen $g_1\colon B\longrightarrow A$ y 
$g_2\colon B\longrightarrow A$ tales que $fg_1=1_B$ y $g_2f=1_A$.\\
Como $fg_1=1_B$ entonces aplicando $g_2$ por la izquierda se tiene que $g_2fg_1=g_21_B$, así $1_Ag_1=g_21_B$.
Por lo tanto $g_1=g_2$ y así $f$ es iso.\\

\boxed{e)} Este inciso es consecuencia de f) y b).
\end{proof}
		\item Para $f\colon A\longrightarrow B$ en una categoría $\mathscr{C}$, pruebe que:
\begin{itemize}
\item[a)] $f$ es un monomorfismo $\iff\quad \forall X\in \mathscr{C},\\ Hom_{\mathscr{C}}(X,f):Hom_{\mathscr{C}}(X,A)\longrightarrow
Hom_{\mathscr{C}}(X,B)$ es inyectivo.\\
\item[b)] $f$ es un epimorfismo $\iff\quad \forall X\in \mathscr{C}, \\Hom_{\mathscr{C}}(f,X):Hom_{\mathscr{C}}(B,X)\longrightarrow
Hom_{\mathscr{C}}(A,X)$ es suprayectivo.
\end{itemize}
\begin{proof}
\boxed{a)} Supongamos $f$ es mono y sean $A,B,X\in \mathscr{C}$. Si \\ $g\in Hom_{\mathscr{C}}(X,A)$ entonces $fg\in Hom_{\mathscr{C}}(X,B)$.\\
Ahora, si $ Hom_{\mathscr{C}}(X,f)(\alpha)=Hom_{\mathscr{C}}(X,f)(\beta)$ para  $\alpha,\beta\in Hom_{\mathscr{C}}(X,A)$, entonces
$f\alpha=f\beta$, pero $f$ es mono, así $\alpha=\beta$ y por lo tanto $Hom_{\mathscr{C}}(X,f)$ es inyectivo.\\

Supongamos ahora que $Hom_\mathscr{C}(X,f)$ es inyectivo para toda $X\in \mathscr{C}$.\\
Si $\alpha\beta\in Mor(\mathscr{C})$ son tales que $f\alpha =f\beta $...(1) entonces $\exists X\in \mathscr{C}$ tal que
 $\alpha,\beta\in Hom_{\mathscr{C}}(X,A)$ más aun, (1) implica que \\ $Hom_{\mathscr{C}}(X,f)(\alpha)=Hom_{\mathscr{C}}(X,f)(\beta)$ y como
$\forall X\in \mathscr{C}$, $Hom_{\mathscr{C}}(X,f)$ es inyectivo, entocnes $\alpha=\beta$ por lo tanto $f$ es mono.\\

\boxed{b)} Supongamos $f$ es epi y sean $A,B,X\in \mathscr{C}$. \\
Si $ Hom_{\mathscr{C}}(f,X)(\alpha)=Hom_{\mathscr{C}}(f,X)(\beta)$ con $\alpha,\beta\in Hom_{\mathscr{C}}(B,X)$, entonces
$\alpha f=\beta f$, pero $f$ es epi, así $\alpha=\beta$ y por lo tanto $Hom_{\mathscr{C}}(f,X)$ es inyectivo.\\

Supongamos ahora que $Hom_\mathscr{C}(f,X)$ es inyectivo para toda $X\in \mathscr{C}$.\\
Si $\alpha\beta\in Mor(\mathscr{C})$ son tales que $\alpha f=\beta f$...(2) entonces $\exists X\in \mathscr{C}$ tal que
 $\alpha,\beta\in Hom_{\mathscr{C}}(A,X)$ más aun, (2) implica que $Hom_{\mathscr{C}}(f,X)(\alpha)=Hom_{\mathscr{C}}(f,X)(\beta)$ y como
$\forall X\in \mathscr{C}$, $Hom_{\mathscr{C}}(f,X)$ es inyectivo, entocnes $\alpha=\beta$ por lo tanto $f$ es epi.


\end{proof}


		\item Sea $\catarrow{f}{A}{B}{}$ en una categoría $\mathscr{C}$, así:
		\begin{enumerate}
			\item si $f$ es un split-epi y monoformismo, entonces $f$ es un isomorfismo;
			\item si $\catarrow{F}{\mathscr{C}}{\mathscr{D}}{}$ es un funtor y $f$ es un isomorfismo, split-mono o split-epi, entonces $F\lrprth{f}$ también lo es.
		\end{enumerate}
		\begin{proof}
			\boxed{a)} Como $f$ es un split-epi $\exists\ g\in\ringmodhom{\mathscr{C}}{B}{A}$ tal que $fg=1_B$. Notemos que
			\begin{align*}
				f\lrprth{gf}&=\lrprth{fg}f=1_B f=f=f1_A\\
				\implies gf&=1_A, && f\text{ es mono}\\
				\therefore\ & f \text{ es un isomorfismo.}
			\end{align*}
			\boxed{b)} Supongamos que $f$ es un split-mono, entonces $\exists\ \catarrow{g}{B}{A}{}$ en $\mathscr{C}$ tal que $gf=1_A$, con lo cual $\catarrow{Ff}{FA}{FB}{}$, $\catarrow{Fg}{FB}{FA}{}$ en $\mathscr{D}$ y
			\begin{align*}
				F\lrprth{g}F\lrprth{f}&=F\lrprth{gf}=F\lrprth{1_A}=1_{F\lrprth{A}}\\
				\implies & F\lrprth{f}\text{ es un split-mono.}
			\end{align*}
			Supongamos ahora que $f$ es un split-epi, luego $\exists\ \catarrow{g}{B}{A}{}$ en $\mathscr{C}$ tal que $fg=1_B$, con lo cual $\catarrow{Ff}{FA}{FB}{}$, $\catarrow{Fg}{FB}{FA}{}$ en $\mathscr{D}$ y
			\begin{align*}
				F\lrprth{f}F\lrprth{g}&=F\lrprth{fg}=F\lrprth{1_B}=1_{F\lrprth{B}}\\
				\implies & F\lrprth{f}\text{ es un split-epi.}
			\end{align*}   
			De lo anterior, en conjunto a la equivalencia dada en el Ej. 1 (f), se sigue que si $f$ es un isomorfismo en $\mathscr{C}$ entonces $F\lrprth{f}$ lo es en $\mathscr{D}$.\\
		\end{proof}
		\item Sean $\mathscr{A}$ y $\mathscr{B}$ categorías.
		\begin{enumerate}
			\item Sea $\eta\in\nattrans{\mathscr{A}}{\mathscr{B}}{F}{G}$. Si $\forall\ A\in\mathscr{A}$ $\catarrow{\eta_{A}}{FA}{GA}{}$ es un isomorfismo en $\mathscr{B}$ y $\eta^{-1}:=\lrbrack{\lrprth{\eta^{-1}}_A}_{A\in\mathscr{A}}$, con $\lrprth{\eta^{-1}}_A:=\lrprth{\eta_{A}}^{-1}$, entonces $\eta^{-1}\in\nattrans{\mathscr{A}}{\mathscr{B}}{G}{F}$.
			\item Si $\eta\in\nattrans{\mathscr{A}}{\mathscr{B}}{F}{G}$, $\rho\in\nattrans{\mathscr{A}}{\mathscr{B}}{G}{H}$ entonces la composición de transformaciónes naturales, con $\rho\eta$ dada por $\lrprth{\rho\eta}_A:=\rho_A\circ\eta_A$ $\forall\ A\in\mathscr{A}$, es una operación asociativa.
			\item Si $T\in\lrsqp{\mathscr{A},\mathscr{B}}$ y $\catarrow{1_T}{T}{T}{}$ está dada por $\lrprth{1_T}_A:=1_{T\lrprth{A}}$ $\forall\ A\in\mathscr{A}$, entonces $1_T\in\nattrans{\mathscr{A}}{\mathscr{B}}{T}{T}$.
			\item Si $\alpha\in\nattrans{\mathscr{A}}{\mathscr{B}}{F}{G}$, entonces
			\begin{equation*}
				\alpha 1_F=\alpha=1_G \alpha.
			\end{equation*}
		\end{enumerate}
		\begin{proof}
			\boxed{a)} Dado que $\forall\ A\in\mathscr{A}$ $\catarrow{\eta_{A}}{FA}{GA}{}$ es un isomorfismo en $\mathscr{B}$, se tiene que $\lrprth{\eta_{A}}^{-1}\in\ringmodhom{\mathscr{B}}{GA}{FA}$ y que si $\catarrow{\alpha}{A}{A'}{}$ está en $\mathscr{A}$, entonces
			\begin{align*}
				G\lrprth{\alpha}\eta_{A}&=\eta_{A'}F\lrprth{\alpha}, && \eta\in\nattrans{\mathscr{A}}{\mathscr{B}}{F}{G}\\
				\implies& F\lrprth{\alpha}\lrprth{\eta_{A}}^{-1}=\lrprth{\eta_{A'}}^{-1}G\lrprth{\alpha}.
			\end{align*} 
		 Así $\catarrow{\eta^{-1}}{G}{F}{}$ es una transformación natural, pues lo anterior garantiza que el siguiente diagrama conmuta
		 \begin{equation*}
		 	\commutativesquare{A=G\lrprth{A},B=F\lrprth{A},C=G\lrprth{A'},D=F\lrprth{A'},f=\lrprth{\eta_{A}}^{-1},g=G\lrprth{\alpha},h=F\lrprth{\alpha},k=\lrprth{\eta_{A'}}^{-1},}.
		 \end{equation*}
	 	\boxed{b)} Notemos que $\forall\ A\in\mathscr{A}$ se tiene que $\rho_A\eta_{A}\in\ringmodhom{\mathscr{B}}
	 	{F\lrprth{A}}{G\lrprth{A}}$. Además si $\catarrow{\alpha}{A}{A'}{}$ está en $\mathscr{A}$, por ser $\eta$ y $\rho$ transformaciones naturales, se tiene que $G\lrprth{\alpha}\eta_{A}=\eta_{A'}F\lrprth{\alpha}$ y $H\lrprth{\alpha}\rho_{A}=\rho_{A'}G\lrprth{\alpha}$, con lo cual
	 	\begin{align*}
	 		H\lrprth{\alpha}\lrprth{\rho_A\eta_A}&=\lrprth{H\lrprth{\alpha}\rho_{A}}\eta_{A}=\lrprth{\rho_{A'}G\lrprth{\alpha}}\eta_{A}\\&=\rho_{A'}\lrprth{G\lrprth{\alpha}\eta_{A}}=\rho_{A'}\lrprth{\eta_{A'}F\lrprth{\alpha}}\\&=\lrprth{\rho_{A'}\eta_{A'}}F\lrprth{\alpha},
	 	\end{align*}
 		de modo que el siguiente diagrama conmuta
 		\begin{equation*}
 			\commutativesquare{A=F\lrprth{A},B=H\lrprth{A},C=F\lrprth{A'},D=H\lrprth{A'},f=\rho_{A}\eta_{A},g=F\lrprth{\alpha},h=G\lrprth{\alpha},k=\rho_{A'}\eta_{A'},},
 		\end{equation*}
 		y por lo tanto $\catarrow{\rho\eta}{F}{H}{}$ es una tranformación natural.\\
 		Verificaremos ahora que la composición de transformaciones naturales es asociativa. Si $\rho$ y $\eta$ están dados como al comienzo, $I\in\lrsqp{\mathscr{A},\mathscr{B}}$ y $\catarrow{\chi}{H}{I}{}$  es una transformación natural, entonces si $A\in\mathscr{A}$
 		\begin{align*}
 			\chi_A\lrprth{\rho_{A}\eta_{A}}&=\lrprth{\chi_A\rho_{A}}\eta_{A}\in\lrprth{\chi\rho}\eta,\\
 			\implies & \chi\lrprth{\rho\eta}\subseteq\lrprth{\chi\rho}\eta.
 		\end{align*}
 		En forma análoga se verifica la otra contención, y así se tiene que $\chi\lrprth{\rho\eta}=\lrprth{\chi\rho}\eta$.\\
 		\boxed{c)} Si $\catarrow{\alpha}{A}{A'}{}$ está en $\mathscr{A}$, entonces
 		\begin{align*}
 			T\lrprth{\alpha}1_{T\lrprth{A}}&=T\lrprth{\alpha}\\
 			&=1_{T\lrprth{A'}}T\lrprth{\alpha},
		\end{align*}
		luego\begin{equation*}
			\commutativesquare{A=T\lrprth{A},B=T\lrprth{A},C=T\lrprth{A'},D=T\lrprth{A'},f=1_{T\lrprth{A}},g=T\lrprth{\alpha},h=T\lrprth{\alpha},k=1_{T\lrprth{A'}},}
		\end{equation*}
		conmuta, y por tanto $\catarrow{1_T}{T}{T}{}$ es una transformación natural.\\
		\boxed{d)} Se tiene que $\forall\ A\in\mathscr{A}$ $\alpha_A1_{F\lrprth{A}}=\eta_A$, con lo cual $\lrprth{\alpha 1_F}_A=\alpha_A$ y por tanto $\alpha 1_F=\alpha$. Análogamente se verifica que $1_G\alpha=\alpha$.
		\end{proof}

\item Sea $F\colon\mathscr{A}\longrightarrow\mathscr{B}$ funtor. Pruebe que $F$ es una equivalencia si y sólo si $F$ es fiel, pleno y denso.\\
\begin{proof}
\boxed{\Rightarrow} Supongamos $F$ es equivalencia, entonces existe \\ $G\colon\mathscr{B}\longrightarrow\mathscr{A}$ tal que existen isomorfismos
$\psi\colon GF\longrightarrow 1_{\mathscr{A}}$ y \\ $\varphi\colon 1_{\mathscr{B}}\longrightarrow FG$.\\

Fiel: \\
Sean $X,Y\in \mathscr{A}$ y consideremos que $F\colon Hom_{\mathscr{A}}(X,Y)\longrightarrow Hom_{\mathscr{B}}(FX,FY)$.
Supongamos que existen $g,h\in Hom_{\mathscr{A}}(X,Y)$ tales que $F(g)=F(h)$, entonces $G(F(g))=G(F(h))$. 
Primero observamos que, si $Hom_{\mathscr{A}}(X,Y)=\emptyset$, entonces $F$ es mono en la categoría Sets por vacuidad.\\

Ahora, si $Hom_{\mathscr{A}}(X,Y)\neq \emptyset$, entonces dado $\alpha\colon A\longrightarrow A'$ en $\mathscr{A}$ se tiene que, por ser $F$ 
equivalencia, el siguiente diagrama conmuta:
\begin{equation*}	\commutativesquare{A=GF(A),B=A,C=GF(A'),D=A',f=\psi_A,
g=GF(\alpha),h=\alpha,k=\psi_{A'},}
\end{equation*}
es decir, $\alpha \psi_A=\psi_{A'}GF(\alpha)$. Por lo tanto $\alpha=\psi_{A'}GF(\alpha)\psi^{-1}_A$.\\

Entonces $g=\phi_YGF(g)\psi_X=\phi_YGF(h)\psi_X=h$, por lo que $F$ es fiel. Observese que Análogamente se demuestra que $G$ es un funtor fiel.\\

Pleno:\\
Supongamos $Hom_{\mathscr{B}}(F(X),F(Y))=\emptyset$, entonces $Hom_{\mathscr{A}}(X,Y)=\emptyset$, por lo que $F$ es la función vacia, la cual
es vacia pues cada que $\alpha f= \beta F$ se tiene que $\alpha=\beta=\emptyset$ la función vacia.\\
Ahora, si $Hom_{\mathscr{B}}(F(X),F(Y))\neq \emptyset$ podemos considerar a $h\in Hom_{\mathscr{B}}(F(X),F(Y))$, entonces 
$G(h)\in Hom_{\mathscr{A}}(X,Y)$ y es tal que los siguientes diagramas conmutan:


\centerline{\xymatrix{GF(X) \ar[d]_{G(h)}\ar[r]^{\psi_X} & X \ar[d]^{\alpha}\ar[r]^{\psi_X^{-1}} & GF(X)\ar[d]^{GF(\alpha)} \\
GF(Y) \ar[r]_{\psi_Y} & Y \ar[r]_{\psi_Y^{-1}} & GF(Y) }}
entonces $G(h)=GF(\alpha)$, pero $G$ es fiel, entonces $h=F(\alpha)$ y por lo tanto $F$ es pleno.\\

\boxed{\Leftarrow} Supongamos F es fiel, pleno y denso.\\
Entonces $F$ es mono y epi en Sets, así para cada
$X,Y\in \mathscr{A},\\ F\colon Hom_{\mathscr{A}}(X,Y)\longrightarrow Hom_{\mathscr{B}}(F(X),F(Y))$ es mono y epi en Sets, por lo tanto 
es isomorfismo en Sets para cada $X,Y\in \mathscr{A}$.\\
Ahora, como $F$ es denso, para toda $B\in \mathscr{B}$ existe $A\in \mathscr{A}$ tal que $F(A)\cong B$, así para cada $B\in \mathscr{B}$ podemos
fijar un obgeto $G(B)\in \mathscr{A}$ y un isomorfismo $\gamma_B\colon F(A)\longrightarrow B$.\\
Así para cada \begin{tikzcd} B\arrow{r}{\beta} &B' \end{tikzcd} en $\mathscr{B}$, se tiene que el siguiente diagrama conmuta:
\begin{equation*}
\commutativesquare{A=B,B=FG(B)\text{=}F(A),C=B',D=FG(B')\text{=}F(A'),f=\gamma^{-1}_B,g=\beta,h=\alpha,k=\gamma^{-1}_{B'},}
\end{equation*}
donde $\alpha=\gamma^{-1}_{B'} \beta \gamma_B$. Así $\alpha\colon FG(B)\longrightarrow FG(B')$.\\
Como $F\colon  Hom_{\mathscr{A}}(A,A')\longrightarrow Hom_{\mathscr{B}}(F(A),F(A'))$ es iso, existe un único morfismo $G(\beta)\colon
G(B)\longrightarrow G(B')$ tal que $\gamma^{-1}_{B'} \beta \gamma_B=F(G(\beta))$. En otras palabras para cada $\beta\colon B\longrightarrow B'$
existe un único morfismo \\ $G(\beta)\colon G(B)\longrightarrow G(B')$ en $\mathscr{A}$ tal que el siguiente diagrama conmuta
\begin{equation*}
\commutativesquare{A=B,B=FG(B),C=B',D=FG(B'),f=\gamma^{-1}_B,g=\beta,h=F(G(\beta)),k=\gamma^{-1}_{B'},}\quad\ldots (1)
\end{equation*}

Ahora veamos que $G$ es funtor.\\
Tomando $\beta=1_B$ en el diagrama (1) se tiene que existe un único \\ $G(1_B)\colon G(B)\longrightarrow G(B')$ tal que $FG(1_B)=1_B$, pero 
como $F$ es pleno, entonces $G(1_B)=1_{G(B)}$.\\
Para probar que $G$ preserva la composición tomaremos $\beta\colon B\longrightarrow B'$ y $\alpha\colon B'\longrightarrow B''$
 morfismos en $\mathscr{B}$, como $F$ es fiel y pleno existe 
un único morfismo $G(\alpha\beta)\colon G(B)\longrightarrow G(B'')$ en $\mathscr{A}$ tal que \\
$\gamma_{B''}(\alpha\beta)\gamma^{-1}_B=F(G(\alpha\beta))$; pero también se tiene que $\gamma_B(\alpha\beta)\gamma^{-1}
=F(G\alpha))F(G(\beta))$. Por lo que $F(G(\alpha\beta))=F(G(\alpha))F(G(\beta))=F(G(\alpha)G(\beta))$. Y como $F$ es fiel, $G(\alpha\beta)
=G(\alpha)G(\beta)$, por lo que $G\colon\mathscr{B}\longrightarrow\mathscr{A}$ es funtor.\\

Por el diagrama (1) se puede apreciar que $\gamma=\{\gamma_\beta\colon B\longrightarrow FG(B)\}, \\ 
(\gamma\colon 1_{\mathscr{B}}\longrightarrow FG)$ es una equivalencia natural.
Entonces para cada $A\in \mathscr{A}$ se tiene el isomorfismo $\gamma_{F(A)}\colon F(A)\longrightarrow FGF(A)$ en $\mathscr{B}$; 
en particular, como $F$ es fiel, existe $\psi'_A\colon A\longrightarrow GF(A)$ tal que $F(\psi'_A)=\gamma_{F(A)}$.\\
Por otro lado, como $\gamma_{F(A)}$ es un isomorfismo, entonces existe \\ $\gamma^{-1}_{F(A)}\colon FGF(A)\longrightarrow F(A)$ tal que 
$\gamma^{-1}_{F(A)}\gamma_{F(A)}=1_{F(A)}$. Y como $F$ es pleno, existe $\psi_A\colon GF(A)\longrightarrow A$ tal que 
$F(\psi_A)=\gamma^{-1}_{F(A)}.$ Por lo tanto $F(\psi_A\psi'_A)=1_{F(A)}$; y como $F$ es fiel, entonces $\psi_A\psi'_A=1_A$ Análogamente
$\psi'_A\psi_A=1_{GF(A)}$ por lo que $\psi_A$ es isomorfismo.\\

Por último veamos que el siguiente diagrama
\begin{equation*}
\commutativesquare{A=GF(A),B=A,C=GF(A'),D=A',f=\psi_A,g=GF(\alpha),h=\alpha,k=\psi_{A'},}\quad\ldots (2)
\end{equation*}
conmuta en $A$.\\

En efecto, aplicando $F$ al diagrama anterior obtenemos que 
\begin{equation*}
\commutativesquare{A=FGF(A),B=F(A),C=FGF(A'),D=F(A'),f=F(\psi_A),g=FGF(\alpha),h=F(\alpha),k=F(\psi_{A'}),}
\end{equation*}
Como $F(\psi_A)=\gamma^{-1}_{F(A)}$ y $F(\psi_{A'})=\gamma^{-1}_{F(A')}$, reemplazando a $\beta$ del diagrama (1) por $F(\alpha)$, obtenemos
que el diagrama anterior conmuta. Peo $F$ es fiel, entonces el diagrama (2) conmuta y así $\psi\colon GF\longrightarrow 1_{\mathscr{A}}$ es una
equivalencia natural.
\end{proof}

\item Sea $\leq$ un preorden en una clase $X$. Pruebe que:
\begin{itemize}
\item[a)] La relación $\sim$ inducida por $\leq$, $\left(a\sim b\,\iff \, (a\leq b\,\,\,\text{y}\,\,\,b\leq a)\right)$ 
es una relación de equivalencia en X.\\
\item[b)] Considere la clase cociente $\faktor{X}{\sim}:=\{[x]\,|\, x\in X\}$, \\
donde $[x]:=\{y\in X\,|\, x\sim y\}.$ Pruebe que el preorden $\leq $ en $X$ 
induce un orden parcial en $\faktor{X}{\sim}$ dado por $[x]\leq [y]\, \iff \, x\leq y.$
\end{itemize}
\begin{proof}
\boxed{a)} Sea $\sim$ la relación descrita en la hipótesis.\\
Reflexividad:\\
Como $a=a$ entonces $a\leq a$, por lo que $a\sim a$.\\

Simetría:\\
Supongamos $a\sim b$ entonces $a\leq b$ y $b\leq a$, entonces $b\leq a$ y $a\leq b$ y por lo tanto $b\sim a$.\\

Transitividad:\\
Supongamos $a\sim b$ y $b\sim c$, entonces $a\leq b, b\leq a, b\leq c$ y $c\leq b$. En particular, como $\leq$ es preorden, $a\leq b\leq c$ y 
$c\leq b\leq a$, es decir, $a\leq c$ y $c\leq a$ por lo tanto $a\sim c$ y en consecuencia $\sim$ es de equivalencia.\\

\boxed{b)} Buena definición:\\
Sean $a\in [x]$ y $b\in [y]$ con $x,y\in X$, en particular $a\leq x, y\leq b$. Si $[x]\leq [y]$ entonces 
$a\leq x\leq y\leq b$ por lo tanto $a\leq b$. Y se tiene que la relación está
bien definida.\\

Reflexividad:\\
Como $a\leq a$ en $X$, pues $a\sim a$, entonces $[a]\leq [a]$.\\

Transitividad:\\
Supongamos $[x]\leq [y]$ y $[y]\leq [z]$. Entonces $x\leq y$ y $y\leq z$, pero $\leq$ es transitiva en $X$, entonces $x\leq z$ y así $[x]\leq [z]$.
\end{proof}
 
		\item Si los siguientes diagramas conmutativos en una categoría $\mathscr{C}$
		\begin{align*}
			\commutativesquare{A=P,B=A_2,C=A_1,D=A,f=\beta_2,g=\beta_1,h=\alpha_2,k=\alpha_1,}\quad  \commutativesquare{A=P',B=A_2,C=A_1,D=A,f=\beta_2',g=\beta_1',h=\alpha_2,k=\alpha_1,}
		\end{align*}
		son pull-backs, entonces $\exists$ $\catarrow{\gamma}{P}{P'}{}$ en $\mathscr{C}$ un isomorfismo tal que $\beta_i=\beta_i'\gamma$, $\forall\ i\in\lrsqp{1,2}$.
		\begin{proof}
			Por la propiedad universal del pull-back aplicada a P', se tiene el siguiente diagrama conmutativo
			\begin{equation*}
				\commutativesquare{up=t,A=P',B=A_2,C=A_1,D=A,f=\beta_2',g=\beta_1',h=\alpha_2,k=\alpha_1,l=\beta_2,m=\beta_1,n=\exists!\ \gamma,},
			\end{equation*}
		mientras que la propiedad universal del pull-back aplicada a P grantiza que el siguiente diagrama conmuta
		\begin{equation*}
			\commutativesquare{up=t,P=P',A=P,B=A_2,C=A_1,D=A,f=\beta_2,g=\beta_1,h=\alpha_2,k=\alpha_1,l=\beta_2',m=\beta_1',n=\exists!\ \gamma',}.
		\end{equation*}
		Así 
		\begin{equation*}
			\begin{split}
				\beta_1\lrprth{\gamma'\gamma}&=\lrprth{\beta_1'}\gamma\\
			&=\beta_1,\\
			\beta_2\lrprth{\gamma'\gamma}&=\lrprth{\beta_2'}\gamma\\
			&=\beta_2,
			\end{split}\tag{*}\label{condiciones}
		\end{equation*}
		de modo que los diagramas
		\begin{equation*}
			\commutativesquare{up=t,A=P,B=A_2,C=A_1,D=A,f=\beta_2,g=\beta_1,h=\alpha_2,k=\alpha_1,l=\beta_1,m=\beta_2,n=\gamma'\gamma,}\quad\commutativesquare{up=t,A=P,B=A_2,C=A_1,D=A,f=\beta_2,g=\beta_1,h=\alpha_2,k=\alpha_1,l=\beta_1,m=\beta_2,n=1_P,}
		\end{equation*}
		y por lo tanto, empleando la propiedad universal del pull-back para $P$, se tiene que $\gamma'\gamma=1_P$. En forma análoga, empleando ahora la propiedad universal del pull-back para $P'$, se verifica que $\gamma\gamma'=1_{P'}$, de modo que $\catarrow{\gamma}{P}{P'}{}$ es un isomorfismo en $\mathscr{C}$. Con lo anterior y (\ref{condiciones}) se tiene lo deseado.\\
		\end{proof} 
		\item Sea el siguiente diagrama conmutativo en una categoría $\mathscr{C}$
		\begin{equation*}
			\commutativesquare{A=P,B=A_2,C=A_1,D=A,f=\beta_2,g=\beta_1,h=\alpha_2,k=\alpha_1,}
		\end{equation*}
		un pull-back, entonces
		\begin{enumerate}
			\item si $\alpha_1$ es un monomorfismo, entonces $\beta_2$ también lo es;
			\item $\beta_2$ es un split-epi si y sólo si $\alpha_2$ se factoriza a través de $\alpha_1$.
		\end{enumerate}
		\begin{proof}
			\boxed{a)} Supongamos que $\alpha_1$ es un monomorfismo y que $\catarrow{f,g}{B}{P}{}$ son morfismos en $\mathscr{C}$ tales que $\beta_2 f=\beta_2 g$. Notemos primeramente que así
			\begin{align*}
				\alpha_1\lrprth{\beta_1 f}&=\lrprth{\alpha_2\beta 2}f=\alpha_2\lrprth{\beta_2 g}=\lrprth{\alpha_1\beta_1}g\\
				&=\alpha_1\lrprth{\beta_1 g}\\
				\implies \beta_1 f&=\beta_1 g, && \alpha_1\text{ es un mono}
			\end{align*}
			con lo cual se tiene el siguiente diagrama conmutativo
			\begin{equation*}
				\commutativesquare{up=t,P=B,A=P,B=A_2,C=A_1,D=A,f=\beta_2,g=\beta_1,h=\alpha_2,k=\alpha_1,l=\beta_2 f,m=\beta_1 f,n=f\text{, }g,},
			\end{equation*}
		y así la propiedad universal del pull-back garantiza que $f=g$, con lo cual $\beta_2$ es un mono.\\
		\boxed{b)\implies} Dado que $\beta_2$ es un split-epi $\exists\ \catarrow{\gamma}{A_2}{P}{}$ tal que $\beta_2\gamma=1_{A_2}$, así
		\begin{align*}
			\alpha_1\lrprth{\beta_1\gamma}&=\lrprth{\alpha_2\beta_2}\gamma=\alpha_2\lrprth{1_{A_2}}\\
			&=\alpha_2,\\
			\implies &\alpha_2\text{ se factoriza a través de }\alpha_1.
		\end{align*}
		\boxed{b)\impliedby} Se tiene que $\exists\ \catarrow{\alpha}{A_2}{A_1}{}$ en $\mathscr{C}$ tal que $\alpha_2=\alpha_1\alpha$, con lo cual a partir de la propiedad universal del pull-back se obtiene el siguiente diagrama conmutativo
		\begin{equation*}
			\commutativesquare{up=t, P=A_2, A=P,B=A_2,C=A_1,D=A,f=\beta_2,g=\beta_1,h=\alpha_2,k=\alpha_1, l=1_{A_2},m=\alpha,n=\exists !\ \gamma,},
		\end{equation*}
		del cual se deduce que en partícular $\beta_2\gamma=1_{A_2}$, y así se tiene lo deseado.\\
		\end{proof}

\item Supongamos que los dos cuadrados del siguiente diagrama en una categoría $\mathscr{C}$ conmutan\\
\centerline{\xymatrix{P\ar[d]^{\beta_1}\ar[r]^g & B'\ar[d]^{\theta_1} & Q\ar[l]_{\alpha_2} \ar[d]^{\theta_2}\\
A\ar[r]_f & B & I\ar[l]^{\gamma_2} }}
Pruebe que: si $\theta_1$ y $\gamma_2$ son monomorfismos en $\mathscr{C}$ y existe $\gamma_1\colon A\longrightarrow I$ 
tal que $f=\gamma_2\gamma_1$, entonces existe $\alpha_1\colon P\rightarrow Q$ tal que $\alpha_2\alpha_1=g$ y el siguiente diagrama es un pull-back
\begin{equation*}
\commutativesquare{,A=P,B=Q,C=A,D=I, f=\alpha_1,g=\beta_1,h=\theta_2,k=\gamma_1,}
\end{equation*}	
\begin{proof}
Como los diagramas del enunciado conmutan, entonces
 \[\theta_1\alpha_2=\alpha_2\theta_2\,,\,\,\, \theta_1 g=f\beta_1.\]
Ahora, como $f=\gamma_2\gamma_1$ se tiene que $\gamma_2\gamma_1\beta_1=f\beta_1$ asi el siguiente diagrama conmuta:\\
\begin{equation*}
\commutativesquare{,A=P,B=B',C=I,D=B, f=g,g=\gamma_1\beta_1,h=\theta_1,k=\gamma_2,}
\end{equation*}	

Pero $Q$ es pull-back de \begin{tikzcd}I\arrow{r}{\gamma_2} & B &B'\arrow{l}[swap]{\theta_1}\end{tikzcd}
por lo que existe un único $\alpha_1\colon P\longrightarrow Q$ tal que 
$g=\alpha_2\alpha_1$\,\,y\,\,$\gamma_1\beta_1=\theta_2\alpha_1.$ veamos ahora que el diagrama 

\begin{equation*}
\commutativesquare{,A=P,B=Q,C=A,D=I, f=\alpha_1,g=\beta_1,h=\theta_2,k=\gamma_1,}
\end{equation*}	
es un pull-back.\\

Supongamos que tenemos el siguiente diagrama 

\begin{equation*}
\commutativesquare{up=t,P=K,A=P,B=Q,C=A,D=I,f=\alpha_1,g=\beta_1,h=\theta_2,k=\gamma_1,l=k_1,m=k_2,n=,},
\end{equation*}

tal que $\gamma_1k_2=\theta_2k_1$. \\

Como  \begin{tikzcd}K\arrow{r}{k_2}&A\end{tikzcd}\,\,y\,\,\begin{tikzcd}K\arrow{r}{\alpha_2k_1}&B'\end{tikzcd}\,
cumple que $fk_2=\gamma_2\gamma_1k_2\,$\quad y\\$ \theta_1\alpha_2k_1=\gamma_2\theta_2k_1$.\\
Entonces como $\theta_2k_1=\gamma_1k_2$, tenemos que $fk_2=\theta_1\alpha_2k_1$.\\

Así el siguiente diagrama conmmuta:
\begin{equation*}
\commutativesquare{,A=K,B=B',C=A,D=B, f=\alpha_2k_1,g=k_2,h=\theta_1,k=f,}
\end{equation*}	
Pero $P$ es un pull-back, por lo tanto existe un único $\eta\colon K\longrightarrow P$ tal que $\beta_1\eta=k_2$ y $g\eta=\alpha_2k_1$, 
entonces $\alpha_2\alpha_1\eta=\alpha_2k_1$ y dado que $\gamma_2$ es mono y $Q$ pull-back,
se tiene que $\alpha_2$ es mono y $\alpha_1\eta=k_1$.\\
Además, si $\gamma\colon K\longrightarrow P$ es tal que $\beta_1\gamma=k_2$\,\,y\,\,$g\gamma=\alpha_2k_1$, entonces \\
$\beta_1\gamma=\beta_1\eta$ y puesto que $\theta_1$ es mono y $P$ pullback, entonces $\beta_1$ es mono y $\gamma=\eta$. 
Por lo tanto $\eta$ es único hasta isomorfismos y en consecuencia $P$ es pull-back de
\begin{tikzcd}Q\arrow{r}{\alpha_1} &I &A\arrow{l}[swap]{\gamma_1}\end{tikzcd}.
\end{proof}

\item Defina la noción dual del pull-back (i.e. push-out) y pruebe que el push-out, de existir, es único hasta isomorfismos.\\
\begin{proof}

Recordemos la definición del pull-back:\\
\textbf{Definición.} Sean $\alpha_1\colon A_1\longrightarrow A$\,\,y\,\, $\alpha_2\colon A_2\longrightarrow A$ morfismos en una categoría 
$\mathscr{C}$. Un pull-back para  \begin{tikzcd} A_1\arrow{r}{\alpha_1}&A&A_2\arrow{l}[swap]{\alpha_2}\end{tikzcd} es un diagrama conmutativo
en $\mathscr{C}$ 
 \begin{equation*}
\commutativesquare{,A=P,B=A_2,C=A_1,D=A, f=\beta_2,g=\beta_1,h=\alpha_2,k=\alpha_1,}
\end{equation*}	
Que satisface la siguiente propiedad universal:\\
$\forall \beta'_2\colon P'\longrightarrow A_2,\,\,\forall \beta_1'\colon P\longrightarrow A_1$\,\, tal que \,\, $\alpha_2\beta_1'=\alpha_1\beta_1'$, 
se tiene que $\exists !\ \catarrow{\gamma}{P'}{P}{}$ tal que $\beta_1'=\beta_1\gamma$ y $\beta_2'=\beta_2\gamma$. Diagramaticamente se 
vé como sigue:

\begin{equation*}
\commutativesquare{up=t,P=P',A=P,B=A_2,C=A_1,D=A,f=\beta_21,g=\beta_1,h=\alpha_2,k=\alpha_1,l=\beta_2',m=\beta_1',n=\exists !\gamma,},
\end{equation*}

Así el concepto de pull-back en la categoría opuesta se puede abreviar de la siguiente manera:

\begin{itemize}
\item[a)] $\exists\ \catarrow{\beta_1^{op}}{A_1}{P}{},\catarrow{\beta_2^{op}}{A_2}{P}{}$ tales que
$\alpha_1^{op}\beta_1^{op}=\alpha_2^{op}\beta_2^{op}$.
\item[b)] $\forall\ P'\in\mathscr{C^{op}}$ y $\forall\ \catarrow{\beta_1'^{op}}{A_1}{P'}{},\catarrow{\beta_2'^{op}}{A_2}{P'}{}$ tales que \\
$\alpha_1^{op}\beta_1'^{op}=\alpha_2^{op}\beta_2'^{op}$, $\exists !\ \catarrow{\gamma^{op}}{P}{P'}{}$ tal que
 $\beta_1'^{op}=\beta_1^{op}\gamma^{op}$ y \\$\beta_2'^{op}=\beta_2^{op}\gamma^{op}$.
\end{itemize}

Con esto en mente, entonces podemos dar la siguiente definición.\\

\textbf{Definición.} Sean $\alpha_1\colon A\longrightarrow A_1$\,\,y\,\,$\alpha_2\colon A\longrightarrow A_2$ morfismos en una categoría 
$\mathscr{C}$. Un push-out para \begin{tikzcd} A_1&A\arrow{l}[swap]{\alpha_1}\arrow{r}{\alpha_2}&A_2\end{tikzcd} es un diagrama 
 \begin{equation*}
\commutativesquare{pp=l,A=P,B=A_1,C=A_2,D=A, f=\beta_1,g=\beta_2,h=\alpha_1,k=\alpha_2,}
\end{equation*}	

 conmutativo en $\mathscr{C}$ que satisface la siguiente propiedad universal:\\

$\forall \beta'_2\colon A_2\longrightarrow P',\,\,\forall \beta_1'\colon A_1\longrightarrow P'$\,\, tal que \,\, $\beta'_1\alpha_1=\beta'_2\alpha_2$, se tiene
que $\exists !\ \catarrow{\gamma}{P}{P'}{}$ tal que $\beta_1'=\gamma\beta_1$ y $\beta_2'=\gamma\beta_2$. Diagramaticamente se 
vé como sigue:

\begin{equation*}
\commutativesquare{up=t,pp=l,P=P',A=P,B=A_1,C=A_2,D=A, f=\beta_1,g=\beta_2,h=\alpha_1,k=\alpha_2,l=\beta'_1,m=\beta'_2,n=\exists !\ \gamma,}
\end{equation*}	

Unicidad.\\
Supongamos que $Q\in \mathscr{C}$, $\eta_1\colon A_1\longrightarrow Q$\,\,y\,\,$\eta_2\colon A_2\longrightarrow Q$ son otro push-out de 
  \begin{tikzcd} A_1&A\arrow{l}[swap]{\alpha_1}\arrow{r}{\alpha_2}&A_2\end{tikzcd}, entonces se tiene el siguiente diagrama conmutativo:
 \begin{equation*}
\commutativesquare{pp=l,A=Q,B=A_1,C=A_2,D=A, f=\eta_1,g=\eta_2,h=\alpha_1,k=\alpha_2,}
\end{equation*}	
Como $P$ es push-out existe un único $\gamma\colon P\longrightarrow Q$ tal que $\gamma\beta_2=\eta_2$\,\,y\\$\gamma\beta_1=\eta_1$,
y como $Q$ es push-out existe un único $\bar{\gamma}\colon Q\longrightarrow P$ tal que $\bar{\gamma}\eta_2=\beta_2$\,\,y
\,\,$\bar{\gamma}\eta_1=\beta_1$.\\
Con estos resultados se obtiene que $\bar{\gamma}\gamma\colon P\longrightarrow P$\,\,,
\[\bar{\gamma}\gamma\beta_2=\bar{\gamma}\eta_2=\beta_2\,\quad \text{y}\quad \bar{\gamma}\gamma\beta_1=\bar{\gamma}\eta_1=\beta_1.
\]

Pero el funtor identidad también cumple dichas igualdades, así, como $P$ es push-out, $\bar{\gamma}\gamma=1_P$ por unicidad. Análogamente
$\gamma\bar{\gamma}=1_Q$ por lo tanto $P\cong Q$.
\end{proof}

		\item Enunciaremos y probaremos la proposición dual al Ej. 8. Notemos primeramente que\\
		\textbf{Pull-back:}\begin{enumerate}[label=PB\Roman*)]
			\item $\exists\ \catarrow{\beta_1}{P}{A_1}{},\catarrow{\beta_2}{P}{A_2}{}$ tales que $\alpha_1\beta_1=\alpha_2\beta_2$.
			\item $\forall\ P'\in\mathscr{C}$ y $\forall\ \catarrow{\beta_1'}{P'}{A_1}{},\catarrow{\beta_2'}{P'}{A_2}{}$ tales que $\alpha_1\beta_1'=\alpha_2\beta_2'$, $\exists !\ \catarrow{\gamma}{P'}{P}{}$ tal que $\beta_1'=\beta_1\gamma$ y $\beta_2'=\beta_2\gamma$.
		\end{enumerate}
		\textbf{Pull-back\textsuperscript{op}:}\begin{enumerate}[label=PB\textsuperscript{op}\Roman*)]
		\item $\exists\ \catarrow{\beta_1^{op}}{A_1}{P}{},\catarrow{\beta_2^{op}}{A_2}{P}{}$ tales que $\alpha_1^{op}\beta_1^{op}=\alpha_2^{op}\beta_2^{op}$.
		\item $\forall\ P'\in\mathscr{C^{op}}$ y $\forall\ \catarrow{\beta_1'^{op}}{A_1}{P'}{},\catarrow{\beta_2'^{op}}{A_2}{P'}{}$ tales que $\alpha_1^{op}\beta_1'^{op}=\alpha_2^{op}\beta_2'^{op}$, $\exists !\ \catarrow{\gamma^{op}}{P}{P'}{}$ tal que $\beta_1'^{op}=\beta_1^{op}\gamma^{op}$ y $\beta_2'^{op}=\beta_2^{op}\gamma^{op}$.
	\end{enumerate}
	\textbf{Pull-back\textsuperscript{*}:}\begin{enumerate}[label=PB\textsuperscript{*}\Roman*)]
	\item $\exists\ \catarrow{\beta_1}{A_1}{P}{},\catarrow{\beta_2}{A_2}{P}{}$ tales que $\beta_1\alpha_1=\beta_2\alpha_2$.
	\item $\forall\ P'\in\mathscr{C}$ y $\forall\ \catarrow{\beta_1'}{A_1}{P'}{},\catarrow{\beta_2'}{A_2}{P'}{}$ tales que $\beta_1'\alpha_1=\beta_2'\alpha_2$, $\exists !\ \catarrow{\gamma}{P}{P'}{}$ tal que $\beta_1'=\gamma\beta_1$ y $\beta_2'=\gamma\beta_2$.
	\end{enumerate}
	Esto último es la definición de que un objeto $P$ sea un push-out de $\catarrow{\alpha_1}{A}{A_1}{}$ y $\catarrow{\alpha_2}{A}{A_2}{}$. Por lo anterior, y dado que las propiedades duales de mono y split-epi son respectivamente epi y split-mono, la proposición dual del Ej. 8 es:\\
	Sea el siguiente diagrama conmutativo en una categoria $\mathscr{C}$
	\begin{equation*}
		\commutativesquare{pp=l,A=P,B=A_2,C=A_1,D=A, f=\beta_2,g=\beta_1,h=\alpha_2,k=\alpha_1,}
	\end{equation*}	
	un push-out, entonces
		\begin{enumerate}
			\item si $\alpha_1$ es un epimorfismo, entonces $\beta_2$ también lo es;
			\item $\beta_2$ es un split-mono  si y sólo si $\exists\ \catarrow{\delta}{A_1}{A_2}{}$ en $\mathscr{C}$ tal que $\alpha_2=\delta\alpha_1$.
		\end{enumerate}
	\begin{proof}
		\boxed{a)} Supongamos que $\catarrow{f}{P}{Q}{}$ y $\catarrow{g}{P}{Q}{}$ en $\mathscr{C}$ son tales que $f\beta_2=g\beta_2$. Notemos que
		\begin{align*}
			\lrprth{f\beta_1}\alpha_1&=f\lrprth{\beta_2\alpha 2}=\lrprth{g\beta_2 }\alpha_2=g\lrprth{\beta_1\alpha_1}\\
			&=\lrprth{g\beta_1}\alpha_1\\
			\implies f\beta_1 &=g\beta_1 , && \alpha_1\text{ es un epi}
		\end{align*}
		con lo cual se tiene el siguiente diagrama conmutativo
		\begin{equation*}
			\commutativesquare{up=t,pp=l,P=B,A=P,B=A_2,C=A_1,D=A,f=\beta_2,g=\beta_1,h=\alpha_2,k=\alpha_1,l=\beta_2 f,m=\beta_1 f,n=f\text{, }g,},
		\end{equation*}
		y así la propiedad universal del push-out garantiza que $f=g$, con lo cual $\beta_2$ es un epi.\\
		\boxed{b)\implies} Por ser $\beta_2$ un split-mono $\exists\ \catarrow{\gamma}{P}{A_2}{}$ en $\mathscr{C}$ tal que $\gamma\beta_2=1_{A_2}$, de modo que si $\delta:=\gamma\beta_1$, entonces
		\begin{align*}
			\delta\alpha_1&=\gamma\lrprth{\beta_1\alpha_1}=\lrprth{\gamma\beta_2}\alpha_2=1_{A_2}\alpha_2\\
			&=\alpha_2. 
		\end{align*}
		\boxed{b)\impliedby} Bajo estas condiciones de la propiedad universal del push-out se obtiene el siguiente diagrama conmutativo
		\begin{equation*}
			\commutativesquare{up=t,pp=l,P=A_2,A=P,B=A_2,C=A_1,D=A,f=\beta_2,g=\beta_1,h=\alpha_2,k=\alpha_1,l=1_{A_2},m=\delta,n=\exists !\ \gamma,},
		\end{equation*}
		del cual se sigue en partícular que $\gamma\beta_2=1_{A_2}$.
	\end{proof}
		\item Si $R$ es un anillo entonces la categoría $Mod\lrprth{R}$ tiene pull-backs.
		\begin{proof}
			Sean $\catarrow{\alpha_1}{A_1}{A}{}$ y $\catarrow{\alpha_2}{A_2}{A}{}$ morfismos de $R$-módulos y
			\begin{align*}
				A_1\times_{A} A_2:=\descset{\lrprth{x,y}}{A_1\times A_2}{\alpha_1\lrprth{x}=\alpha_2\lrprth{y}}
			\end{align*}
			\renewcommand{\copyandpaste}{A_1\times_{A} A_2}
			Notemos que $\copyandpaste\neq\varnothing$, pues si $0_1$, $0_2$ y $0$ son los neutros aditivos de $A_1$, $A_2$ y $A$, respectivamente, entonces $\alpha_1\lrprth{0_1}=0=\alpha_2\lrprth{0_2}$, con lo cual $\lrprth{0_1,0_2}\in\copyandpaste$. Más aún, $\copyandpaste\leq A_1\times A_2$, con $A_1\times A_2$ dotado de la estructura usual de $R$-módulo, pues si $\lrprth{a,b},\lrprth{c,d}\in\copyandpaste$ y $r\in R$, entonces
			\begin{align*}
				\alpha_1\lrprth{ra-b}&=r\alpha_1\lrprth{a}-\alpha_1\lrprth{b}\\
				&=r\alpha_2\lrprth{c}-\alpha_2\lrprth{d}\\
				&=\alpha_2\lrprth{rc-d},\\
				&\implies r\lrprth{a,b}-\lrprth{c,d}\in\copyandpaste.
			\end{align*}
			Con lo cual $\copyandpaste\in Mod\lrprth{R}$. Así, si $\pi_1$ y $\pi_2$ son las proyecciones canónicas de $\copyandpaste$ sobre $A_1$ y $A_2$, respectivamente, y $\lrprth{x,y}\in\copyandpaste$, entonces $\pi_1,\ \pi_2$ son morfismos de $R$-módulos y 
			\begin{align*}
				\alpha_1\pi_1\lrprth{x,y}&=\alpha_1\lrprth{x}=\alpha_2\lrprth{y}=\alpha_2\lrprth{\pi_2\lrprth{x,y}}\\
				&=\alpha_2\pi_2\lrprth{x,y},\\
				\implies \alpha_1\pi_1&=\alpha_2\pi_2.
			\end{align*}
			Es decir, se tiene que el siguiente diagrama conmuta
			\begin{align*}
				\commutativesquare{A=\copyandpaste,B=A_2,C=A_1,D=A,f=\pi_2,g=\pi_1,h=\alpha_2,k=\alpha_1,}.
			\end{align*}
			Ahora, si $P\in Mod\lrprth{R}$ y $\catarrow{\beta_1}{P}{A_1}{},\catarrow{\beta_2}{P}{A_2}{}$ son morfismos de $R$-módulos tales que $\alpha_1\beta_1=\alpha_2\beta_2$, entonces sea
			\begin{align*}
				\descapp{\gamma}{P}{\copyandpaste}{p}{\lrprth{\beta_1\lrprth{p},\beta_2\lrprth{p}}}{.}
			\end{align*}
			Notemos que $\gamma$ es un morfismo de $R$-módulos, puesto que $\beta_1$ y $\beta_2$ lo son, y que si $p\in P$ entonces
			\begin{align*}
				\pi_1\gamma\lrprth{p}&=\pi_1\lrprth{\beta_1\lrprth{p},\beta_2\lrprth{p}}=\beta_1\lrprth{p}\\
				\implies \pi_1\gamma&=\beta_1.
			\end{align*}
			Análogamente se verifica que $\pi_2\gamma=\beta_2$, con lo cual el siguiente diagrama conmuta
			\begin{equation*}
				\commutativesquare{up=t,A=\copyandpaste,B=A_2,C=A_1,D=A,f=\pi_2,g=\pi_1,h=\alpha_2,k=\alpha_1,l=\beta_2,m=\beta_1,n=\gamma,}
			\end{equation*}
			Finalmente, si $\catarrow{\gamma'}{P}{\copyandpaste}{}$ es un morfismo de $R$-módulos tal que $\pi_1\gamma'=\beta_1$ y $\pi_2\gamma'=\beta_2$ y $p\in P$, entonces
			\begin{align*}
				\pi_1\gamma'\lrprth{p}&=\beta_1\lrprth{p},\\
				\pi_2\gamma'\lrprth{p}&=\beta_2\lrprth{p},
			\end{align*}
			con lo cual $\gamma'\lrprth{p}=\lrprth{\pi_1\lrprth{\gamma'\lrprth{p}},\pi_2\lrprth{\gamma'\lrprth{p}}}=\lrprth{\beta_1\lrprth{p},\beta_2\lrprth{p}}=\gamma\lrprth{p}$ y por lo tanto $\gamma'=\gamma$.\\
		\end{proof}
\item Para un anillo $R$ pruebe que $Mod(R)$ tiene Push-ots.
\begin{proof}
Sea \begin{tikzcd} A_1&A\arrow{l}[swap]{\alpha_1}\arrow{r}{\alpha_2}&A_2\end{tikzcd} en $Mod(R)$. Consideremos \\
$N:=\{(\alpha_2(a),-\alpha_1(a))\in A_2\times A_1\,|\,a\in A\}$. Observemos que $N\leq A_2\times A_1$, pues $\forall r\in R$ y $\forall a\in A$
\begin{gather*}
r(\alpha_2(a),-\alpha_1(a))+(\alpha_2(b),-\alpha_1(b)\\
=(r\alpha_2(a),-r\alpha_1(a))+(\alpha_2(b),-\alpha_1(b))\\
=(\alpha_2(ra),-\alpha_1(ra))+(\alpha_2(b),-\alpha_1(b))\\
=(\alpha_2(ra)+\alpha_2(b),-\alpha_1(ra)+\alpha_1(b))\\
=(\alpha_2(ra+b),-\alpha_1(ra+b))\in N.
\end{gather*}

Sea $\displaystyle A_2\times^A A_1:=\faktor{A_2\times A_1}{N}$. Consideremos los morfismos\\ $\mu_i\colon A_i\longrightarrow A_2\times^AA_1$, 
dados por las composiciones\\ \begin{tikzcd} A_i\arrow{r}{inc_i}&A_2\times A_1\arrow{r}{\pi}&A_2\times^AA_1\end{tikzcd} donde 
\begin{align*}
inc_1(a_1)=(0,a_1)\\
inc_2(a_2)=(a_2,0)\\
\pi(x)=x+N.
\end{align*}
Entonces 
\begin{align*}
\mu_1\alpha_1(a)=\mu_1(\alpha_1(a))\\
=\pi[(0,(\alpha_1(a)))]
=(0,(\alpha_1(a)))+N\\
=(0,(\alpha_1(a)))+(\alpha_2(a),-\alpha_1(a))+N\\
=(\alpha_2(a),0)+N
=\pi(\alpha_2(a),0)\\
=\mu_2(\alpha_2(a))
=\mu_2\alpha_2(a)
\end{align*}

Por lo tanto el siguiente diagrama conmuta:
\begin{align*}
\commutativesquare{A=A,B=A_1,C=A_2,D=A_2\times^AA_1\,.,f=\alpha_1,g=\alpha_2,h=\mu_1,k=\mu_2,}
\end{align*}
Ahora, sea $P\in Mod(R)$ tal que el siguiente diagrama conmuta:
\begin{align*}
\commutativesquare{A=A,B=A_1,C=A_2,D=P\,.,f=\alpha_1,g=\alpha_2,h=\beta_1,k=\beta_2,}
\end{align*}
Afirmamos que existe un único $\gamma\colon A_2\times^AA_1\longrightarrow P$ tal que $\gamma\mu_1=\beta_1$\,\, y \,\, 
$\gamma\mu_2=\beta_2$. Sea $\gamma\colon A_2\times^AA_1\longrightarrow P$ dada por $\gamma(a,b)=\beta_2(a)\beta_1(a)$,
 entonces $\gamma\mu_1(a_1)=\gamma\pi(0,a_1)=\gamma[(0,a_1)+N]=0+\beta_1(a_1)$. Análogamente
$\beta_2=\gamma\mu_2(a_2)\quad \forall a_2\in A_2$.\\

Ahora, si $(a,b),(c,d)\in A_2\times^AA_1$, se tiene que 
\begin{align*}
\gamma[r(a,b)]-\gamma(c,d)=\gamma(ra,rb)-\gamma(c,d)\\
=\beta_1(ra)+\beta_2(rb)-\beta_1(c)-\beta_2(d)\\
=\beta_1(ra-c)+\beta_2(rb-d)\\
=\gamma[(ra-c,rb-d)]=\gamma[r(a,b)-(c,d)].
\end{align*}
Mas aún, si $(a,b)-(c,d)\in N$ entonces $(a-c,b-d)=(\alpha_2(x),-\alpha_1(x))$ para algun $x\in A$. Así 
\begin{align*}
\gamma(a,b)-\gamma(c,d)=\gamma[(a,b)-(c,d)]\\
=\gamma(\alpha_2(x),-\alpha_1(x))\\
=\beta_2(\alpha_2(x))+\beta_1(-\alpha_1(x))\\
=\beta_1\alpha_1(x)-\beta_1\alpha_1(x)=0
\end{align*}
Por lo tanto $\gamma$ es un morfismo de $A_2\times^AA_1$ en $P$ y está bien definido.\\

Por último, si $\eta:A_2\times^AA_1\longrightarrow P$ es otro morfismo tal que \\$\eta\mu_1=\beta_1$\,\,y\,\,$\eta\mu_2=\beta_2$, entonces
para cada $(a,b)\in A_2\times^AA_1$
\begin{align*}
\eta(a,b)=\eta[(a,0)+(0,b)]\\
=\eta[\mu_2(a)+\mu_1(b)]\\
=\eta(\mu_2(a))+\eta(\mu_1(b))\\
=\beta_2(a)+\beta_1(b)\\
=\gamma(a,b)
\end{align*}
Por lo que $\gamma=\eta$.
\end{proof}

\item Las categorías $Sets$ y $Mod(R)$, con $R$ un anillo, tienen intersecciones
\begin{proof}
Sea $\{\alpha_i\colon A_i\longrightarrow A\}_{i\in I}$ una familia de morfismos en $Mod(R)$ y sea $\displaystyle\bigcap_{i\in I}Im(\alpha_i)$ la 
intersección usual de módulos. Sea $\nu\colon \displaystyle\bigcap_{i\in I}Im(\alpha_i)\longrightarrow A$ la inclusión canónica (de conjuntos) entonces
se tiene lo siguiente:\\

Dado $a\in\displaystyle\bigcap_{i\in I}Im(\alpha_i)$ se tiene que $a\in Im(\alpha_i)$ para cada $i\in I$ es decir: $\exists  a_i\in A_i$ tal que 
$\alpha_i(a_i)=a$ para cada $i\in I$. Así $a=\nu(a)=\alpha_i\nu_i(a)$ para cada $i\in I$ donde $\nu_i\colon \displaystyle\bigcap_{i\in I}Im(\alpha_i)
\longrightarrow A_i$ está dada por $\nu_i(a)=a_i$ por lo que el siguiente diagrama conmuta

\centerline{\xymatrix{\displaystyle\bigcap_{i\in I}Im(\alpha_i)\ar[d]_{\nu_i}\ar[rd]^\nu\\
A_i\ar[r]_{\alpha_i}&A}}

Sea $\theta\colon B\longrightarrow A$ en $Mod(R)$. Si $\theta$ se factoriza a travéz de $\alpha_i\colon A_i\longrightarrow A$ entonces existe
$\theta_i\colon B\longrightarrow A_i$ tal que $\theta=\alpha_i\theta_i$. Así para toda $b\in B$
\begin{align*}
\theta(b)=\alpha_i\theta_i(b)=\alpha_i(\theta_i(b)),\\
\theta(b)\in Im(\alpha_i)\quad \forall i\in I,\\
\theta(b)\in \displaystyle\bigcap_{i\in I}Im(\alpha_i)\subset A,\\
\theta(b)=\nu(a)
\end{align*}
con $a=\theta(b)\in \displaystyle\bigcap_{i\in I}Im(\alpha_i)$. Así si $\eta\colon B\longrightarrow A$  se define como $\eta(b)=\theta(b)$, entonces 
$\theta=\nu\eta$ y el siguiente diagrama conmuta

\centerline{\xymatrix{B\ar[d]_{\eta}\ar[rd]^\theta\\
A'\ar[r]_{\mu}&A}}

Solo se usaron argumentos conjuntistas (no exclusivos de teoría de módulos) para esta prueba salvo el que intersección de módulos es módulo
 (intersección de conjuntos es conjunto) y que la inclusión conjuntista y la composición de morfismos es morfismo (composición de funciones es función),
por lo que este mismo resultado se demuestra de manera análoga para la categoría $Sets$.
\end{proof}


		\item Si $\mathscr{C}$ es una categoría con pull-backs, entonces $\mathscr{C}$ tiene intersecciones finitas.
		\begin{proof}
			Sea $A\in\mathscr{A}$ y $\lrbrack{\catarrow{\mu_i}{A_i}{A}{}}_{i\in I}$ una familia de subobjetos de $A$. Si $I=\varnothing$, el resultado es inmediato pues en tal caso $\catarrow{1_A}{A}{A}{}$ es una intersección para la familia. Así pues, podemos suponer sin pérdida de generalidad que $I=\lrsqp{1,n}\subseteq\mathbb{N}$, con $n\geq 1$ y proceder por inducción sobre $n$.\\
			Si $n=1$, entonces se tiene que $\catarrow{\mu_1}{A_1}{A}{}$ es una intersección para la familia $\lrbrack{\mu_1}$, puesto que 
			$\mu_1=\mu_1 1_{A_1}$ y $\mu_1$ satisface en forma inmediata la propiedad universal de la intersección.\\
			Si $n=2$, el resultado se sigue de la Proposición 1.3.2 en conjunto a que $\mathscr{C}$ es una categoría con pull-backs.\\
			Así pues supongamos por Hipótesis de Inducción que la proposición es válidad para $n=k$, $k\geq 2$, y verifiquémosla para $k+1$.
			Si $\fntfam{\mu}{i}{k+1}$ es una familia de $k+1$ subobjetos de $A$ entonces por la Hipótesis de Inducción la familia $\fntfam{\mu}{i}{k}$  admite intersecciones, digamos $\catarrow{\nu}{\bigcap\limits_{i=1}^{k}A_i}{A}{}$. Recordemos que $\nu$ es un monomorfismo, y por lo tanto, por el caso $n=2$, se tiene que la familia de subobjetos $\lrbrack{\nu,\mu_{k+1}}$ admite intersecciones, digamos $\catarrow{\mu}{\lrprth{\bigcap\limits_{i=1}^{k}A_i}\cap A_{k+1}}{A}{}$. Afirmamos que $\mu$ es una intersección para $\fntfam{\mu}{i}{k+1}$. En efecto, del hecho de que $\mu$ sea una intersección para $\lrbrack{\nu,\mu_{k+1}}$ se sigue que $\mu$ se factoriza a través de $\mu_{k+1}$ y a través de $\nu$. Por su parte $\nu$ se factoriza a través de $\mu_i$, $\forall\ i\in\lrsqp{1,k}$, y en consecuencia $\mu$ también lo hace; de modo que $\mu\leq\mu_i\ \forall\ i\in\lrsqp{1,k+1}$. Finalmente, si $\catarrow{\theta}{B}{A}{}$ se factoriza a través de $\mu_i$ $\forall\ i\in\lrsqp{1,k+1}$, en partícular se factoriza a través de $\mu_i$ $\forall\ i\in\lrsqp{1,k}$, y así por la propiedad universal de la intersección se sigue que $\theta$ se factoriza a través de $\nu$. Así $\theta\leq \nu$ y $\theta\leq \mu_{k+1}$, con lo cual, por la propiedad universal de la intersección, $\nu$ se factoriza a través de $\mu$. Con lo cual se ha verificado la afirmación y así se concluye la inducción.\\
		\end{proof}
	\end{enumerate}	
\end{document}