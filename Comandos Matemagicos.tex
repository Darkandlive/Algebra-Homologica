
\def\subnormeq{\mathrel{\scalerel*{\trianglelefteq}{A}}}
\newcommand{\Z}{\mathbb{Z}}
\newcommand{\La}{\mathscr{L}}
\newcommand{\crdnlty}[1]{
	\left|#1\right|
}
\newcommand{\lrprth}[1]{
	\left(#1\right)
}
\newcommand{\lrbrack}[1]{
	\left\{#1\right\}
}
\newcommand{\lrsqp}[1]{
	\left[#1\right]
}
\newcommand{\descset}[3]{
	\left\{#1\in#2\ \vline\ #3\right\}
}
\newcommand{\descapp}[6]{
	#1: #2 &\rightarrow #3\\
	#4 &\mapsto #5#6 
}
\newcommand{\arbtfam}[3]{
	{\left\{{#1}_{#2}\right\}}_{#2\in #3}
}
\newcommand{\arbtfmnsub}[3]{
	{\left\{{#1}\right\}}_{#2\in #3}
}
\newcommand{\fntfmnsub}[3]{
	{\left\{{#1}\right\}}_{#2=1}^{#3}
}
\newcommand{\fntfam}[3]{
	{\left\{{#1}_{#2}\right\}}_{#2=1}^{#3}
}
\newcommand{\fntfamsup}[4]{
	\lrbrack{{#1}^{#2}}_{#3=1}^{#4}
}
\newcommand{\arbtuple}[3]{
	{\left({#1}_{#2}\right)}_{#2\in #3}
}
\newcommand{\fntuple}[3]{
	{\left({#1}_{#2}\right)}_{#2=1}^{#3}
}
\newcommand{\gengroup}[1]{
	\left< #1\right>
}
\newcommand{\stblzer}[2]{
	St_{#1}\lrprth{#2}
}
\newcommand{\cmmttr}[1]{
	\left[#1,#1\right]
}
\newcommand{\grpindx}[2]{
	\left[#1:#2\right]
}
\newcommand{\syl}[2]{
	Syl_{#1}\lrprth{#2}
}
\newcommand{\grtcd}[2]{
	mcd\lrprth{#1,#2}
}
\newcommand{\lsttcm}[2]{
	mcm\lrprth{#1,#2}
}
\newcommand{\amntpSyl}[2]{
	\mu_{#1}\lrprth{#2}
}
\newcommand{\gen}[1]{
	gen\lrprth{#1}
}
\newcommand{\ringcenter}[1]{
	C\lrprth{#1}
}
\newcommand{\zend}[2]{
	End_{\mathbb{Z}}^{#2}\lrprth{#1}
}
\newcommand{\cend}[2]{
	End_{#1}\lrprth{#2}
}
\newcommand{\catmatrix}[4]{Mat_{#1\times #2}\lrprth{#3,#4}}
\newcommand{\genmod}[2]{
	\left< #1\right>_{#2}
}
\newcommand{\genlin}[1]{
	\mathscr{L}\lrprth{#1}
}
\newcommand{\opst}[1]{
	{#1}^{op}
}
\newcommand{\ringmod}[3]{
	\if#3l
	{}_{#1}#2
	\else
	\if#3r
	#2_{#1}
	\fi
	\fi
}
\newcommand{\ringbimod}[4]{
	\if#4l
	{}_{#1-#2}#3
	\else
	\if#4r
	#3_{#1-#2}
	\else 
	\ifstrequal{#4}{lr}{
		{}_{#1}#3_{#2}
	}
	\fi
	\fi
}
\newcommand{\ringmodhom}[3]{
	Hom_{#1}\lrprth{#2,#3}
}
\newcommand{\catarrow}[4]{
	\if #4e
			#1:#2\twoheadrightarrow #3
	\else \if #4m
			#1:#2\hookrightarrow #3
	\else \if #4i
		    #1:#2\tilde{\to} #3
			\else #1:#2\to #3
	\fi
	\fi
	\fi
}
\newcommand{\nattrans}[4]{
	Nat_{\lrsqp{#1,#2}}\lrprth{#3,#4}
}
\ExplSyntaxOn

\NewDocumentCommand{\functor}{O{}m}
{
	\group_begin:
	\keys_set:nn {nicolas/functor}{#2}
	\nicolas_functor:n {#1}
	\group_end:
}

\keys_define:nn {nicolas/functor}
{
	name     .tl_set:N = \l_nicolas_functor_name_tl,
	dom   .tl_set:N = \l_nicolas_functor_dom_tl,
	codom .tl_set:N = \l_nicolas_functor_codom_tl,
	arrow      .tl_set:N = \l_nicolas_functor_arrow_tl,
	source   .tl_set:N = \l_nicolas_functor_source_tl,
	target   .tl_set:N = \l_nicolas_functor_target_tl,
	Farrow      .tl_set:N = \l_nicolas_functor_Farrow_tl,
	Fsource   .tl_set:N = \l_nicolas_functor_Fsource_tl,
	Ftarget   .tl_set:N = \l_nicolas_functor_Ftarget_tl,	
	delimiter .tl_set:N= \_nicolas_functor_delimiter_tl,	
}

\dim_new:N \g_nicolas_functor_space_dim

\cs_new:Nn \nicolas_functor:n
{
	\begin{tikzcd}[ampersand~replacement=\&,#1]
		\dim_gset:Nn \g_nicolas_functor_space_dim {\pgfmatrixrowsep}		
		\l_nicolas_functor_dom_tl
		\arrow[r,"\l_nicolas_functor_name_tl"] \&
		\l_nicolas_functor_codom_tl
		\\[\dim_eval:n {1ex-\g_nicolas_functor_space_dim}]
		\l_nicolas_functor_source_tl
		\xrightarrow{\l_nicolas_functor_arrow_tl}
		\l_nicolas_functor_target_tl
		\arrow[r,mapsto] \&
		\l_nicolas_functor_Fsource_tl
		\xrightarrow{\l_nicolas_functor_Farrow_tl}
		\l_nicolas_functor_Ftarget_tl
		\_nicolas_functor_delimiter_tl
	\end{tikzcd}
}

\ExplSyntaxOff


\newcommand{\limseq}[3]{
	\if#3u
	\lim\limits_{#2\to\infty}#1	
		\else
			\if#3s
			#1\to
			\else
				\if#3w
				#1\rightharpoonup
				\else
					\if#3e
					#1\overset{*}{\to}
					\fi
				\fi
			\fi
	\fi
}

\newcommand{\norm}[1]{
	\crdnlty{\crdnlty{#1}}
}

\newcommand{\inter}[1]{
	int\lrprth{#1}
}
\newcommand{\cerrad}[1]{
	cl\lrprth{#1}
}

\newcommand{\restrict}[2]{
	\left.#1\right|_{#2}
}

\newcommand{\realprj}[1]{
	\mathbb{R}P^{#1}
}

\newcommand{\fungroup}[1]{
	\pi_{1}\lrprth{#1}	
}

\ExplSyntaxOn

\NewDocumentCommand{\shortseq}{O{}m}
{
	\group_begin:
	\keys_set:nn {nicolas/shortseq}{#2}
	\nicolas_shortseq:n {#1}
	\group_end:
}

\keys_define:nn {nicolas/shortseq}
{
	A     .tl_set:N = \l_nicolas_shortseq_A_tl,
	B   .tl_set:N = \l_nicolas_shortseq_B_tl,
	C .tl_set:N = \l_nicolas_shortseq_C_tl,
	f      .tl_set:N = \l_nicolas_shortseq_f_tl,
	g   .tl_set:N = \l_nicolas_shortseq_g_tl,	
	lcr   .tl_set:N = \l_nicolas_shortseq_lcr_tl,	
	
	A		.initial:n =A,
	B		.initial:n =B,
	C		.initial:n =C,
	f    .initial:n =,
	g   	.initial:n=,
	lcr   	.initial:n=lr,
	
}

\cs_new:Nn \nicolas_shortseq:n
{
	\begin{tikzcd}[ampersand~replacement=\&,#1]
		\IfSubStr{\l_nicolas_shortseq_lcr_tl}{l}{0 \arrow{r} \&}{}
		\l_nicolas_shortseq_A_tl
		\arrow{r}{\l_nicolas_shortseq_f_tl} \&
		\l_nicolas_shortseq_B_tl
		\arrow[r, "\l_nicolas_shortseq_g_tl"] \&
		\l_nicolas_shortseq_C_tl
		\IfSubStr{\l_nicolas_shortseq_lcr_tl}{r}{ \arrow{r} \& 0}{}
	\end{tikzcd}
}

\ExplSyntaxOff

\newcommand{\pushpull}{texto}
\newcommand{\testfull}{texto}
\newcommand{\testdiag}{texto}
\newcommand{\testcodiag}{texto}
\ExplSyntaxOn

\NewDocumentCommand{\commutativesquare}{O{}m}
{
	\group_begin:
	\keys_set:nn {nicolas/commutativesquare}{#2}
	\nicolas_commutativesquare:n {#1}
	\group_end:
}

\keys_define:nn {nicolas/commutativesquare}
{	
	A     .tl_set:N = \l_nicolas_commutativesquare_A_tl,
	B   .tl_set:N = \l_nicolas_commutativesquare_B_tl,
	C .tl_set:N = \l_nicolas_commutativesquare_C_tl,
	D .tl_set:N = \l_nicolas_commutativesquare_D_tl,
	P .tl_set:N = \l_nicolas_commutativesquare_P_tl,
	f      .tl_set:N = \l_nicolas_commutativesquare_f_tl,
	g   .tl_set:N = \l_nicolas_commutativesquare_g_tl,
	h   .tl_set:N = \l_nicolas_commutativesquare_h_tl,
	k .tl_set:N = \l_nicolas_commutativesquare_k_tl,
	l .tl_set:N = \l_nicolas_commutativesquare_l_tl,
	m .tl_set:N = \l_nicolas_commutativesquare_m_tl,
	n .tl_set:N = \l_nicolas_commutativesquare_n_tl,
	pp .tl_set:N = \l_nicolas_commutativesquare_pp_tl,
	up .tl_set:N = \l_nicolas_commutativesquare_up_tl,
	diag .tl_set:N = \l_nicolas_commutativesquare_diag_tl,	
	codiag .tl_set:N = \l_nicolas_commutativesquare_codiag_tl,		
	diaga .tl_set:N = \l_nicolas_commutativesquare_diaga_tl,	
	codiaga .tl_set:N = \l_nicolas_commutativesquare_codiaga_tl,		
	
	A		.initial:n =A,
	B		.initial:n =B,
	C		.initial:n =C,
	D		.initial:n =D,
	P		.initial:n =P,
	f    .initial:n =,
	g    .initial:n =,
	h    .initial:n =,
	k    .initial:n =,
	l    .initial:n =,
	m    .initial:n =,
	n	 .initial:n =,
	pp .initial:n=h,	
	up .initial:n=f,
	diag .initial:n=f,
	codiag .initial:n=f,
	diaga .initial:n=,
	codiaga .initial:n=,
}

\cs_new:Nn \nicolas_commutativesquare:n
{
	\renewcommand{\pushpull}{\l_nicolas_commutativesquare_pp_tl}
	\renewcommand{\testfull}{\l_nicolas_commutativesquare_up_tl}
	\renewcommand{\testdiag}{\l_nicolas_commutativesquare_diag_tl}
	\renewcommand{\testcodiag}{\l_nicolas_commutativesquare_codiag_tl}
	\begin{tikzcd}[ampersand~replacement=\&,#1]
		\if \pushpull h
			\if \testfull t
				\l_nicolas_commutativesquare_P_tl
				\arrow[bend~left]{drr}{\l_nicolas_commutativesquare_l_tl
				}
				\arrow[bend~right,swap]{ddr}{\l_nicolas_commutativesquare_m_tl}\arrow{dr}{\l_nicolas_commutativesquare_n_tl}\& \& \\
			\fi
			\if \testfull t
			\&
			\fi\l_nicolas_commutativesquare_A_tl
			\if \testdiag t
				\arrow{dr}[near~end]{\l_nicolas_commutativesquare_diaga_tl}
			\fi
			\arrow{r}{\l_nicolas_commutativesquare_f_tl} \arrow{d}[swap]{\l_nicolas_commutativesquare_g_tl} \&
			\l_nicolas_commutativesquare_B_tl
			 \arrow{d}{\l_nicolas_commutativesquare_h_tl}
			 \if \testcodiag t
			 \arrow{dl}[near~end,swap]{\l_nicolas_commutativesquare_codiaga_tl}
			 \fi
			 \\
			\if \testfull t
			\&
			\fi\l_nicolas_commutativesquare_C_tl
			\arrow{r}[swap]{\l_nicolas_commutativesquare_k_tl} \& \l_nicolas_commutativesquare_D_tl
		\else \if \pushpull l
			\if \testfull t
			\l_nicolas_commutativesquare_P_tl
			\& \& \\
			\& \arrow{ul}[swap]{\l_nicolas_commutativesquare_n_tl}
			\fi
			\l_nicolas_commutativesquare_A_tl
			  \&
			\l_nicolas_commutativesquare_B_tl
			\if \testfull t
			\arrow[bend~right,swap]{ull}{\l_nicolas_commutativesquare_l_tl
			}
			\fi
			 \arrow{l}[swap]{\l_nicolas_commutativesquare_f_tl}
			 \\
			\if \testfull t
			\& \arrow[bend~left]{uul}{\l_nicolas_commutativesquare_m_tl}
			\fi
			\l_nicolas_commutativesquare_C_tl\arrow{u}{\l_nicolas_commutativesquare_g_tl}
			\if \testcodiag t
			\arrow{ur}[near~start]{\l_nicolas_commutativesquare_codiaga_tl}
			\fi
			  \& \l_nicolas_commutativesquare_D_tl\arrow{l}{\l_nicolas_commutativesquare_k_tl}
			  \if \testdiag t
			  \arrow{ul}[near~start,swap,crossing~over]{\l_nicolas_commutativesquare_diaga_tl}
			  \fi
			\arrow{u}[swap]{\l_nicolas_commutativesquare_h_tl}
			\fi
		\fi
		
	\end{tikzcd}
}

\ExplSyntaxOff

\ExplSyntaxOn

\NewDocumentCommand{\commutativehouse}{O{}m}
{
	\group_begin:
	\keys_set:nn {nicolas/commutativehouse}{#2}
	\nicolas_commutativehouse:n {#1}
	\group_end:
}

\keys_define:nn {nicolas/commutativehouse}
{	
	A     .tl_set:N = \l_nicolas_commutativehouse_A_tl,
	B   .tl_set:N = \l_nicolas_commutativehouse_B_tl,
	C .tl_set:N = \l_nicolas_commutativehouse_C_tl,
	D .tl_set:N = \l_nicolas_commutativehouse_D_tl,
	E .tl_set:N = \l_nicolas_commutativehouse_E_tl,
	f      .tl_set:N = \l_nicolas_commutativehouse_f_tl,
	g   .tl_set:N = \l_nicolas_commutativehouse_g_tl,
	h   .tl_set:N = \l_nicolas_commutativehouse_h_tl,
	k .tl_set:N = \l_nicolas_commutativehouse_k_tl,
	l .tl_set:N = \l_nicolas_commutativehouse_l_tl,
	m .tl_set:N = \l_nicolas_commutativehouse_m_tl,
	diag .tl_set:N = \l_nicolas_commutativehouse_diag_tl,	
	codiag .tl_set:N = \l_nicolas_commutativehouse_codiag_tl,		
	diaga .tl_set:N = \l_nicolas_commutativehouse_diaga_tl,	
	codiaga .tl_set:N = \l_nicolas_commutativehouse_codiaga_tl,		
	
	A		.initial:n =A,
	B		.initial:n =B,
	C		.initial:n =C,
	D		.initial:n =D,
	E		.initial:n =E,
	f    .initial:n =f,
	g    .initial:n =g,
	h    .initial:n =h,
	k    .initial:n =k,
	l    .initial:n =l,
	m    .initial:n =m,	
	diag .initial:n=f,
	codiag .initial:n=f,
	diaga .initial:n=,
	codiaga .initial:n=,
}

\cs_new:Nn \nicolas_commutativehouse:n
{
	\begin{tikzcd}[ampersand~replacement=\&,#1]
		\& \l_nicolas_commutativehouse_A_tl  \arrow{dr}{\l_nicolas_commutativehouse_g_tl} \& \\
		\l_nicolas_commutativehouse_B_tl \arrow{ur}{\l_nicolas_commutativehouse_f_tl}\arrow{rr}{\l_nicolas_commutativehouse_h_tl}\arrow{d}[swap]{\l_nicolas_commutativehouse_k_tl} \& \&  \l_nicolas_commutativehouse_C_tl\arrow{d}{\l_nicolas_commutativehouse_l_tl} \\
		\l_nicolas_commutativehouse_D_tl\arrow[swap]{rr}{\l_nicolas_commutativehouse_m_tl}\& \& \l_nicolas_commutativehouse_E_tl
	\end{tikzcd}
}

\ExplSyntaxOff

\newcommand{\redhomlgy}[2]{
	\tilde{H}_{#1}\lrprth{#2}
}
\newcommand{\copyandpaste}{t}
\newcommand{\moncategory}[2]{Mon_{#1}\lrprth{-,#2}}
\newcommand{\epicategory}[2]{Mon_{#1}\lrprth{#2,-}}
\theoremstyle{definition}
\newtheorem{define}{Definición}
\newtheorem*{definesn}{Definición}
\newtheorem{lem}{Lema}
\newtheorem*{lemsn}{Lema}
\newtheorem{teor}{Teorema}
\newtheorem*{teosn}{Teorema}
\newtheorem{prop}{Proposición}
\newtheorem*{propsn}{Proposición}
\newtheorem{coro}{Corolario}
\newtheorem*{obs}{Observación}
