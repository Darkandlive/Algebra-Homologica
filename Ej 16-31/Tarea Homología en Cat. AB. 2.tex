\documentclass{article}
\usepackage[utf8]{inputenc}
\usepackage{mathrsfs}
\usepackage[spanish,es-lcroman]{babel}
\usepackage{amsthm}
\usepackage{amssymb}
\usepackage{enumitem}
\usepackage{graphicx}
\usepackage{caption}
\usepackage{float}
\usepackage{amsmath,stackengine,scalerel,mathtools}
\usepackage{xparse, tikz-cd, pgfplots}
\usepackage{xstring}
\usepackage{mathrsfs}
\usepackage{comment}
\usepackage{faktor}
\usepackage[all]{xy}
\usepackage{tikz-cd}

\input{C:/Users/Darkandlive/Desktop/Latex/Álgebra/Tercer semestre/Comandos Matemagicos}
\title{Ejercicios 16-31}
\author{Arruti, Sergio}
\date{}
\begin{document}
	\maketitle
	%Lema para caracterizar monos y epis en Sets y, en consecuencia, en Mod(R)
	\begin{lem}
		Sea $f$ un morfismo en $Sets$, entonces
		\begin{enumerate}[label=$\alph*)$]
			\item $\catarrow{f}{A}{B}{m}$ es un mono en $Sets$ si y sólo si $f$ es inyectiva;
			\item $\catarrow{f}{A}{B}{e}$ es un epi en $Sets$ si y sólo si $f$ es suprayectiva.
		\end{enumerate}
		\begin{proof}
			\boxed{a)} Notemos primeramente que una función vacía $\varnothing_C$, $C\in Sets$, es inyectiva por la vacuidad de su dominio. Más aún, es un mono en $Sets$, en efecto: si $g,h\in\ringmod{Sets}{D}{\varnothing}$ son tales que $\varnothing_C f=\varnothing_A g$, entonces necesariamente $D=\varnothing$ y así, dado que existe una única función de $\varnothing$ en $\varnothing$, $f=g$. Con lo cual la afirmación es válida para funciones vacía y podemos suponer sin pérdida de generalidad que $A\neq\varnothing$ (y en consecuencia que $B\neq\varnothing$).\\
			\boxed{a)\implies} Sean $a,b\in A$ tales que $f\lrprth{a}=f\lrprth{b}$, entonces las funciones
			\begin{align*}
				\descapp{g}{A}{A}{x}{a}{,}\\
				\descapp{h}{A}{A}{x}{b}{,}
			\end{align*}
			satisfacen que $fg=fh$, luego $g=h$ por  ser $f$ mono y por tanto $a=b$.\\
			\boxed{a)\impliedby} Supongamos que  $g,h\in\ringmod{Sets}{A'}{A}$ son tales que $fg=fh$. Si $A'=\varnothing$ entonces $g=\varnothing_A=h$; en caso contrario sea $a\in A'$, así
			\begin{align*}
				f\lrprth{g\lrprth{a}}&=fg\lrprth{a}=fh\lrprth{a}=f\lrprth{h\lrprth{a}}\\
				\implies g\lrprth{a}&=h\lrprth{a}, && f\text{ es inyectiva}\\
				\implies g=h.
			\end{align*}
			\boxed{b)} Verificaremos primero que la función $\varnothing_\varnothing$ i.e. la única función cuyo dominio y contradominio es $\varnothing$ es epi y suprayectiva. Si  $g,h\in\ringmod{Sets}{\varnothing}{Z}$ son tales que $g\varnothing_\varnothing=h\varnothing_\varnothing$, entonces $g=\varnothing_Z=h$; por su parte la suprayectividad de $\varnothing_\varnothing$ se sigue por la vacuidad de su contradominio. Así, en adelante podemos suponer sin pérdida de generalidad que $B\neq\varnothing$.\\
			\boxed{b)\implies} Notemos que necesariamente $A\neq\varnothing$, pues en caso contrario las aplicaciones
			\begin{align*}
				\descapp{\phi}{B}{\lrbrack{0,1}}{x}{0}{,}\\
				\descapp{\psi}{B}{\lrbrack{0,1}}{x}{1}{,}
			\end{align*}			
			 son funciones bien definidas, pues $B\neq\varnothing$, las cuales satisfacen que $\phi\neq\psi$ y sin embargo $\phi f=\varnothing_{\lrbrack{0,1}}=\psi f$, lo cual contradeciría que $f$ es epi. Así $\restrict{1_B}{f\lrprth{A}}$ no es una función vacía y más aún satisface que
			\begin{align*}
				\restrict{1_B}{f\lrprth{A}}f&=f=1_B f\\
				\implies 1_B&=\restrict{1_B}{f\lrprth{A}}, && f\text{ es epi}\\
				&\implies f\lrprth{A}=B\\
				&\implies f\text{ es suprayectiva.}
			\end{align*}
		\boxed{b)\impliedby} Sean $g,h\in\ringmodhom{Sets}{B}{C}$ tales que $gf=hf$ y $b\in B$. Como $f$ es suprayectiva $\exists\ a\in A$ $f\lrprth{a}=b$, así
		\begin{align*}
			g\lrprth{b}&=gf(a)=hf\lrprth{a}=h\lrprth{b}\\
			\implies g&=h.
		\end{align*}
		\end{proof}
	\end{lem}
	\begin{enumerate}[label=\textbf{Ej \arabic*.}]
	\setcounter{enumi}{15}
	%16
\item
\item Pruebe que, para un anillo $R$, La categoría $Mod(R)$ tiene uniones.\\
\begin{proof}
Sean $A\in Mod(R)$, $\{\alpha_i \catarrow{}{A_i}{A}{m}\}_{i\in I}$ en $Mod(R)$ y la inclusión de submódulos
\[
\nu\colon \displaystyle \sum_{i\in I}Im(\alpha_i)\longrightarrow A.
\]
Recordemos que $\left(x\in \displaystyle\sum_{i\in I}Im(\alpha_i)\,\,\iff\,\, x=\sum_{i\in J}\alpha_j(a_j)\right)$ \\
con $J$ finito y $a_j\in A_j$ para cada $j\in J$.\\

\boxed{U_1)}$\quad  (\alpha_i\leq \nu\,\,\forall i\in I)$\\

Como $\alpha_i(x)\in Im(\alpha_i)\,\,\forall x\in A_i$, entonces definimos $\nu_i : A_i\to Im(\alpha_i)$ como $\nu_i(x)=\alpha_i(x)$.
Observemos que $\nu_i(x)\in \displaystyle\sum_{i\in I}Im(\alpha_i)$ pues si $J=\{i\}$ entonces 
$\displaystyle\sum_{i\in J}\alpha_i(x)=\alpha_i(x)=\nu(x)$.\,\, Por lo tanto $\alpha_i(x)=\nu\circ\nu_i(x)$ y así $\alpha_i\leq \nu\,\,\,\,\forall i\in I$.\\

\boxed{U_2)}\quad  Supongamos $f:A\to B$ en $\mathscr{C}$ es tal que cada $u_i$ es llevado via $f$, a algún subobjeto $\mu\catarrow{}{B'}{B}{m}$. 
Tal como se muestra en el siguiente diagrama:

\begin{tikzcd}
\displaystyle\sum_{i\in I}Im(\alpha_i)
\arrow[bend right]{ddr}[swap]{\nu}
 & & \\
&A_i \arrow[dotted]{r}{f'_i} \arrow[hook]{d}[swap]{\alpha_i} & B' \arrow[hook]{d}{\mu} \\
& A \arrow{r}[swap]{f} & B
\end{tikzcd}




Como para todo $x\in \displaystyle\sum_{j\in I}Im(\alpha_j)$, \,\,\,$x=\alpha_{i_0}(x_0)+\ldots+\alpha_{i_n}(x_n)$ donde $x_n\in A_{i_n}$\quad e\quad
 $i_n\in J\,\, \forall i\in\{1,\ldots,n\}$, así definimos $g : \displaystyle\sum_{j\in I}Im(\alpha_j)\longrightarrow B'$ como 
$g(x)=f'_{i_0}(x_o)+\ldots+f'_{i_n}(x_n)$.\\
Observemos que es morfismo de módulos:\\

Sean $r\in R$, $a,b\in \displaystyle\sum_{i\in I}Im(\alpha_i)$ y supongamos que 
\begin{align*}
a&=\alpha_{h_0}(a_0)+\ldots+\alpha_{h_n}(a_n)\\
b&=\alpha_{k_0}(b_0)+\ldots+\alpha_{k_m}(b_m)\qquad n,m\in \mathbb{N}.
\end{align*}

Así 
\begin{gather*}
g(ra+b)=g\left(r\alpha_{h_0}(a_0)+\ldots+r\alpha_{h_n}(a_n)+r\alpha_{k_0}(b_0)+\ldots+r\alpha_{k_m}(b_m)\right)\\
=g\left(\alpha_{h_0}(ra_0)+\ldots+\alpha_{h_n}(ra_n)+\alpha_{k_0}(b_0)+\ldots+\alpha_{k_m}(b_m)\right)\\
=f'_{h_0}(ra_0)+\ldots+f'_{h_n}(ra_n)+f'_{k_0}(b_0)+\ldots+f'_{k_m}(b_m)\\
=\left(rf'_{h_0}(a_0)+\ldots+rf'_{h_n}(a_n)\right)+f'_{k_0}(b_0)+\ldots+f'_{k_m}(b_m)\\
=r\left(f'_{h_0}(a_0)+\ldots+f'_{h_n}(a_n)\right)+f'_{k_0}(b_0)+\ldots+f'_{k_m}(b_m)\\
=rg(a)+g(b).
\end{gather*}
Por lo tanto es morfismo.\\

Así $\forall x\in \displaystyle\sum_{i\in I}Im(\alpha_i)$ se tiene que 
\begin{gather*}
\mu g(x)=\mu \left(\sum_{k=0}^nf'_{i_k}(x_k)\right)
=\sum_{k=0}^n\mu f'_{i_k}(x_k)\\
=\sum_{k=0}^n f\alpha_{i_k}(x_k)
=f\left(\sum_{k=0}^n\alpha_{i_k}(x_k)\right)\\
=f\nu(x).
\end{gather*}
Por lo tanto $\displaystyle\sum_{i\in I}Im(\alpha_i)$ es la unión categorica.

\end{proof}

\item Sean $\mathscr{C}$ una categoría con ecualizadores $\alpha,\beta\colon A\to B$ y \\ $\{\mu_i\catarrow{}{A_i}{A}{m}\}_{i\in I}$ tal que 
existe $\mu:\displaystyle\bigcup_{i\in I}A_i\longrightarrow A.$ Pruebe que \\ $\left(\alpha\mu_i=\beta\mu_i\,\,\,\forall i\in I\right)\Rightarrow
\left(\alpha\mu=\beta\mu\right).$
\begin{proof}

Supongamos $\alpha\mu_i=\beta\mu_i\,\,\,\forall i\in I$ entonces se tiene el siguiente diagrama:
\\
\centerline{
\xymatrix{
A_i\ar[d]^{\mu_i}          \\
A \ar@<1ex>[r]^\beta\ar@<-1ex>[r]_\alpha & B\,\,.}\,
}
Como $\mathscr{C}$ tiene ecualizadores, existe $\eta:K\to A$ tal que $\alpha\eta=\beta\eta$ y si $f:X\to A$ en $\mathscr{C}$ es tal que
 $\beta f=\alpha f$, entonces $\exists ! f':X\to K$ tal que $\eta f'=f$. \\

Así como $\alpha\mu_i=\beta\mu_i\,\,\forall i\in I$, entonces para cada $i\in I$ $\exists ! \mu_i':A_i\to K$ tal que $\eta\mu_i'=\mu_i$, es decir, se tiene que 
para cada $i\in I$ el siguiente diagrama conmuta:

\centerline{
\xymatrix{
& A_i\ar@{-->}[dl]_{\exists ! f_i}\ar@{^{(}->}[d]^{\mu_i} \\
K\ar[r]_\eta & A\ar@<1ex>[r]^\beta\ar@<-1ex>[r]_\alpha & B
}}
es decir, $\eta f_i=\mu_i$. Ahora por $(U_1)$ de las propiedades de la unión, se tiene que el siguiente diagrama conmuta

 \centerline{
\xymatrix{
& A_i\ar@{-->}[dl]_{f'_i}\ar@{^{(}->}[d]^{\mu_i} \\
\displaystyle\bigcup_{i\in I}A_i\ar[r]_\mu & A\ar@<1ex>[r]^\beta\ar@<-1ex>[r]_\alpha & B
}}

por lo que $\mu f'_i=\mu_i=\eta f_i$.\\

Notemos entonces que, como $\alpha\eta=\beta\eta$, se tiene el siguiente diagrama para cada $i\in I$ y para $f$ igual a $\alpha$ y $\beta$:

\begin{tikzcd}
\displaystyle\bigcup_{i\in I}A_i
\arrow[bend right]{ddr}[swap]{\mu}
 & & \\
&A_i \arrow{r}{f'_i} \arrow[hook]{d}[swap]{\alpha_i} & K \arrow[hook]{d}{\alpha\eta} \\
& A \arrow{r}[swap]{f} & B
\end{tikzcd}

Entonces por $(U_2)$ de las propiedades de la unión, existen \\$\gamma_\alpha,\gamma_\beta:\displaystyle\bigcup_{i\in I}A_i\longrightarrow K$ tal que 
$\alpha\eta\gamma_\alpha=\alpha\mu$ y $\alpha\eta\gamma_\beta=\beta\mu$. Pero $\beta\eta=\alpha\eta$ por lo tanto 
$\alpha\mu=\alpha\eta\gamma_\alpha=\beta\eta\gamma_\beta=\beta\mu$. 













Además, como $\alpha\mu_i=\beta\mu_i$, entonces se tiene el siguiente diagrama conmutativo:

\centerline{
\xymatrix{
& A_i\ar@{-->}[dl]^{\exists ! \mu_i'}\ar@{^{(}->}[d]^{\mu_i}\ar[r]^{\beta\mu_i}& B\ar[d]^{Id} \\
K\ar[r]_\eta & A\ar[r]^\alpha & B\\
& \displaystyle \bigcup_{i\in I}A_i\ar[u]^\mu
}}

Entonces, por la propiedad universal de la unión existe $\theta: \displaystyle \bigcup_{i\in I}A_i\longrightarrow B$ tal que $\alpha\mu=Id\theta=\theta$.
Análogamente se tiene que existe $\theta': \displaystyle \bigcup_{i\in I}A_i\longrightarrow B$ tal que $\beta\mu=Id\theta'=\theta'$, y como $\mathscr{C}$ 
tiene igualadores entonces existe $\gamma:X\to  \displaystyle \bigcup_{i\in I}A_i$ tal que $\mu_i=\eta\mu_i'$.\\

Por lo que $ \displaystyle \bigcup_{i\in I}A_i$





\end{proof}




\item 

\item

\item Pruebe que Sets tiene coimagenes.

\begin{proof}
Sea $f:A\to B$ en Sets. Consideremos la relación $\sim_f$ en $A$, donde $x\sim_f y$ si y sólo si $f(x)=f(y)$.\\

Esta relación (que denotaremos por $\sim$ por simplicidad) es una relación de equivalencia como se muestra a continuación:\\

\boxed{\text{Reflexividad}}\quad Sea $x\in A$, como $f(x)=f(x)$ entonces $x\sim x$.\\

\boxed{\text{Simetr\'ia}}\quad Sean $a,b\in A$ tales que $a\sim b$, entonces $f(a)=f(b)$, por lo que $f(b)=f(a)$ y así $b\sim a$.\\

\boxed{\text{Transitividad}}\quad Sean $x,y,z\in A$ tales que $x\sim y$, $y\sim z$, entonces $f(x)=f(y)=f(z)$ por lo tanto $f(x)=f(z)$ y en consecuencia
$x\sim z$.\\

Sea $\pi:a\to \faktor{A}{\sim}$ el epi canonico donde $\pi(a)=[a]:=\{x\in A\,|\,x\sim a\}$, se afirma que es una coimagen de $f$.\\

Observemos que, si $A,B\neq \emptyset$, para toda $b\in B$ tal que $b=f(a)$ con $a\in A$ se tiene que $\pi(a)=[a]$ por lo que se puede definir 
$f': \faktor{A}{\sim}\to B$ como $f'([a])=f(a)$. Así se tiene que:\\

(1) $f'$ está bie definida.\\

Sean $[a][b]\in [x]$ con $[x]\in  \faktor{A}{\sim}$, entonces $a\sim x\sim b$, por lo que \\$f(a)=f(x)=f(b)$, es decir, $f'([a])=f'([x])=f'([b])$.\\

(2) $(f=f'\pi)$.\\

Sea $a\in A$. $f'\pi(a)=f'([a])=f(a)$.\\

Para ver que $(CoIm_2)$ se cumple, supongamos que existe $p'\catarrow{}{A}{J'}{e}$  un objeto cociente de $A$ tal que $\exists f'':J'\to B$ donde 
$f=f''p'$.\\
Sea $a\in A$, entonces $\pi(a)=[a]$\,\,\,y\,\,\,$p'(a)=a'\in J'$. Como $p'$ es epi en Sets entonces es supra, así para todo $x\in J'$ existe $a_x\in A$
tal que $p'(a_x)=x$, así definimos $\nu:J' \to \faktor{A}{\sim}$ como $\nu(x)=\pi(a_x)$.\\

Se tiene entonces que $\forall a\in A,\quad \nu p'(a)=\nu(p'(a))=\pi(a).$\\

En el caso de que $B$ sea el conjunto vacio, entonces $A$ tiene que ser el conjunto vacio y $f:A\to B$ es la función vacia, así $f=p$ tiene que ser su
coimagen pues si $f':B\to B$ es la función identidad en $B$, entonces $f=f'p$ y si $p'\catarrow{}{B}{B}{e}$ es un objeto cociente de $A$ tal que $f'':J'\to B$
con $f''p'=f$ entonces $f'':J'\to B$ es la función vacia y $J'$ es el conjunto vacio. Así $p':A\to J'$ es la función vacia y por lo tanto $p'=p$ y 
$Id_{J'}\circ p'=p$.\\

En caso de que $A$ sea el conjunto vacio y $B$ sea distinto del vacio, entonces $(CoIm_1)$ se cumple igual que en el caso anterior, tomando a 
$p: \emptyset\to  \emptyset$.\\

Para probar $(CoIm_2)$ supongamos que $p'\catarrow{}{A}{J'}{e}$ es un objeto cociente de $A$ tal que $exists f'':J'\to B$ tal que $f=f''p'$, pero
$p'$ es epi, y como $A=\emptyset$ entonces $J'=\emptyset$. Así, si definimos $u$ como la identidad en el vacio se tiene que $p=up'$.

\end{proof}

\item Pruebe que $Mod(R)$ tiene coimagenes.\\

\begin{proof}
Sea $A\in Obj(Mod(R)),$ entonces  $A\neq \emptyset$. Se afirma que el epi canonico $\pi:A\to \faktor{A}{Ker(f)}$ es una coimagen.\\

Sea $a\in A$, entonces $f(a)\in B$. Definimos $f': \faktor{A}{Ker(f)}\to B$ como $f'([a])=f(a)$.\\

Probemos que está bien definido. Sean $a,b\in [x]$ entonces \\$a+k_1=b+k_2=x$ con $k_1, k_2\in Ker(f)$, asi

\begin{align*}
f'([a])=f(a)=f(a)+f(K_1)=f(a+K_1)\\
=f(b+K_2)=f(b)+f(K_2)=f(b)=f'([b]).
\end{align*}

Veamos que es morfismo. Sean $r\in R,\,\,[a],[b]\in  \faktor{A}{Ker(f)}$ entonces 

\begin{align*}
f'(r[a]+[b])=f'([ra+b])=f(ra+b)=rf(a)+f(b)=f'(r[a])+f'([b]).
\end{align*}

En consecuencia se tiene que $\pi$ cumple $(CoIm_1)$.\\

Ahora supongamos que $p'\catarrow{}{A}{J'}{e}$ es un objeto cociente de $A$ tal que existe $f'':J'\to B$ que cumple que $f=f''p'$. Como $p'$ es
epi, entonces es suprayectiva en $Mod(R)$, porlo que para cada $x\in J'$ existe $a\in A$ tal que $p'(a)=x$.\\

Definimos $\nu:J'\to \faktor{A}{Ker(f)}$ como $\nu(x)=[a]$ donde $p'(a)=x$. Esta función está bien definida pues si $a,b\in A$ son tales que 
$p'(a)=p'(b)$ entonces $f''p'(a)=f''p'(b)$ y así $f(a)=f(b)$, entonces $f(a-b)=0$, por lo que $a-b\in Ker(f)$ y en consecuencia $[a]=[b]$.\\

Veamos que $\nu$ es morfismo. Si $r\in R\,\,a,b\in J'$ donde $\nu(a)=[x]$, $\nu(b)=[y]$, $a=p'(x)$ y $b=p'(y)$, entonces 
\begin{gather*}
\nu(ra+b)=\nu(rp'(x)+p'(y))=\nu(p'(rx+y))\\
=[rx+y]=r[x]+[y]=r\nu(a)+\nu(b).
\end{gather*}

Así se tiene que $\forall a\in A\,\,\,\nu p'(a)=\nu(p'(a))=[a]=\pi(a)$ por lo que $(CoIm_2)$ se cumple y $Mod(R)$ tiene coimagenes.

\end{proof}


\item

\item

\item Considere el siguiente diagrama conmutativo en una categoría $\mathscr{C}$

\centerline{
\xymatrix{
& B'\ar@{^{(}->}[d]^{h} \\
A\ar[dr]_{f'}\ar[r]^f & B\\
& I\ar@{^{(}->}[u]_\mu
}}
Pruebe que: si $\exists f^{-1}(B')$ y $B'\cap Y$, entonces $f^{-1}(I\cap B')= f^{-1}(B')$ en $\overline{Mon_{\mathscr{C}}(-,A)}.$ 

\begin{proof}
Como $f^{-1}(B')$ y $B'\cap I$ existen, entonces se tienen los siguientes diagramas conmutativos:\\

\centerline{
\xymatrix{
f^{-1}(B')\ar[r]^{\beta_2}\ar@{^{(}->}[d]_{\beta_1} & B'\ar@{^{(}->}[d]^{h} & & & & 
I\cap B'\quad \ar[d]_{\nu_2}\ar[r]^{\nu_1}\ar@{^{(}->}[dr]^{i} & I\ar@{^{(}->}[d]^{\mu}\\
A\ar[r]^{f}\ar[dr]_{f'}& B & & & & B'\ar@{^{(}->}[r]_{h}& B\\
& I\ar@{^{(}->}[u]_{\mu} & & & 
}}
Así se tiene que este diagrama

\centerline{
\xymatrix{
f^{-1}(B')\ar[r]^{f'\beta_1}\ar[dr]^{f\beta_1}\ar[d]_{\beta_2} &I\ar[d]^{\mu}\\
B'\ar[r]_{h}& B
}}
es conmutativo. Por lo tanto, como $I\cap B'$ es pull-back existe un único $ \gamma: f^{-1}(B')\to I\cap B'$ tal que el siguiente diagrama conmuta:\\

\centerline{
\xymatrix{
f^{-1}(B')\ar[r]^{\gamma}\ar@{^{(}->}[d]_{\beta_1} & I\cap B'\ar@{^{(}->}[d]_i \ar[r]^{\nu_2}& B'\ar@{^{(}-}[dl]^{h}\\
A\ar[r]_{f}& B & &\ldots (1)
}}

Sean $\eta: X\to I\cap B'$, \, $\eta_2: X\to A$ tales que $i\eta_1=f\eta_2$.\\

Observamos que, entonces, $\nu_2\eta_1 : X\to B'$ y es tal que $h(\nu_2\eta_1)=i\eta_1=f\eta_2$.\\

Así, como $f^{-1}(B')$ es pull-back de \xymatrix{A\ar[r]^{f}& B & B'\ar[l]_\mu}, existe una única $\gamma' : X\to f^{-1}(B')$ tal que 
$\nu_2\gamma\gamma'=\nu_2\eta_1$ y $\beta_1\gamma'=\eta_2$ pero $\nu_2$ es mono por ser $i$ mono. Entonces 
$\gamma\gamma'=\eta_1$ y $\beta_1\gamma'=\eta_2$.\\

Ahora, si existiera $\alpha: X\to f^{-1}(B')$ tal que $\beta_1\alpha=\eta_2$ y $\gamma\alpha=\eta_1$, entonces 
$\nu_2\gamma\alpha=\gamma_2\eta_1$ y por lo anterior $\alpha=\gamma'$ pues es el único con esas propiedades. Por lo tanto 
$f^{-1}(B')$ es un pull-back, del diagrama (1), e implica que $f^{-1}(I\cap B')$ existe y sea igual a $f^{-1}(B)$ con los morfismos
$\gamma$ y $\beta_1$.

\end{proof}

\item Sea $f:A\to B$ en una categoría $\mathscr{C}$. Consideremos subobjetos $A_1\subseteq A_2\subseteq A$ y  $B_1\subseteq B_2\subseteq B$.
Pruebe que se satisfacen las siguientes relaciones cada vez que ambos lados estén definidos.

\begin{align*}
a)&\quad f(A_1)\subseteq f(A_2)\\
b)&\quad f^{-1}(B_1)\subseteq f^{-1}(B_2)\\
c)&\quad A_1\subseteq f^{-1}(f(A_1))\\
d)&\quad f(f^{-1}(B_1))\subseteq B_1
\end{align*}

\begin{proof}

Comenzaremos por nombrar monomorfismos correspondientes como subobjetos de $A$ y de $B$\\
\centerline{
\xymatrix{
A_1\ar@{^{(}->}[r]^{\mu_1} & A_2\ar@{^{(}->}[r]^{\mu_2} & A\\
B_1\ar@{^{(}->}[r]^{\gamma_1} & B_2\ar@{^{(}->}[r]^{\gamma_2} & B
}}\\

\boxed{a)} Sabemos que $f(A_1)=Im(f\mu_2\mu_1)$ y $f(A_2)=Im(f\mu_2)$. Llamaremos \\

\centerline{
$\mu_1':Im(f\mu_2\mu_1)\to B$\,,\qquad $\alpha_1:A_2\to Im(f\mu_2\mu_1)$\,,}

\centerline{
$\mu_2':Im(f\mu_2)\to B$\quad y \quad $\alpha_2:A_2\to Im(f\mu_2)$
}

\,\\ a los morfismos tales que $f\mu_2=\mu_2'\alpha_2$\quad y\quad $f\mu_2\mu_1=\mu_1'\alpha_1$.\\
Entonces $f\mu_2\mu_1=(\mu_2'\alpha_2)\mu_1\,.$ Por la propiedad universal de la imagen en $Im(f\mu_2\mu_1)$ existe 
$\gamma : Im(f\mu_2\mu_1)\to Im(f\mu_2)$ tal que $\mu_2'\gamma=\mu_1$.\\
 En particular $\gamma$ es mono, entonces $Im(f\mu_2\mu_1)\subseteq Im(f\mu_2)$ y así \\$f(A_1)\subseteq f(A_2)$.\\

\boxed{b)} Como se tienen los siguientes diagramas conmutativos\\

\centerline{
\xymatrix{
f^{-1}(B_1)\ar@{^{(}->}[d]_{\beta_1}\ar[r]^{\beta_2} & B_1\ar@{^{(}->}[d]^{\nu_2\nu_1} & & &
f^{-1}(B_2)\ar@{^{(}->}[d]_{\beta_1'}\ar[r]^{\beta_2'} & B_2\ar@{^{(}->}[d]^{\nu_2}\\
A\ar[r]_f & B & & & A\ar[r]_f & B 
}}

en particular se tiene que $f\beta_1=\nu_2(\nu_1\beta_2)$ y este diagrama es conmutativo:\\

\centerline{
\xymatrix{
f^{-1}(B_1)\ar@{^{(}->}[d]_{\beta_1}\ar[r]^{\nu_1\beta_2} & B_2\ar@{^{(}->}[d]^{\nu_2}\\
A\ar[r]_f & B 
}}

Entonces $\exists \eta : f^{-1}(B_1)\to f^{-1}(B_2)$ tal que $\beta_2'\eta=\nu\beta_2$\,\,y\,\,$\beta_1'\eta=\beta_1$\\

Como $f^{-1}(B_2)$ es pull-back, y $\nu_2$ es mono, entonces $\beta_1'$ es mono y por lo tanto $\eta$ es mono. Así $f^{-1}(B_1)\subseteq f^{-1}(B_2)$.
\\

\boxed{c)} Puesto que $f^{-1}(f(A_1))$ es un pull back, tenemos un diagrama conmutativo de la siguiente forma:\\

\centerline{
\xymatrix{
f^{-1}(f(A_1))\ar[d]_{f_1}\ar[r]^{f_2} & f(A_1)\ar[d]^{\mu_1'}\\
A\ar[r]_f & B
}}

Además (apoyandonos con la notación del inciso a) ) tenemos que el siguiente diagrama conmuta

\centerline{
\xymatrix{
A_1\ar[d]_{\mu_2\mu_1}\ar[r]^{\alpha_1} & f(A_1)\ar[d]^{\mu_1'}\\
A\ar[r]_f & B
}}

Entonces, por ser $f^{-1}(f(A_1))$ un pull-back, $\exists ! g:A_1\to f^{-1}(f(A_1))$ tal que $f_2g=\alpha_1$ y $f_1g=\mu_2\mu_1$.\\

Como $\mu_2\mu_1$ es mono por ser $\mu_2$ y $\mu_1$ monos, entonces $g$ es mono y así $A_1\subseteq f^{-1}(f(A_1))$.\\

\boxed{d)} Observemos que, como $f^{-1}(B_1)$ es pull-back, el diagrama\\

\centerline{
\xymatrix{
f^{-1}(B_1)\ar[r]^{\beta_2}\ar@{^{(}->}[d]_{\beta_1} & B_1\ar@{^{(}->}[d]^{\nu_2\nu_1}\\
A\ar[r]_{f} & B
}}

conmuta, entonces por propiedades de las imagenes, existen \\
$\mu:\catarrow{}{Im(f\beta_1)}{B}{m}$ y $f':f^{-1}(B_1)\to Im(f\beta_1)$ tales que el siguiente diagrama\\

\centerline{
\xymatrix{
f^{-1}(B_1)\ar@{^{(}->}[d]_{\beta_2}\ar[dr]_{f\beta_1}\ar[r]^{f'} & Im(f\beta_1)\ar@{^{(}->}[d]^{\mu}\\
B_1\ar[r]_{\nu_2\nu_1} & B
}}

es un diagrama conmutativo, por lo que existe un único \\$g':Im(f\beta_1)\to B_1$, tal que $\nu_2\nu_1 g'=\mu$\,\, y \,\,$gf'=\beta_2$ dado por la\\
propiedad universal de las imagenes. Mas aún, notemos que $g'$ es mono, pues $\mu$ es mono y $\mu=\nu_2\nu_1g'$. 
Así  $f\beta_1=\nu_2\nu_1\beta_2$. \\
Por lo que $Im(f\beta_1)=f(f^{-1}(B_1))\subseteq B_1.$


\end{proof}



\item

\item

\item Pruebe que $Mod(R)$ tiene kerneles.

\begin{proof}

Sea $f:A\to B$ morfismo en $Mod(R)$,\,\, \\$K=\{x\in A\,|\,f(x)=0\}$ y $\mu:K\to A$ la función inclusión.\\

Primero demostraremos que $K\leq A$. \\

Sean $r\in R$ $a,b\in K$, entonces $f(ra+b)=rf(a)+f(b)=r\cdot 0+0=0$, por lo tanto $ra+b\in K$, entonces $K\in Mod(R)$ y $\mu$ es morfismo.\\

\boxed{Ker_1} Como $f\mu:K\to B$ y para toda $x\in K$ se tiene que $f\mu(x)=f(\mu(x))=f(x)=0$ entonces $f\mu=0$.\\

\boxed{Ker_2} Supongamos $g:X\to A$ es un morfismo tal que $fg=0$, entonces $g(x)\in K$ pues $f(g(x))=0$. Así
definimos el morfismo $h:X\to K$ tal que $h(x)=g(x)$, entonces $\mu h(x)=\mu(g(x))=g(x)$ \,\,$\forall x\in X$, por lo tanto $\mu h=g$ y así $K$ es
kernel de $f$.\\

Por lo tanto $Mod(R)$ tiene kernels.

\end{proof}

\item Pruebe que $Mod(R)$ tiene cokernels.

\begin{proof}
Sea $f:M\to N$ en $Mod(R)$. Como $f$ es morfismo de \\ $R-$ módulos, entonces $im(f)\leq N$. \\

Consideremos $\pi:N\to \faktor{N}{Im(f)}$, donde $\pi(k)=k+Im(f)$ es la proyección canónica. Se afirma que $\pi$ es un cokernel de $f$.\\

\boxed{CoKer_1} Para toda $x\in M$ se tiene que $\pi f(x)=\pi(f(x))=0$ pues $f(x)\in Im(f)$.\\

\boxed{CoKer_2} Propiedad universal. Supongamos existe $g:N\to X$ un morfismo de modulos tal que $gf=0$, entonces definimos 
$g':\faktor{N}{Im(f)}\to X$ de tal forma tal que $\forall [x]\in \faktor{N}{Im(f)},\quad g'([x])=g(x)$, donde $[x]$ es el representante de
la clase de equivalenia de $x$.\\

Sean $[x],[y]\in \faktor{N}{Im(f)}$ y $r\in R$, entonces 
\[g'(r[x]+[y])=g'([rx+y])=g(rx+y)=rg(x)+g(y)=rg'(x)+g'(y).\]

Observamos que $g'$ está bien definida, pues si $a,b\in [x]$, etonces existen $k_1,k_2\in Im(f)$ tales que $a+k_1=b+k_2=x$ y $g(k_1)=g(k_2)=0$,
entonces 
\[g'([a])=g(a)=g(a)+g(k_1)=g(a+k_1)=g(b+k_2)=g(b)+g(k_2)=g(b)=g([b]).\]

Por lo tanto $g'$ es un morfismo de $R$-módulos y $g'\pi(x)=g'([x])=g(x)$ por lo que $g'\pi=g$ y así $\pi$ es Cokernel.









\end{proof}








	\end{enumerate}		
\end{document}