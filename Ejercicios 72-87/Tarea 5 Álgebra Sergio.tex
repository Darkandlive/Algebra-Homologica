\documentclass{article}
\usepackage[utf8]{inputenc}
\usepackage{mathrsfs}
\usepackage[spanish,es-lcroman]{babel}
\usepackage{amsthm}
\usepackage{amssymb}
\usepackage{enumitem}
\usepackage{graphicx}
\usepackage{caption}
\usepackage{float}
\usepackage{amsmath,stackengine,scalerel,mathtools}
\usepackage{xparse, tikz-cd, pgfplots}
\usepackage{xstring}
\usepackage{mathrsfs}
\usepackage{comment}
\usepackage{faktor}
\usepackage[all]{xy}

\newcommand{\spmat}[1]{%
  \left(\begin{smallmatrix}#1\end{smallmatrix}\right)%
}





\def\subnormeq{\mathrel{\scalerel*{\trianglelefteq}{A}}}
\newcommand{\Z}{\mathbb{Z}}
\newcommand{\La}{\mathscr{L}}
\newcommand{\crdnlty}[1]{
	\left|#1\right|
}
\newcommand{\lrprth}[1]{
	\left(#1\right)
}
\newcommand{\lrbrack}[1]{
	\left\{#1\right\}
}
\newcommand{\lrsqp}[1]{
	\left[#1\right]
}
\newcommand{\descset}[3]{
	\left\{#1\in#2\ \vline\ #3\right\}
}
\newcommand{\descapp}[6]{
	#1: #2 &\rightarrow #3\\
	#4 &\mapsto #5#6 
}
\newcommand{\arbtfam}[3]{
	{\left\{{#1}_{#2}\right\}}_{#2\in #3}
}
\newcommand{\arbtfmnsub}[3]{
	{\left\{{#1}\right\}}_{#2\in #3}
}
\newcommand{\fntfmnsub}[3]{
	{\left\{{#1}\right\}}_{#2=1}^{#3}
}
\newcommand{\fntfam}[3]{
	{\left\{{#1}_{#2}\right\}}_{#2=1}^{#3}
}
\newcommand{\fntfamsup}[4]{
	\lrbrack{{#1}^{#2}}_{#3=1}^{#4}
}
\newcommand{\arbtuple}[3]{
	{\left({#1}_{#2}\right)}_{#2\in #3}
}
\newcommand{\fntuple}[3]{
	{\left({#1}_{#2}\right)}_{#2=1}^{#3}
}
\newcommand{\gengroup}[1]{
	\left< #1\right>
}
\newcommand{\stblzer}[2]{
	St_{#1}\lrprth{#2}
}
\newcommand{\cmmttr}[1]{
	\left[#1,#1\right]
}
\newcommand{\grpindx}[2]{
	\left[#1:#2\right]
}
\newcommand{\syl}[2]{
	Syl_{#1}\lrprth{#2}
}
\newcommand{\grtcd}[2]{
	mcd\lrprth{#1,#2}
}
\newcommand{\lsttcm}[2]{
	mcm\lrprth{#1,#2}
}
\newcommand{\amntpSyl}[2]{
	\mu_{#1}\lrprth{#2}
}
\newcommand{\gen}[1]{
	gen\lrprth{#1}
}
\newcommand{\ringcenter}[1]{
	C\lrprth{#1}
}
\newcommand{\zend}[2]{
	End_{\mathbb{Z}}^{#2}\lrprth{#1}
}
\newcommand{\genmod}[2]{
	\left< #1\right>_{#2}
}
\newcommand{\genlin}[1]{
	\mathscr{L}\lrprth{#1}
}
\newcommand{\opst}[1]{
	{#1}^{op}
}
\newcommand{\ringmod}[3]{
	\if#3l
	{}_{#1}#2
	\else
	\if#3r
	#2_{#1}
	\fi
	\fi
}
\newcommand{\ringbimod}[4]{
	\if#4l
	{}_{#1-#2}#3
	\else
	\if#4r
	#3_{#1-#2}
	\else 
	\ifstrequal{#4}{lr}{
		{}_{#1}#3_{#2}
	}
	\fi
	\fi
}
\newcommand{\ringmodhom}[3]{
	Hom_{#1}\lrprth{#2,#3}
}
\newcommand{\catarrow}[4]{
	\if #4e
			#1:#2\twoheadrightarrow #3
	\else \if #4m
			#1:#2\hookrightarrow #3
	\else 	#1:#2\to #3
	\fi
	\fi
}
\newcommand{\nattrans}[4]{
	Nat_{\lrsqp{#1,#2}}\lrprth{#3,#4}
}
\ExplSyntaxOn

\NewDocumentCommand{\functor}{O{}m}
{
	\group_begin:
	\keys_set:nn {nicolas/functor}{#2}
	\nicolas_functor:n {#1}
	\group_end:
}

\keys_define:nn {nicolas/functor}
{
	name     .tl_set:N = \l_nicolas_functor_name_tl,
	dom   .tl_set:N = \l_nicolas_functor_dom_tl,
	codom .tl_set:N = \l_nicolas_functor_codom_tl,
	arrow      .tl_set:N = \l_nicolas_functor_arrow_tl,
	source   .tl_set:N = \l_nicolas_functor_source_tl,
	target   .tl_set:N = \l_nicolas_functor_target_tl,
	Farrow      .tl_set:N = \l_nicolas_functor_Farrow_tl,
	Fsource   .tl_set:N = \l_nicolas_functor_Fsource_tl,
	Ftarget   .tl_set:N = \l_nicolas_functor_Ftarget_tl,	
	delimiter .tl_set:N= \_nicolas_functor_delimiter_tl,	
}

\dim_new:N \g_nicolas_functor_space_dim

\cs_new:Nn \nicolas_functor:n
{
	\begin{tikzcd}[ampersand~replacement=\&,#1]
		\dim_gset:Nn \g_nicolas_functor_space_dim {\pgfmatrixrowsep}		
		\l_nicolas_functor_dom_tl
		\arrow[r,"\l_nicolas_functor_name_tl"] \&
		\l_nicolas_functor_codom_tl
		\\[\dim_eval:n {1ex-\g_nicolas_functor_space_dim}]
		\l_nicolas_functor_source_tl
		\xrightarrow{\l_nicolas_functor_arrow_tl}
		\l_nicolas_functor_target_tl
		\arrow[r,mapsto] \&
		\l_nicolas_functor_Fsource_tl
		\xrightarrow{\l_nicolas_functor_Farrow_tl}
		\l_nicolas_functor_Ftarget_tl
		\_nicolas_functor_delimiter_tl
	\end{tikzcd}
}

\ExplSyntaxOff


\newcommand{\limseq}[3]{
	\if#3u
	\lim\limits_{#2\to\infty}#1	
		\else
			\if#3s
			#1\to
			\else
				\if#3w
				#1\rightharpoonup
				\else
					\if#3e
					#1\overset{*}{\to}
					\fi
				\fi
			\fi
	\fi
}

\newcommand{\norm}[1]{
	\crdnlty{\crdnlty{#1}}
}

\newcommand{\inter}[1]{
	int\lrprth{#1}
}
\newcommand{\cerrad}[1]{
	cl\lrprth{#1}
}

\newcommand{\restrict}[2]{
	\left.#1\right|_{#2}
}

\newcommand{\realprj}[1]{
	\mathbb{R}P^{#1}
}

\newcommand{\fungroup}[1]{
	\pi_{1}\lrprth{#1}	
}

\ExplSyntaxOn

\NewDocumentCommand{\shortseq}{O{}m}
{
	\group_begin:
	\keys_set:nn {nicolas/shortseq}{#2}
	\nicolas_shortseq:n {#1}
	\group_end:
}

\keys_define:nn {nicolas/shortseq}
{
	A     .tl_set:N = \l_nicolas_shortseq_A_tl,
	B   .tl_set:N = \l_nicolas_shortseq_B_tl,
	C .tl_set:N = \l_nicolas_shortseq_C_tl,
	f      .tl_set:N = \l_nicolas_shortseq_f_tl,
	g   .tl_set:N = \l_nicolas_shortseq_g_tl,	
	lcr   .tl_set:N = \l_nicolas_shortseq_lcr_tl,	
	
	A		.initial:n =A,
	B		.initial:n =B,
	C		.initial:n =C,
	f    .initial:n =,
	g   	.initial:n=,
	lcr   	.initial:n=lr,
	
}

\cs_new:Nn \nicolas_shortseq:n
{
	\begin{tikzcd}[ampersand~replacement=\&,#1]
		\IfSubStr{\l_nicolas_shortseq_lcr_tl}{l}{0 \arrow{r} \&}{}
		\l_nicolas_shortseq_A_tl
		\arrow{r}{\l_nicolas_shortseq_f_tl} \&
		\l_nicolas_shortseq_B_tl
		\arrow[r, "\l_nicolas_shortseq_g_tl"] \&
		\l_nicolas_shortseq_C_tl
		\IfSubStr{\l_nicolas_shortseq_lcr_tl}{r}{ \arrow{r} \& 0}{}
	\end{tikzcd}
}

\ExplSyntaxOff

\newcommand{\pushpull}{texto}
\newcommand{\testfull}{texto}
\newcommand{\testdiag}{texto}
\newcommand{\testcodiag}{texto}
\ExplSyntaxOn

\NewDocumentCommand{\commutativesquare}{O{}m}
{
	\group_begin:
	\keys_set:nn {nicolas/commutativesquare}{#2}
	\nicolas_commutativesquare:n {#1}
	\group_end:
}

\keys_define:nn {nicolas/commutativesquare}
{	
	A     .tl_set:N = \l_nicolas_commutativesquare_A_tl,
	B   .tl_set:N = \l_nicolas_commutativesquare_B_tl,
	C .tl_set:N = \l_nicolas_commutativesquare_C_tl,
	D .tl_set:N = \l_nicolas_commutativesquare_D_tl,
	P .tl_set:N = \l_nicolas_commutativesquare_P_tl,
	f      .tl_set:N = \l_nicolas_commutativesquare_f_tl,
	g   .tl_set:N = \l_nicolas_commutativesquare_g_tl,
	h   .tl_set:N = \l_nicolas_commutativesquare_h_tl,
	k .tl_set:N = \l_nicolas_commutativesquare_k_tl,
	l .tl_set:N = \l_nicolas_commutativesquare_l_tl,
	m .tl_set:N = \l_nicolas_commutativesquare_m_tl,
	n .tl_set:N = \l_nicolas_commutativesquare_n_tl,
	pp .tl_set:N = \l_nicolas_commutativesquare_pp_tl,
	up .tl_set:N = \l_nicolas_commutativesquare_up_tl,
	diag .tl_set:N = \l_nicolas_commutativesquare_diag_tl,	
	codiag .tl_set:N = \l_nicolas_commutativesquare_codiag_tl,		
	diaga .tl_set:N = \l_nicolas_commutativesquare_diaga_tl,	
	codiaga .tl_set:N = \l_nicolas_commutativesquare_codiaga_tl,		
	
	A		.initial:n =A,
	B		.initial:n =B,
	C		.initial:n =C,
	D		.initial:n =D,
	P		.initial:n =P,
	f    .initial:n =,
	g    .initial:n =,
	h    .initial:n =,
	k    .initial:n =,
	l    .initial:n =,
	m    .initial:n =,
	n	 .initial:n =,
	pp .initial:n=h,	
	up .initial:n=f,
	diag .initial:n=f,
	codiag .initial:n=f,
	diaga .initial:n=,
	codiaga .initial:n=,
}

\cs_new:Nn \nicolas_commutativesquare:n
{
	\renewcommand{\pushpull}{\l_nicolas_commutativesquare_pp_tl}
	\renewcommand{\testfull}{\l_nicolas_commutativesquare_up_tl}
	\renewcommand{\testdiag}{\l_nicolas_commutativesquare_diag_tl}
	\renewcommand{\testcodiag}{\l_nicolas_commutativesquare_codiag_tl}
	\begin{tikzcd}[ampersand~replacement=\&,#1]
		\if \pushpull h
			\if \testfull t
				\l_nicolas_commutativesquare_P_tl
				\arrow[bend~left]{drr}{\l_nicolas_commutativesquare_l_tl
				}
				\arrow[bend~right,swap]{ddr}{\l_nicolas_commutativesquare_m_tl}\arrow{dr}{\l_nicolas_commutativesquare_n_tl}\& \& \\
			\fi
			\if \testfull t
			\&
			\fi\l_nicolas_commutativesquare_A_tl
			\if \testdiag t
				\arrow{dr}[near~end]{\l_nicolas_commutativesquare_diaga_tl}
			\fi
			\arrow{r}{\l_nicolas_commutativesquare_f_tl} \arrow{d}[swap]{\l_nicolas_commutativesquare_g_tl} \&
			\l_nicolas_commutativesquare_B_tl
			 \arrow{d}{\l_nicolas_commutativesquare_h_tl}
			 \if \testcodiag t
			 \arrow{dl}[near~end,swap]{\l_nicolas_commutativesquare_codiaga_tl}
			 \fi
			 \\
			\if \testfull t
			\&
			\fi\l_nicolas_commutativesquare_C_tl
			\arrow{r}[swap]{\l_nicolas_commutativesquare_k_tl} \& \l_nicolas_commutativesquare_D_tl
		\else \if \pushpull l
			\if \testfull t
			\l_nicolas_commutativesquare_P_tl
			\& \& \\
			\& \arrow{ul}[swap]{\l_nicolas_commutativesquare_n_tl}
			\fi
			\l_nicolas_commutativesquare_A_tl
			  \&
			\l_nicolas_commutativesquare_B_tl
			\if \testfull t
			\arrow[bend~right,swap]{ull}{\l_nicolas_commutativesquare_l_tl
			}
			\fi
			 \arrow{l}[swap]{\l_nicolas_commutativesquare_f_tl}
			 \\
			\if \testfull t
			\& \arrow[bend~left]{uul}{\l_nicolas_commutativesquare_m_tl}
			\fi
			\l_nicolas_commutativesquare_C_tl\arrow{u}{\l_nicolas_commutativesquare_g_tl}
			\if \testcodiag t
			\arrow{ur}[near~start]{\l_nicolas_commutativesquare_codiaga_tl}
			\fi
			  \& \l_nicolas_commutativesquare_D_tl\arrow{l}{\l_nicolas_commutativesquare_k_tl}
			  \if \testdiag t
			  \arrow{ul}[near~start,swap,crossing~over]{\l_nicolas_commutativesquare_diaga_tl}
			  \fi
			\arrow{u}[swap]{\l_nicolas_commutativesquare_h_tl}
			\fi
		\fi
		
	\end{tikzcd}
}

\ExplSyntaxOff


\newcommand{\redhomlgy}[2]{
	\tilde{H}_{#1}\lrprth{#2}
}
\newcommand{\copyandpaste}{t}
\newcommand{\moncategory}[2]{Mon_{#1}\lrprth{-,#2}}
\newcommand{\epicategory}[2]{Mon_{#1}\lrprth{#2,-}}
\theoremstyle{definition}
\newtheorem{define}{Definición}
\newtheorem{lem}{Lema}
\newtheorem*{lemsn}{Lema}
\newtheorem{teor}{Teorema}
\newtheorem*{teosn}{Teorema}
\newtheorem{prop}{Proposición}
\newtheorem*{propsn}{Proposición}
\newtheorem{coro}{Corolario}
\newtheorem*{obs}{Observación}

\title{Ejercicios 43-53}
\author{Luis Gerardo Arruti Sebastian\\ Sergio Rosado Zúñiga}
\date{}
\begin{document}
	\maketitle
	\begin{enumerate}[label=\textbf{Ej \arabic*.}]
		\setcounter{enumi}{53}
\item Sea $G:\mathscr{A}\longrightarrow \mathscr{B}$ un funtor contravariante entre categorías abelianas. Pruebe que las siguientes condiciones son
equivalentes:
\begin{itemize}
\item[a)] $G$ es exacto a izquierda (derecha).
\item[b)] $G_{op}:=G\circ D_{\mathscr{A}^{op}}\colon\mathscr{A}^{op}\longrightarrow\mathscr{B}$ es exacto a izquierda (derecha).
\item[c)] $G^{op}:= D_{\mathscr{B}}\circ G\colon\mathscr{A}\longrightarrow\mathscr{B}^{op}$ es exacto a derecha (izquierda).
\end{itemize}
\begin{proof}
\boxed{a)\Rightarrow b)} Supongamos $G$ es exacta a izquierda.\\

Observamos que, como $D_{\mathscr{A}^{op}}$ es contravariante, entonces $G_{op}$ es covariante. Sea 
\xymatrix{0\ar[r]&M_1\ar[r]^{f_1^{op}}&M_2\ar[r]^{f_2^{op}}&M_3} una sucesión exacta en $\mathscr{A}^{op}$. 
Como $\mathscr{A}$ y $\mathscr{B}$ son categorías abelianas en particular son exactas y $\mathscr{A}^{op}$, $\mathscr{B}^{op}$ también lo son, 
así, \xymatrix{M_3\ar[r]^{f_2}&M_2\ar[r]^{f_1}&M_1\ar[r]&0} es exacta en $\mathscr{A}$, y como $G$ es exacto a izquierda, entonces \\
\xymatrix{0\ar[r]&G(M_1)\ar[r]^{G(f_1)}&G(M_2)\ar[r]^{G(f_2)}&G(M_3)} es exacta.\\

Como $G(f_i)=G\circ D_{\mathscr{A}^{op}}(f_i^{op})$ para $i\in \{1,2\}$, y $G(M_j)=G\circ D_{\mathscr{A}^{op}}(M_j)$ para $j\in \{1,2,3\}$, entonces
\xymatrix{0\ar[r]&G_{op}(M_1)\ar[r]^{G_{op}(f_1^{op})}&G_{op}(M_2)\ar[r]^{G_{op}(f_2^{op})}&G_{op}(M_3)} es exacta
 y en consecuencia $G_{op}$ es exacta a izquierda.\\

\boxed{b)\Rightarrow a)} Supongamos $G_{op}$ es exacta a izquierda.\\

Sea \xymatrix{M_3\ar[r]^{f_2}&M_2\ar[r]^{f_1}&M_1\ar[r]&0} una sucesión exacta en $\mathscr{A}$, entonces 
\xymatrix{0\ar[r]&M_1\ar[r]^{f_1^{op}}&M_2\ar[r]^{f_2^{op}}&M_3} es exacta en $\mathscr{A}^{op}$. Como $G_{op}$ es exacta a izquierda
\xymatrix{0\ar[r]&G_{op}(M_1)\ar[r]^{G_{op}(f_1^{op})}&G_{op}(M_2)\ar[r]^{G_{op}(f_2^{op})}&G_{op}(M_3)} es exacta en $\mathscr{B}$,
pero $G_{op}(M_i)=G(M_i)$ para $i\in\{1,2,3\}$ y $G(f_i)=G\circ D_{\mathscr{A}^{op}}(f_i^{op})$ para $i\in \{1,2\}$ entonces 
\xymatrix{0\ar[r]&G(M_1)\ar[r]^{G(f_1)}&G(M_2)\ar[r]^{G(f_2)}&G(M_3)} es exacta en $\mathscr{B}$ y en consecuencia $G$ es exacta.\\

\boxed{a)\Rightarrow c)} Supongamos $G$ es exacta a izquierda.\\

Observemos que $G^{op}$ es covariante por ser $ D_{\mathscr{B}}$ contravariante. Sea  \\ 
\xymatrix{M_1\ar[r]^{f_1}&M_2\ar[r]^{f_2}&M_3\ar[r]&0} una sucesión exacta en $\mathscr{A}$, entonces \\
\xymatrix{0\ar[r]&G(M_3)\ar[r]^{G(f_2)}&G(M_2)\ar[r]^{G{(f_1)}}&G(M_1)} es exacta en $\mathscr{B}$ por ser $G$ exacta, así por 1.7.3
\xymatrix{G(M_1)\ar[r]^{(G(f_1))^{op}}&G(M_2)\ar[r]^{(G(f_2))^{op}}&G(M_3)\ar[r]&0}es exacta en $\mathscr{B}^{op}$.
Pero $ D_{\mathscr{B}}$ manda \xymatrix{B\ar[r]^{f^{op}}&A} en \xymatrix{A\ar[r]^f&B}, entonces \\$G(M_i)=D_{\mathscr{B}}\circ G(M_i)$
para $i\in \{1,2,3\}$ y $D_{\mathscr{B}}\circ G(f_j)=(G(f_j))^{op}$ para $j\in \{1,2\}$, así
\xymatrix{G^{op}(M_1)\ar[r]^{G^{op}(f_1)}&G^{op}(M_2)\ar[r]^{G^{op}(f_2)}&G^{op}(M_3)\ar[r]&0} es exacta en $\mathscr{B}^{op}$, y así
$G^{op}$ es exata a derecha.\\

\boxed{c)\Rightarrow a)} Supongamos $G^{op}$ es exacta a derecha.\\

Sea \xymatrix{M_1\ar[r]^{f_1}&M_2\ar[r]^{f_2}&M_3\ar[r]&0} una sucesión exacta en $\mathscr{A}$ entonces, como $G^{op}$ es exacta a derecha,\\
\xymatrix{G^{op}(M_1)\ar[r]^{G^{op}(f_1)}&G^{op}(M_2)\ar[r]^{G^{op}(f_2)}&G^{op}(M_3)\ar[r]&0} es exacta en $\mathscr{B}^{op}$. \\Asi 
\xymatrix{0\ar[r]&G^{op}(M_3)\ar[r]^{(G^{op}(f_2))^{op}}&G^{op}(M_2)\ar[r]^{(G^{op}(f_1))^{op}}&G^{op}(M_1)} es exacta en $\mathscr{B}$
pero $G(M_i)=D_{\mathscr{B}}\circ G(M_i)$ para $i\in \{1,2,3\}$ y $D_{\mathscr{B}}\circ G(f_j)=(G(f_j))^{op}$ para $j\in \{1,2\}$.
Entonces \xymatrix{0\ar[r]&G(M_3)\ar[r]^{G(f_2)}&G(M_2)\ar[r]^{G{(f_1)}}&G(M_1)} es exacto, es decir, $G$ es exacto a izquierda.\\

 Las equivalencias entre 
\begin{itemize}
\item[a)] $G$ es exacto a derecha.
\item[b)] $G_{op}:=G\circ D_{\mathscr{A}^{op}}\colon\mathscr{A}^{op}\longrightarrow\mathscr{B}$ es exacto a derecha.
\item[c)] $G^{op}:= D_{\mathscr{B}}\circ G\colon\mathscr{A}\longrightarrow\mathscr{B}^{op}$ es exacto a izquierda.
\end{itemize}

se demuestra de manera análoga a lo anterior.

\end{proof}

\item
\item

\item Sea $F:\mathscr{A}\longrightarrow \mathscr{B}$ un funtor entre categorías abelianas. Pruebe que $F$ es exacto izquierdo $\iff$ 
$F_{op}^{op}:=D_{\mathscr{B}}\circ F \circ  D_{\mathscr{A}^{op}}\colon \mathscr{A}^{op}\longrightarrow\mathscr{B}^{op}$ es exacto a 
derecha.
\begin{proof}
Primero observemos que $F_{op}^{op}=(F_{op})^{op}=(F^{op})_{op}\,\\ F=(F^{op})^{op}$\,\,\, y \,\,\, $F=(F_{op})_{op}$. 
Esto pasa por lo siguiente:

\begin{gather*}
(F_{op})^{op}=D_{\mathscr{B}}\circ F_{op}=D_{\mathscr{B}}\circ F\circ D_{\mathscr{A}^{op}}=F_{op}^{op}.\\
(F^{op})_{op}=F^{op}\circ D_{\mathscr{A}^{op}}=D_{\mathscr{B}}\circ F\circ D_{\mathscr{A}^{op}}=F_{op}^{op}.\\
(F^{op})^{op}=D_{\mathscr{B}^{op}}\circ D_{\mathscr{B}}\circ F=1_{\mathscr{B}}F=F.\\
(F_{op})_{op}=F\circ D_{\mathscr{A}^{op}}\circ D_{\mathscr{A}}=F1_\circ D_{\mathscr{A}}=F.
\end{gather*}

Caso 1) $F$ es covariante.\\

En este caso se tiene que, como $D_{\mathscr{C}}:\mathscr{C}\longrightarrow \mathscr{C}^{op}$ es un funtor contravariante para cualquier 
categoría $\mathscr{C}$ , entonces $F_{op}^{op}$ es covariante y $F_{op}$ es contravariante. Como $F_{op}^{op}=(F_{op})^{op}$ entonces
por el ejercicio 54 $F_{op}^{op}$ es exacto a derecha si y sólo si $F_{op}$ es exacto a izquierda, y como $(F_{op})_{op}=F$, entonces 
$F_{op}$ es exacto a izquierda si y sólo si $(F_{op})_{op}=F$ es exacto a izquierda.\\

Caso 2) $F$ es contravariante.\\

En este caso se tiene que $F_{op}^{op}$ es contravariante y $F_{op}$ es covariante. Como $F$ es contravariante entonces por el ejercicio 54
$F_{op}$ es exacto a izquierda si y sólo si $F$ es exacto a izquierda, y como $(F_{op}^{op})^{op}=((F_{op})^{op})^{op}=F^{op}$ entonces 
por el ejercicio 54 $F_{op}^{op}$ es exacto a derecha si y sólo si \\$(F_{op}^{op})^{op}=F^{op}$ es exacto a izquierda. Por lo tanto
$F$ es exacto a izquierda si y sólo si $F_{op}^{op}$ es exacta a derecha.

\end{proof}

\item Para un funtor $F:\mathscr{A}\longrightarrow \mathscr{B}$ entre categorías abelianas, pruebe que las siguientes condiciones son equivalentes.

\begin{itemize}
\item[a)] $F$ es exacto a derecha.
\item[b)] $F$ preserva cokernels, i.e.\\
 $F(Coker(\xymatrix{X\ar[r]^\alpha& Y}))\simeq Coker(\xymatrix{F(X)\ar[r]^{F(\alpha)}& F(Y)})$.
\item[c)] Para toda sucesión exacta \xymatrix{0\ar[r]&K\ar[r]^f&L\ar[r]^g&M\ar[r]&0} en $\mathscr{A}$, se tiene que 
\xymatrix{F(K)\ar[r]^{F(f)}&F(L)\ar[r]^{F(g)}&F(M)\ar[r]&0} es exacta en $\mathscr{B}$.
\end{itemize}
\begin{proof}
\boxed{a)\iff c)} Por el ejercicio 57 se tiene que \\$F_{op}^{op}$ es exacto a izquierda $\iff$ $(F_{op}^{op})^{op}_{op}$ es exacto a derecha,
pero \[(F_{op}^{op})^{op}_{op}=\{[(F_{op})^{op}]^{op}\}_{op}=(f_{op})_{op}=F\] por lo tanto 
 $F_{op}^{op}$ es exacto a izquierda $\iff$ $F$ es exacto a derecha.\\

Así por 1.10.3 las siguientes condiciones son equivalentes:

\begin{itemize}
\item[a*)] $F_{op}^{op}$ es exacto a derecha.
\item[b*)] $F_{op}^{op}$ preserva kernels.
\item[c*)] Para toda sucesión exacta \xymatrix{0\ar[r]&K\ar[r]^f&L\ar[r]^g&M\ar[r]&0} en $\mathscr{A}^{op}$, se tiene que 
\xymatrix{0\ar[r]&F_{op}^{op}(K)\ar[r]^{F_{op}^{op}(f)}&F_{op}^{op}(L)\ar[r]^{F_{op}^{op}(g)}&F_{op}^{op}(M)} es exacta 
en $\mathscr{B}^{op}$.
\end{itemize}

Entonces $F$ es exacta a izquierda si y sólo si \\
para toda sucesión exacta \xymatrix{0\ar[r]&K\ar[r]^f&L\ar[r]^g&M\ar[r]&0} en $\mathscr{A}^{op}$, se tiene que 
\xymatrix{0\ar[r]&F_{op}^{op}(K)\ar[r]^{F_{op}^{op}(f)}&F_{op}^{op}(L)\ar[r]^{F_{op}^{op}(g)}&F_{op}^{op}(M)} es exacta 
en $\mathscr{B}^{op}$ si y sólo si\\
Para toda sucesión exacta \xymatrix{0\ar[r]&M\ar[r]^{g^{op}}&L\ar[r]^{f^{op}}&K\ar[r]&0} en $\mathscr{A}$, se tiene que 
\xymatrix{F_{op}^{op}(M)\ar[r]^{[F_{op}^{op}(g)]^{op}}&F_{op}^{op}(L)\ar[r]^{[F_{op}^{op}(f)]^{op}}&F_{op}^{op}(K)\ar[r]&0} es exacta 
en $\mathscr{B}$.\\

Observando que $F_{op}^{op}(A)=F(A)$ para cada $A\in \mathscr{A}$ y que para cada \\

$\alpha\in Mor( \mathscr{A}^{op})$ 
 $\displaystyle F_{op}^{op}(\alpha)=D_{\mathscr{B}}\circ F\circ D_{\mathscr{A}^{op}}(\alpha)=[F(\alpha^{op})]^{op}$\\

Entonces se tiene que $F$ es exacta a izquierda si y sólo si para toda sucesión exacta 
\xymatrix{0\ar[r]&M\ar[r]^{g^{op}}&L\ar[r]^{f^{op}}&K\ar[r]&0} en $\mathscr{A}$ se tiene que 
\xymatrix{F(M)\ar[r]^{F(g^{op})}&F(L)\ar[r]^{F(f^{op})}&F(K)\ar[r]&0} en $\mathscr{A}$.\\

\boxed{a)\Rightarrow b)} Sea $\alpha:X\to Y$ en $\mathscr{A}$. Luego, se tiene la sucesión exacta 
\xymatrix{X\ar[r]^{\alpha}&Y\ar[r]^{C_\alpha\quad}&Coker(\alpha)\ar[r]&0} en $\mathscr{A}$. Ahora, como $F$ es exacta a derecha tenemos que 
\xymatrix{F(X)\ar[r]^{F(\alpha)}&F(Y)\ar[r]^{F(C_\alpha)\quad}&F(Coker(\alpha))\ar[r]&0} es exacta en $\mathscr{B}$, por lo tanto 
$F(C_\alpha)\simeq CoIm(F_\alpha)\simeq Coker(F_\alpha)$ en $Epi_{\mathscr{B}}(\bullet,F(Y))$.\\

\boxed{b)\Rightarrow c)} Supongamos $F$ preserva cokerneles.\\

Sea \xymatrix{0\ar[r]&K\ar[r]^f&L\ar[r]^g&M\ar[r]&0}  una sucesión exacta en $\mathscr{A}^{op}$. Luego,
$(\xymatrix{M\ar[r]&0})\simeq Coker(\xymatrix{L\ar[r]^g&M})$ \quad y\\
$(\xymatrix{L\ar[r]^g&M})\simeq Coker(\xymatrix{K\ar[r]^f&L})\ldots (*)$ 
por la exactitud en $L$; y como $F$ preserva cokernels, aseguramos que $F(0)=0$. \\

En efecto, como 
$(\xymatrix{0\ar[r]^{1_0}&0})\simeq Coker(\xymatrix{0\ar[r]^{1_0}&0})$, y $F$ preserva cokernels, se tiene que \\
$(\xymatrix{F(0)\ar[r]^{1_{F(0)}}&F(0)})\simeq Coker(\xymatrix{F(0)\ar[r]^{1_{F(0)}}&F(0)})\simeq Coker(\xymatrix{F(0)\ar[r]^0&F(0)})$.
Por lo tanto $1_{F(0)}=0$ y por el ejercicio 51, se tiene que $F(0)=0$.\\

Ahora bien, por $(*)$ y dado que $F$ preserva cokerneles, se tiene que 
\[(\xymatrix{F(M)\ar[r]&0})\simeq Coker(\xymatrix{F(L)\ar[r]^{F(g)}&F(M)})\] y
\[(\xymatrix{F(L)\ar[r]^{F(g)}&F(M)})\simeq Coker(\xymatrix{F(K)\ar[r]^{F(f)}&F(L)})\] de donde se sigue que 
\xymatrix{F(K)\ar[r]^{F(f)}&F(L)\ar[r]^{F(g)}&F(M)\ar[r]&0} es exacta en $\mathscr{B}$.\\

\end{proof}

\item
\item

\item Para un funtor $F:\mathscr{A}\longrightarrow \mathscr{B}$, entre categorías abelianas, pruebe que las siguientes condiciones son equivalentes:

\begin{itemize}
\item[a)] $F$ es exacto.
\item[b)] $\forall \mathscr{D}=\{\xymatrix{A\ar[r]^f&B\ar[r]^g&C}\}$ en $\mathscr{A}$, se tiene que: \\$\mathscr{D}$ es exacto en  $\mathscr{A}$
$\Rightarrow F(\mathscr{D})$ es exacto en $\mathscr{B}$.
\end{itemize}

\begin{proof}



\end{proof}




\item Consideremos el siguiente diagrama conmutativo\\

\centerline{
\xymatrix{
A_1\ar[r]^{u_1}\ar[d]^{f_1}&A_2\ar[r]^{u_2}\ar[d]^{f_2}&A_3\ar[r]^{u_3}\ar[d]^{f_3}&A_4\ar[r]^{u_4}\ar[d]^{f_4}&A_5\ar[d]^{f_5}\\
A_1'\ar[r]^{u_1'}&A_2'\ar[r]^{u_2'}&A_3'\ar[r]^{u_3'}&A_4'\ar[r]^{u_4'}&A'_5
}}

y con filas exactas en una categoría abeliana $\mathscr{A}$. Pruebe que:

\begin{itemize}
\item[a)] Si $f_2$ y $f_4$ son monos y $f_1$ es epi, entonces $f_3$ es mono.
\item[b)] Si $f_2$ y $f_4$ son epis y $f_5$ es mono, entonces $f_3$ es epi.
\end{itemize}

\begin{proof}
Por el teorema 1.10.9 existe una subcategoría abeliana $\mathscr{A}'$ de $\mathscr{A}$ tal que $\mathscr{A}'$ es subcategoría plena y pequeña
de $\mathscr{A}$ y $\mathscr{D}\subseteq \mathscr{A}'$. Por 1.10.10 existe un anillo $R$ y un funtor fiel, pleno y exacto \\$F:\mathscr{A}'
\longrightarrow Mod(R)$.\\

Considerando el diagrama $F(\mathscr{D})$ en $Mod(R)$

\centerline{
\xymatrix{
F(A_1)\ar[r]^{F(u_1)}\ar[d]^{F(f_1)}&F(A_2)\ar[r]^{F(u_2)}\ar[d]^{F(f_2)}&
F(A_3)\ar[r]^{F(u_3)}\ar[d]^{F(f_3)}&F(A_4)\ar[r]^{F(u_4)}\ar[d]^{F(f_4)}&F(A_5)\ar[d]^{F(f_5)}\\
F(A_1')\ar[r]^{F(u_1')}&F(A_2')\ar[r]^{F(u_2')}&F(A_3')\ar[r]^{F(u_3')}&F(A_4')\ar[r]^{F(u_4')}&F(A_5')
}}

por 1.10.7 se tiene que $F(\mathscr{D})$ es conmutativo y con filas exactas en $Mod(R)$.\\
Veamos que a) y b) se cumplen para el diagrama en $Mod(R)$.\\

\boxed{a)} Si $F(f_2)$ y $F(f_4)$ son monos y $F(f_1)$ es epi afirmamos que \\$Ker(F(f_3))=0$.\\

Sea $x\in Ker(F(f_3))\leq F(A_3)$, en particular $F(f_4)F(u_3)(x)=F(u'_3)F(f_3)(x)$ el cual es $0$, entonces $x\in Ker(F(f_4)F(u_3))=Ker(F(u_3))$ pues 
$F(f_4)$ es mono, así, como los renglones son exactos, $x\in Ker(F(u_3))=Im(F(u_2))$ y por lo tanto existe $y\in F(A_2)$ tal que $F(u_2)(y)=x$.\\

Ahora, $F(f_3)(x)=0$ entonces $0=F(f_3)F(u_2)(y)=F(u'_2)F(f_2)(y)$ por lo que $y\in Ker(F(u'_2)F(f_2))$ entonces, como $F(f_2)$ es mono,\\
$F(f_2)(y)\in Ker(F(u_2'))=Im(F(u'_1)).$ Por lo anterior, se tiene entonces que $\exists z'\in F(A'_1)$ tal que $F(u_1')(z')=F(f_2)(y)$, y como 
$F(f_1)$ es epi, entonces existe $z\in F(A_1)$ tal que $F(f_1)(z)=z'$, es decir, \\$F(u_1')F(f_1)(z)=F(f_2)(y)$ y esto implica que $F(f_2)(y)
=F(f_2)F(u_1)(z)$ pero $F(f_2)$ es mono en $Mod(R)$, entonces es inyectivo, por lo que \\$y=F(u_1)(z)$ y así $x=F(u_2)F(u_1)(z)=0$.\\
Por lo tanto $F(f_3)$ es mono.\\

\boxed{b)} Dado un diagrama en $\mathscr{A}$ como se muestra en las hipótesis se tiene que el siguiente diagrama es un diagrama conmutativo
con renglones exactos en $\mathscr{A}^{op}$

\centerline{
\xymatrix{
A_5'\ar[d]^{f^{op}_5}\ar[r]^{u_4'^{op}}&A_4'\ar[d]^{f^{op}_4}\ar[r]^{u_3'^{op}}&
A_3'\ar[d]^{f^{op}_3}\ar[r]^{u_2'^{op}}&A_2'\ar[d]^{f^{op}_2}\ar[r]^{u_1'^{op}}&A_1'\ar[d]^{f^{op}_1}\\
A_5\ar[r]^{u_4^{op}}&A_4\ar[r]^{u_3^{op}}&A_3\ar[r]^{u_2^{op}}&A_2\ar[r]^{u_1^{op}}&A_1
}}

Ahora, si $f_2$ y $f_4$ son epi y $f_5$ es mono, entonces $f_2^{op}$ y $f_4^{op}$ son monos y $f_5^{op}$ es epi, así por el inciso a) se tiene 
que $f_3^{op}$ es mono lo cual implica que $f_3$ es epi.

\end{proof}

\item
\item

\item Pruebe que para una sucesión exacta \xymatrix{0\ar[r]&A\ar[r]^{\alpha}&B\ar[r]^{\beta}&C\ar[r]&0} en una categoría abeliana $\mathscr{A}$
y $\gamma \in \operatorname{Hom}_{\mathscr{A}}(A,A')$, las siguientes condiciones se satisfacen:

\begin{itemize}
\item[a)] Si $(B',\alpha',\gamma')$ es un push-out de \xymatrix{A'&A\ar[l]_{\gamma}\ar[r]^{\alpha}&B}, entonces existe $\beta':B'\to C$ tal que 
hace conmutar el siguiente diagrama\\
\centerline{
\xymatrix{
0\ar[r]&A\ar[d]^{\gamma}\ar[r]^{\alpha}&B\ar[d]^{\gamma'}\ar[r]^{\beta}&C\ar@{=}[d]\ar[r]&0\\
0\ar[r]&A'\ar[r]^{\alpha'}&B'\ar[r]^{\beta'}&C'\ar[r]&0& \ldots (*)
}}
en $\mathscr{A}$, cuyas filas son sucesiones exactas.
\item[b)] Si se tiene un diagrama conmutativo como en (*), con filas exactas, entonces $(B',\alpha',\gamma')$ es un push-out de
 \xymatrix{A'&A\ar[l]_{\gamma}\ar[r]^{\alpha}&B}.
\end{itemize}

\begin{proof}

Puesto que $\mathscr{A}$ es abeliana y  \xymatrix{0\ar[r]&A\ar[r]^{\alpha}&B\ar[r]^{\beta}&C\ar[r]&0} es exacta, se tiene entonces que 
\xymatrix{0\ar[r]&C\ar[r]^{\beta^{op}}&B\ar[r]^{\alpha^{op}}&A\ar[r]&0} es exacta en $\mathscr{A}^{op}$, además por hipótesis 
$\gamma\in \operatorname{Hom}_\mathscr{A}(A,A')$, entonces \\$\gamma^{op}\in \operatorname{Hom}_{\mathscr{A}^{op}}(A',A)$.\\

Sea $(B',\alpha',\gamma')$ un push-out de \xymatrix{A'&A\ar[l]_{\gamma}\ar[r]^{\alpha}&B}, entonces 
$(B',(\alpha')^{op},(\gamma')^{op})$ es un pull-back de \xymatrix{A'\ar[r]^{\gamma^{op}}&A&B\ar[l]_{\alpha^{op}}}, así por 1.10.16
existe \\$(\beta')^{op}:C\to B'$ tal que hace conmutar el siguiente diagrama:\\

\centerline{
\xymatrix{
0\ar[r]&C\ar@{=}[d]\ar[r]^{(\beta')^{op}}&B'\ar[r]^{(\alpha')^{op}}\ar[d]^{(\gamma')^{op}}&A'\ar[r]\ar[d]^{\gamma^{op}}&0\\ 
0\ar[r]&C\ar[r]^{\beta^{op}}&B\ar[r]^{\alpha^{op}}&A\ar[r]&0&\ldots (1)
}}

en $\mathscr{A}^{op}$ cuyas silas son exactas.\\

Así, $\beta':B'\to C$ es tal que hace conmutar el siguiente diagrama en $\mathscr{A}$, cuyas filas son exactas\\

\centerline{
\xymatrix{
0\ar[r]&A\ar[d]^{\gamma}\ar[r]^{\alpha}&B\ar[d]^{\gamma'}\ar[r]^{\beta}&C\ar@{=}[d]\ar[r]&0\\
0\ar[r]&A'\ar[r]^{\alpha'}&B'\ar[r]^{\beta'}&C'\ar[r]&0\,\,.
}}

Veamos ahora que se cumple b). Si tenemos un diagrama conmutativo como en (*) con filas exactas en $\mathscr{A}$, entonces tenemos un diagrama
conmutativo con filas exactas en $\mathscr{A}^{op}$ como se muestra en (1). Así, por 1.10.16 $(B',(\alpha')^{op},(\gamma')^{op})$
 es un pull-back de \xymatrix{A'\ar[r]^{\gamma^{op}}&A&B\ar[l]_{\alpha^{op}}} por lo que $(B',\alpha',\gamma')$ es un push-out de
 \xymatrix{A'&A\ar[l]_{\gamma}\ar[r]^{\alpha}&B} en $\mathscr{A}$.

\end{proof}

\item Para una categoría abeliana $\mathscr{A}$, pruebe que las siguientes condiciones son equivalentes.

\begin{itemize}
\item[a)] El diagrama \xymatrix{ A\ar[r]^{\alpha_2}\ar[d]_{\alpha_1}&A_2\ar[d]^{\beta_2}\\
A_1\ar[r]_{\beta_1}&Q}
es un Push-out en $\mathscr{A}$.

\item[b)] La sucesión \begin{equation*}
\xymatrix@C+2pc{
 A \ar[r]^{\spmat{\alpha_1\\ -\alpha_2}}  & A_1\coprod A_2\ar[r]^{\spmat{\beta_1&\beta_2}}& Q\ar[r]&0}
\end{equation*}
es exacta en  $\mathscr{A}$.
\end{itemize}

\begin{proof}

El diagrama \xymatrix{ A\ar[r]^{\alpha_2}\ar[d]_{\alpha_1}&A_2\ar[d]^{\beta_2}\\
A_1\ar[r]_{\beta_1}&Q}
es un Push-out en $\mathscr{A}$ si y sólo si el diagrama\\
 \xymatrix{ Q\ar[r]^{(\beta_2)^{op}}\ar[d]_{(\beta_1)^{op}}&A_2\ar[d]^{(\alpha_2)^{op}}\\
A_1\ar[r]_{(\alpha_1)^{op}}&A}
es un Pull-back en $\mathscr{A}^{op}$ lo cual, por la proposición 1.10.17, es equivalente a que la sucesión 

\begin{equation*}
\xymatrix@C+4pc{
 0\ar[r]&Q \ar[r]^{\spmat{\beta_1^{op}\\ \beta_2^{op}}}  & A_1\coprod A_2\ar[r]^{\spmat{\alpha_1^{op}& (-\alpha_2)^{op}}}& A}
\end{equation*}

es exacta en $\mathscr{A}^{op}$. Pero esto pasa si y sólo si

 \begin{equation*}
\xymatrix@C+4pc{
A \ar[r]^{\spmat{\alpha_1^{op}& (-\alpha_2)^{op}}^{op}} & A_1\prod A_2\ar[r]^{\spmat{\beta_1^{op}\\ \beta_2^{op}}^{op}}
&Q\ar[r]&0}
\end{equation*}

es exacta en $\mathscr{A}$ si y sólo si 
\begin{equation*}
\xymatrix@C+2pc{
 A \ar[r]^{\spmat{\alpha_1\\ -\alpha_2}}  & A_1\coprod A_2\ar[r]^{\spmat{\beta_1&\beta_2}}& Q\ar[r]&0}
\end{equation*}
es exacta en  $\mathscr{A}$.

\end{proof}
\item
\item\,\\


\textbf{Ej 68*.} Sean $\mathscr{A}$ una categoría abeliana y $M\in \mathscr{A}$. Pruebe que 

\begin{itemize}
\item[a)] $M\in Proj(\mathscr{A})\iff \operatorname{Hom}_\mathscr{A}(M,\bullet):\mathscr{A}\to Ab$ es exacto.
\item[a)] $M\in Inj(\mathscr{A})\iff \operatorname{Hom}_\mathscr{A}(\bullet,M):\mathscr{A}\to Ab$ es exacto.
\end{itemize}
\begin{proof}
Observemos que $\operatorname{Hom}_\mathscr{A}(M,\bullet)$ y $\operatorname{Hom}_\mathscr{A}(\bullet,M)$ son aditivos y exactos a izquierda
por el ejercicio 55.\\

\boxed{a), \Rightarrow)} Supongamos $M\in Proj(\mathscr{A})$ y sea \xymatrix{0\ar[r]&A\ar[r]^{f}&B\ar[r]^{g}&C\ar[r]^{}&0} \\una sucesión
exacta en $\mathscr{A}$. Como $\operatorname{Hom}_\mathscr{A}(M,\bullet)$ es exacta a izquierda, se tiene que \\
\centerline{
\xymatrix@C+1pc{
0\ar[r]&\operatorname{Hom}_\mathscr{A}(M,A)\ar[r]^{\operatorname{Hom}_\mathscr{A}(M,f)}&\operatorname{Hom}_\mathscr{A}(M,B) 
\ar[r]^{\operatorname{Hom}_\mathscr{A}(M,g)}&\operatorname{Hom}_\mathscr{A}(M,C)
}}
es exacta en $Ab$.\\

 Basta mostrar que $\operatorname{Hom}_\mathscr{A}(M,g)$ es epi en $Ab$. Mostraremos que, de hecho, 
$\operatorname{Hom}_\mathscr{A}(M,g)$ es suprayectiva.\\

Sea $\alpha\in\operatorname{Hom}_\mathscr{A}(M,C)$, como \xymatrix{0\ar[r]&A\ar[r]^{f}&B\ar[r]^{g}&C\ar[r]^{}&0} es exacta, se tiene que $g$ es 
epi y $g\in \operatorname{Hom}_\mathscr{A}(B,C)$ así, como $M$ es proyectivo, existe un $\eta\in \operatorname{Hom}_\mathscr{A}(M,B)$ tal
que $g\eta=\alpha$, por lo tanto $\operatorname{Hom}_\mathscr{A}(M,g)$ es suprayectivo y en particular es epi, por lo tanto la sucesión \\

\xymatrix@C+1pc{
0\ar[r]&\operatorname{Hom}_\mathscr{A}(M,A)\ar[r]^{\operatorname{Hom}_\mathscr{A}(M,f)}&\operatorname{Hom}_\mathscr{A}(M,B) 
\ar[r]^{\operatorname{Hom}_\mathscr{A}(M,g)}&\operatorname{Hom}_\mathscr{A}(M,C)\ar[r]&0
}
es exacta en $Ab$.\\

\boxed{a), \Leftarrow)} Supongamos ahora que $\operatorname{Hom}_\mathscr{A}(M,\bullet)$ es exacto. Sea $h\in
\operatorname{Hom}_\mathscr{A}(M,C)$ y $\gamma\in \operatorname{Hom}_\mathscr{A}(X,C)$ en epi, entonces tenemos el siguiente 
diagrama con renglón exacto\\
\centerline{
\xymatrix{
&M\ar[d]^{\eta}&\\
X\ar@{->>}[r]_{\gamma}&C\ar[r]&0\,.
}}
Como $\operatorname{Hom}_\mathscr{A}(M,\bullet)$ es exacto, entonces \\
\centerline{
\xymatrix@C+1pc{\operatorname{Hom}_\mathscr{A}(M,X)\ar[r]_{\operatorname{Hom}_\mathscr{A}(M,g)}&
\operatorname{Hom}_\mathscr{A}(M,C)\ar[r]&0
}}
es exacto en $Ab$, por lo tanto $\operatorname{Hom}_\mathscr{A}(M,g)$ es epi y en consecuencia suprayectivo. Así como $\eta\in 
\operatorname{Hom}_\mathscr{A}(M,C)$, existe $f\in \operatorname{Hom}_\mathscr{A}(M,X)$ tal que 
$\operatorname{Hom}_\mathscr{A}(M,\gamma)(f)=\eta$ es decir, $\gamma f=\eta$, por lo tanto $M$ es proyectivo.\\

\boxed{b), \Rightarrow)} Supongamos $M$ es inyectivo en $\mathscr{A}$. Sea \\
\xymatrix{0\ar[r]&A\ar[r]^{f}&B\ar[r]^{g}&C\ar[r]^{}&0} una sucesión exacta en $\mathscr{A}$. Como \\$\operatorname{Hom}_\mathscr{A}(\bullet,M)$
es un funtor exacto izquierdo contravariante, por el ejercicio 55, entonces \\

\centerline{
\xymatrix@C+1pc{
0\ar[r]&\operatorname{Hom}_\mathscr{A}(C,M)\ar[r]^{\operatorname{Hom}_\mathscr{A}(g,M)}&
\operatorname{Hom}_\mathscr{A}(B,M)\ar[r]^{\operatorname{Hom}_\mathscr{A}(f,M)}&
\operatorname{Hom}_\mathscr{A}(A,M)
}}

es exacta. Basta probar entonces que $\operatorname{Hom}_\mathscr{A}(f,M)$ es suprayectivo.\\

Sea $\eta\in \operatorname{Hom}_\mathscr{A}(A,M)$, como \xymatrix{0\ar[r]&A\ar[r]^{f}&B\ar[r]^{g}&C\ar[r]^{}&0} una sucesión exacta en 
$\mathscr{A}$, en particular $f$ es mono así, como $M$ es inyectivo, se tiene entonces que $\exists \gamma \in  \operatorname{Hom}_\mathscr{A}(B,M)$
tal que $\eta=\gamma f$ por lo tanto $ \operatorname{Hom}_\mathscr{A}(f,M)$ es suprayectivo, en particular es epi. Así\\
\centerline{
\xymatrix@C+1pc{
0\ar[r]&\operatorname{Hom}_\mathscr{A}(C,M)\ar[r]^{\operatorname{Hom}_\mathscr{A}(g,M)}&
\operatorname{Hom}_\mathscr{A}(B,M)\ar[r]^{\operatorname{Hom}_\mathscr{A}(f,M)}&
\operatorname{Hom}_\mathscr{A}(A,M)
}}

es exacta en $Ab$ lo que implica que $\operatorname{Hom}_\mathscr{A}(\bullet,M)$ es exacto.\\

\boxed{b), \Leftarrow)}  Supongamos  $\operatorname{Hom}_\mathscr{A}(\bullet,M)$ es exacto. Si 
$\eta\in \operatorname{Hom}_\mathscr{A}(A,M)$ y $\alpha\in \operatorname{Hom}_\mathscr{A}(A,X)$ es mono, entonces  se tiene el 
siguiente diagrama con renglón exacto\\

\centerline{
\xymatrix{
0\ar[r]&A\ar[d]_{\eta}\ar[r]^{\alpha}&X\\
&M
}}

como $ \operatorname{Hom}_\mathscr{A}(\bullet,M)$ es exacto, entonces \\

\centerline{
\xymatrix@+1pc{
 \operatorname{Hom}_\mathscr{A}(X,M)\ar[r]^{ \operatorname{Hom}_\mathscr{A}(\alpha,M)}&
 \operatorname{Hom}_\mathscr{A}(A,M)\ar[r]&0
}}
es exacto, entonces $ \operatorname{Hom}_\mathscr{A}(\alpha,M)$ es epi en $Ab$ por lo tanto es suprayectivo.\\

Así, como $\eta\in  \operatorname{Hom}_\mathscr{A}(A,M)$, \quad $\exists \gamma\in  \operatorname{Hom}_\mathscr{A}(X,M)$ tal que\\
$\eta= \operatorname{Hom}_\mathscr{A}(\alpha,M)(\gamma)=\gamma\circ \alpha$. Por lo tanto $M$ es proyectivo.

\end{proof}

\item Sea $\mathscr{A}$ una categoría abeliana y $\catarrow{\beta}{I}{M}{e}$ un split-epi en $\mathscr{A}$, con \\
$I\in Inj(\mathscr{A})$. Pruebe que $M\in Inj(\mathscr{A})$.

\begin{proof}
Sea $\alpha:M\to I$ tal que $\beta\alpha=1_M$, este morfismo existe por ser $\beta$ split-epi, y consideremos un diagrama en $\mathscr{A}$ de la
forma\\

\centerline{
\xymatrix{
A\ar[d]_{g}\ar@{^(->}[r]^{f}&B\\ M\,.
}}
Como $\alpha\in  \operatorname{Hom}_\mathscr{A}(M,I)$, entonces $\alpha g:A\to I$. Como $I$ es inyectivo en $\mathscr{A}$ existe 
$\eta:B\to I$ tal que $\alpha g=\eta f$, entonces $\beta\eta f=\beta\alpha g=1_M g=g.$ Por lo tanto $M$ es inyectivo en $A$.

\end{proof}

\item Para una categoría abeliana $A$ e $I\in A$, pruebe las siguientes condiciones equivalentes

\begin{itemize}
\item[a)] $I\in Inj(\mathscr{A})$.
\item[b)] Todo mono $\alpha:I\to X$ en $\mathscr{A}$ es split-mono.
\end{itemize}

\begin{proof}
\boxed{a)\Rightarrow b)} Supongamos $I$ es inyectivo en $\mathscr{A}$ y $\alpha:I\to X$ es mono, entonces se tiene el siguiente diagrama\\

\centerline{
\xymatrix{
I\ar[d]_{Id_I}\ar@{^(->}[r]^{\alpha}&X\\ I& .
}}

Como $I$ es inyectivo $\exists \beta:X\to I$ tal que $\beta\alpha=Id_I$, es decir, $\alpha$ es split-mono.\\

\boxed{b)\Rightarrow a)} Supongamos que todo mono $\alpha:I\to X$ en $\mathscr{A}$ es split-mono.\\

Consideremos el diagrama en $\mathscr{A}$\\

\centerline{
\xymatrix{
A\ar[d]_{\eta}\ar@{^(->}[r]^{f}&B\\ I& .
}}

Como  $\mathscr{A}$ es abeliana, existe el Push-out de $f$ y $\eta$\\


\centerline{
\xymatrix{
A\ar[d]_{\eta}\ar@{^(->}[r]^{f}&B\ar[d]^{\alpha_1}\\ I\ar[r]_{\alpha_2}&P& .
}}

Por el ejercicio 64 a) se tiene que, como $f$ es mono, $\alpha_2$ es mono, y por el ejercicio 64 b), que $\eta$ se factoriza a travéz de $f$ si y sólo si
$\alpha_2$ es split-mono. Pero por hipótesis $\alpha_2$ al ser mono, tiene que ser split-mono, por lo tanto $\eta$ se factoriza a travéz de $f$ 
e implica que $I$ es inyectivo.

\end{proof}










\end{enumerate}
\end{document}